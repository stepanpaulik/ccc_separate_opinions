% ARTICLE 2 ----
% This is just here so I know exactly what I'm looking at in Rstudio when messing with stuff.
% Options for packages loaded elsewhere
\PassOptionsToPackage{unicode}{hyperref}
\PassOptionsToPackage{hyphens}{url}
%
\documentclass[
  11pt,
]{article}
\usepackage{lmodern}
\usepackage{amssymb,amsmath}
\usepackage{ifxetex,ifluatex}
\ifnum 0\ifxetex 1\fi\ifluatex 1\fi=0 % if pdftex
  \usepackage[T1]{fontenc}
  \usepackage[utf8]{inputenc}
  \usepackage{textcomp} % provide euro and other symbols
\else % if luatex or xetex
  \usepackage{unicode-math}
  \defaultfontfeatures{Scale=MatchLowercase}
  \defaultfontfeatures[\rmfamily]{Ligatures=TeX,Scale=1}
\fi
% Use upquote if available, for straight quotes in verbatim environments
\IfFileExists{upquote.sty}{\usepackage{upquote}}{}
\IfFileExists{microtype.sty}{% use microtype if available
  \usepackage[]{microtype}
  \UseMicrotypeSet[protrusion]{basicmath} % disable protrusion for tt fonts
}{}
\makeatletter
\@ifundefined{KOMAClassName}{% if non-KOMA class
  \IfFileExists{parskip.sty}{%
    \usepackage{parskip}
  }{% else
    \setlength{\parindent}{0pt}
    \setlength{\parskip}{6pt plus 2pt minus 1pt}
    }
}{% if KOMA class
  \KOMAoptions{parskip=half}}
\makeatother
\usepackage{xcolor}
\IfFileExists{xurl.sty}{\usepackage{xurl}}{} % add URL line breaks if available
\urlstyle{same} % disable monospaced font for URLs
\usepackage[margin=1in]{geometry}
\setlength{\emergencystretch}{3em} % prevent overfull lines
\providecommand{\tightlist}{%
  \setlength{\itemsep}{0pt}\setlength{\parskip}{0pt}}
\setcounter{secnumdepth}{5}

\ifluatex
  \usepackage{selnolig}  % disable illegal ligatures
\fi


\title{„I have spoken and saved my soul``: a qualitative analysis of
Czech constitutional judges dissenting behaviou}
\author{true \and true}
\date{November 12, 2023}

% Jesus, okay, everything above this comment is default Pandoc LaTeX template. -----
% ----------------------------------------------------------------------------------
% I think I had assumed beamer and LaTex were somehow different templates.


\usepackage{kantlipsum}

\usepackage{abstract}
\renewcommand{\abstractname}{}    % clear the title
\renewcommand{\absnamepos}{empty} % originally center

\renewenvironment{abstract}
 {{%
    \setlength{\leftmargin}{0mm}
    \setlength{\rightmargin}{\leftmargin}%
  }%
  \relax}
 {\endlist}

\makeatletter
\def\@maketitle{%
  \newpage
%  \null
%  \vskip 2em%
%  \begin{center}%
  \let \footnote \thanks
      {\fontsize{18}{20}\selectfont\raggedright  \setlength{\parindent}{0pt} \@title \par}
    }
%\fi
\makeatother


\title{„I have spoken and saved my soul``: a qualitative analysis of
Czech constitutional judges dissenting behaviou }

\date{}

\usepackage{titlesec}

% 
\titleformat*{\section}{\large\bfseries}
\titleformat*{\subsection}{\normalsize\itshape} % \small\uppercase
\titleformat*{\subsubsection}{\normalsize\itshape}
\titleformat*{\paragraph}{\normalsize\itshape}
\titleformat*{\subparagraph}{\normalsize\itshape}

% add some other packages ----------

% \usepackage{multicol}
% This should regulate where figures float
% See: https://tex.stackexchange.com/questions/2275/keeping-tables-figures-close-to-where-they-are-mentioned
\usepackage[section]{placeins}



\makeatletter
\@ifpackageloaded{hyperref}{}{%
\ifxetex
  \PassOptionsToPackage{hyphens}{url}\usepackage[setpagesize=false, % page size defined by xetex
              unicode=false, % unicode breaks when used with xetex
              xetex]{hyperref}
\else
  \PassOptionsToPackage{hyphens}{url}\usepackage[draft,unicode=true]{hyperref}
\fi
}

\@ifpackageloaded{color}{
    \PassOptionsToPackage{usenames,dvipsnames}{color}
}{%
    \usepackage[usenames,dvipsnames]{color}
}
\makeatother
\hypersetup{breaklinks=true,
            bookmarks=true,
            pdfauthor={Štěpán Paulík (Humboldt Universität zu Berlin,
\href{mailto:stepan.paulik.1@hu-berlin.de}{\nolinkurl{stepan.paulik.1@hu-berlin.de}}) and Gor
Vartazaryan (Charles University,
\href{mailto:gorike2000@gmail.com}{\nolinkurl{gorike2000@gmail.com}})},
             pdfkeywords = {dissent, dissent aversion, constitutional
court, decision-making, qualiative research},
            pdftitle={„I have spoken and saved my soul``: a qualitative
analysis of Czech constitutional judges dissenting behaviou},
            colorlinks=true,
            citecolor=blue,
            urlcolor=blue,
            linkcolor=magenta,
            pdfborder={0 0 0}}
\urlstyle{same}  % don't use monospace font for urls

% Add an option for endnotes. -----



% This will better treat References as a section when using natbib
% https://tex.stackexchange.com/questions/49962/bibliography-title-fontsize-problem-with-bibtex-and-the-natbib-package

% set default figure placement to htbp
\makeatletter
\def\fps@figure{htbp}
\makeatother



\usepackage{longtable}
\LTcapwidth=.95\textwidth
\linespread{1.05}
\usepackage{hyperref}
\usepackage{float}

\newtheorem{hypothesis}{Hypothesis}

\usepackage{setspace}

% trick for moving figures to back of document
% really wish we'd knock this shit off with moving tables/figures to back of document
% but, alas...

% 
% Optional code chunks ------
% SOURCE: https://stackoverflow.com/questions/50702942/does-rmarkdown-allow-captions-and-references-for-code-chunks



\begin{document}

% \textsf{\textbf{This is sans-serif bold text.}}
% \textbf{\textsf{This is bold sans-serif text.}}


% \maketitle

{% \usefont{T1}{pnc}{m}{n}
\setlength{\parindent}{0pt}
\thispagestyle{plain}
{%\fontsize{18}{20}\selectfont\raggedright
\maketitle  % title \par

}




{
   \vskip 13.5pt\relax \normalsize\fontsize{11}{12}
   \MakeUppercase{Štěpán Paulík}, \small{Humboldt Universität zu Berlin,
\href{mailto:stepan.paulik.1@hu-berlin.de}{\nolinkurl{stepan.paulik.1@hu-berlin.de}}}   \par \vskip -3.5pt \MakeUppercase{Gor
Vartazaryan}, \small{Charles University,
\href{mailto:gorike2000@gmail.com}{\nolinkurl{gorike2000@gmail.com}}}   

}

}








\begin{abstract}

%    \hbox{\vrule height .2pt width 39.14pc}

    \vskip 8.5pt % \small

\noindent \small{Dissent presents an oportunity for judges to break away
from the majority opinion and express their stance. Doctrinal research
has presented many views on positive and negative aspects of the
possibility to dissent. Research on judicial decision-making explained a
various aspects of dissent aversion and its effect in the courts of
various instances. The rule of thumb is that judges dissent most in the
superior courts. This paper contributes to existing research by
analysing the attitudes of judges towards dissent. A thematic analysis
of semi-structured interview with Czech constitutional court judges
(N=9) of the third decade was conducted. Our results present the process
od dissent making and the role of opinion leaders in the proces. We also
reveal that there are three attitudinal groups of dissenting justices -
haters, fans and strategist. The interviews revealed that four factors
primarily influence CCC judges: prior professional experience,
regulation of one's own emotions and frustrations, caseload and
importance of the particular case.}


\vskip 8.5pt \noindent \emph{Keywords}: dissent, dissent aversion,
constitutional court, decision-making, qualiative research \par

%    \hbox{\vrule height .2pt width 39.14pc}



\end{abstract}


\vskip -8.5pt

{
\hypersetup{linkcolor=black}
\setcounter{tocdepth}{2}
\tableofcontents
}

 % removetitleabstract

{
\setcounter{tocdepth}{2}
\tableofcontents
}

\setlength{\parindent}{16pt}
\setlength{\parskip}{0pt}

% We'll put doublespacing here
\doublespacing
% Remember to cut it out later before bib
\hypertarget{introduction}{%
\section{Introduction}\label{introduction}}

In collegial court judges can disagree for many reasons in heated
debates. Some courts enable judges to dissent and publicly express a
different opinion (For the purposes of this article, concurring
judgments that are written separately from the reasons of the majority
will be treated the same as dissenting judgments). This decision-making
question, „why judges dissent?{}``, has attracted the attention of
scholars in many fields: political scientis, lega scholars sociologist
(Blackstone \& Collins, 2014; Cross, 2007; Epstein et al., 2011, 2013;
Garoupa \& Botelho, 2022; Posner, 1993; Sunstein et al., 2006). There
are many empirical approaches for judicial-decision explanation: the
rational-choice theory, principal-agent model, explanation based on
legal culture and others. In empirical research models are based on
doctrinal theory or legal philosophy that brings hypotheses of judicial
dissent making. Research in US suggest to use indirect method to
understand judicial behaviour since judges are permmited to be, and most
are, quite secretive (Epstein et al., 2013). This paper challenge this
view in different legal culture and present a qualitative approach that
would explore how judges perceive dissent.\\
Our previous research replicated rational-choice theory on the Czech
Constitutional Court (Paulík \& Vartazaryan, 2023). We decided to
further investigate why, when and how judges decide to break from the
majority opinion. Qualitative approach of the Czech Constitutional Court
(„CCC``) brings new insights into the decision-making process of judges.
We choose CCC for four main reasons: firstly, CCC allowes judges to
dissent, however it is not an obligation. The vote is not public, so if
a judge votes against a decision but does not exercise a dissent, there
is no way for the public to know. This allows judges to perform
so-called dissent aversion (Epstein et al., 2011, 2013; Posner, 2010).
From previous interview with CCC judges many confirmed that they do so
(Kysela et al., 2015; Smekal et al., 2021). Secondly, judges can dissent
solely or jointy with other judges. The question of dissenting coalition
is less research topic so far. Thirdly, the literature suggest that the
third decade of CCC presents the most polarized period in terms of
voting or dissenting coalitions (Chmel, 2021; Smekal et al., 2021;
Vartazaryan, 2022) and finally, judges of the third decade experienced a
period before and after the implementation of chamber rotation, which
influenced the panel-compositio of each chamber of CCC. This paper
presents the proces of dissenting making and the the importance of
unwritten rules, the typology of dissenting Justices and four main
themes of explanation of their dissenting behaviour.

\hypertarget{context-and-literature-review}{%
\section{Context and Literature
review}\label{context-and-literature-review}}

In this section, we situate our research in the existing literature.
First, we present the normative and positive theory and then we further
investigate in existing literature on dissenting behaviour.
Subsequently, to give the reader sufficient context, we present
back-ground information on the CCC and its Justices.

\hypertarget{dissenting-behaviour-of-judges-in-literature}{%
\subsection{Dissenting behaviour of judges in
literature}\label{dissenting-behaviour-of-judges-in-literature}}

In theory there are two main approaches towards dissents. We can
research the advantages and disadvantages of dissenting - the normative
theory or we can focuses on the explanation, why judges dissent - the
positive theory (Garoupa \& Botelho, 2022). However, these two
approaches are not separate since judges can be influenced by normative
and positive factors and vice versa. In theory, judges can benefit from
the dissent but always at some cost. The benefit is that dissent can
undermine the influence of the majority, draw their attention, bring
reputation, be a factor for overruling in the future etc. On the other
hand it cost an effort it can possibly fray the collegiality between
judges, it takes a time to write a dissent and the caseload my be a
factor for judges and some may say that it can burden reputation or also
it can undermine the authority of the Court. That said, Peterson presens
the influences cause and impact of a dissent which he categorize to four
parts: a) legal culture, b) organizational and institutional factors, c)
socio-political system and d) individuals (Peterson, 1981). He also
presetens set of hypothesis derived from those categories that were
mostly not test at that time. For example he proposes that (i) size of
the court is directly related to the likelihood, (ii) the greater the
workload, the less the dissent , (iii) greater the diversity among
judges on a court, the higher will be dissent level. Most of those
hypotheses were already empirically tested. Most empirical models are
testing hypotheses taken from doctrinal research or it is based on a
vision of the judge as a rational, self-interested utility maximizer
(Posner, 2010). Empirical research explored the phenomenon of dissent
aversion - motivation of judges not to dissent. Epstein et al.~tested
hypothesis on Court of Appleas and Supreme Court. They found out that
dissents are negatively related to the caseload and positively related
to ideological diversity among judges in the circuit and circuit size in
the court of appeals. In the Supreme Court, the dissent rate is
negatively related to the caseload and positively related to ideological
differences, that majority opinions are longer when there is a dissent
(Epstein et al., 2011). However not all of those hypotheses and models
are replicable at different legal context. In some CCE countries,
information about voting is not public, also the process of appointment
of judges differs and it affect the politicization of their
decision-making. Our replication on the CCC was methodologically
limited, since in CCC data of judges voting are not public. We proved
that a dissent opinion imposes costs on the majority that produces
longer arguments to address a dissent. The effect is stronger the more
disagreement there is on the bench. We find that the workload of a judge
does decrease the likelihood of dissent. Moreover, although
inconclusively, the dissenting behavior of a CCC judge seems to vary
depending on the stage of their term. Lastly, we reveal similar trends
in behavior of judicial coalitions from plenary proceedings also in the
3-member panel proceedings. Qualitative research based on interviews
with justices are not uncommon (Domnarski, 2009; Epstein \& Knight,
1997; Smekal et al., 2021) however they are not a typical tool for their
decision-making analysis since they can be bound by the law on
confidentiality, not wiling to participate (Nir, 2018) or a guinea pig
effect can occur (Smekal et al., 2021). Despite these problems and
possible distortions of qualitative research we argue that in some legal
culture, empirical limitation can be further developed on qualitative
research. Thus deper understanding of the institutional context is
needed.

\hypertarget{czech-constitutitonal-court}{%
\subsection{Czech Constitutitonal
Court}\label{czech-constitutitonal-court}}

The Czech Constitutional Court was established i 1993 and consist of 15
judges appointent for 10 year office term. The national literature
distinguishes between the ``first'' ``second'' and ``third'' decade of
the Constitutional Court because of the ten-year mandate. Of the 15
Justices three of are not members of chambers since they perform a
certain function (Chairman, Vice-Chairmen). The rest is divided into
four three-member chambers. CCC decides only in two panel formats -
chambers or plenary. Chambers decide on constitutional complaints while
plenary deals with abstract control of norms (repeals a law or part of a
law). During the third decade, in 2016, CCC decided to implement
rotation of the Chambers, since the panel-composition in Chambers was
unchanged for the duration of the judge's term of office. Which meant
that the same judges spent their entire terms together. After 2016 the
chairman of the Chamber shifts to the next Chamber (Chairmain of first
chamber shifts two second, second to third, third to fourth and fourth
to first). The composition of Chambers rotate every two year. Justice
can dissent both in chambers or plenary decision. The Venice Commision
report on separate opinion of Constiutional Courts states that: „In the
Czech Republic, the experience of the communist regime led to the
introduction of separate opinions, which were seen as a means of
protecting the personal integrity of individual judges. \emph{They
continue to fulfil this role to this day. It is therefore important for
a judge of the Czech Constitutional Court that a clear indication in the
heading of each decision is included stating the name of the judge
rapporteur who prepared the majority finding\ldots.The Czech doctrine
claims that judges who draft separate opinions take off their mask of
anonymity, because they have openly admitted that they do not agree with
the majority and that the Court's decision was not reached unanimously.
It also shows that the winning legal opinion was not accepted
unequivocally, but that it was reached after difficult deliberations and
after consideration of various arguments. Linking separate opinions to
the name of a particular judge increases his or her responsibility for
voting and content of the separate opinion.``} (Venice Commission, b.r.)
However it is important to note, that dissent in chambers reveal the
result of vote since the outcome is 2-1. Also most of the cases (95\%)
will result in inadmissibility by decision of the case. This is a
quasi-meritorious review that says the case does not rise to the level
of constitutional law. In the case of a rejection, it is already a
merits review, which is decided by a judgment. For inadmissibility
decision one need an unanonimous vote. If one of the judges decide to
dissent in chamber, not only he transformers decision to judgment, he
also makes it legally more binding and the Judge-Rapporteur is forced to
change the structure of the decision. In plenary judgements, judges can
choose if they want to write dissent jointly and form coalitions or be
write it soliterly. Dissents have appeared at 18\% of all cases at CCC.

Deskritiptivní data o disentování soud a soudce

\hypertarget{method-and-data}{%
\section{Method and Data}\label{method-and-data}}

The study asks three research questions: 1. What are the attitudes of
judges of the CCC towards dissent? 2. What motivates a judge of the CCC
to write a dissent? 3. How a joint dissenting opinion arises at the CCC
? To answer these questions, we conducted semi-structured interview, ``a
qualitative data collection strategy in which the researcher asks
informants a series of predetermined but open-ended questions'' (Given,
2008) with judges of the third decade of the CCC. We contacted all
fifteen constitutional court judges of the third decade to explore their
perspectives and experiences. Of the fifteen judges, 9 agreed to
participate in the research, 2 refused to participate and the rest did
not respond to our repeated requests. Interviewee provided interviews
with the condition of subsequent anonymization and approval. To unify
anonymization, we present quotes from all interviewees in generic
masculin. A written interview guide with a list of topics to be covered
was developed in advance in accordance with the literature reivew. Once
informed consent was obtained from the participants, interviews were
conducted by one of the authors in a office of individual judges
directly in the building of the CCC located in Brno from July 2023 to
August 2023. The interviews took 50 minutes on average. The interviews
were recorded, transcribed verbatim, and analysed using the ATLAS.ti
software kit. Thematic analysis, ``a method for identifying, analysing
and reporting patterns (themes) within data'' (Braun \& Clarke, 2006),
was used to organise and describe the dataset. The authors coded the
interviews and consolidated codes into several content domains using the
inductive approach. Thematic analysis revealed two main themes. Firstly
we will focus on the proces of making a dissent: what are the unwritten
rules, how rotation of chambers affected the CCC, which cases made an
impact and aspects of interaction and communication at CCC. Secondly we
present the three types of dissenting judges and motivation of judges
towards disents. There are four subthemes: previous experience,
emotional vent, caseload and importance of the case.

\hypertarget{the-birth-of-dissent-at-ccc}{%
\section{The birth of dissent at
CCC}\label{the-birth-of-dissent-at-ccc}}

Since CCC do not have a act on rule of procedure, it is important to map
the internal process that takes place before courts decision. Since
panel-composition affect the process we start with the decision-making
process in Chambers. First of all, when case arises at CCC
judge-rapporteur is assigned. Judge-rapporteur mostly prepare the draft
of the court decision after some familiarisation with the file and send
it to his colleagues for further discussion. Judge-rapporteur than
disscuses the case with other judges. The style of communication in
chambers depends on prior agreement. Some of the judges like to meet in
person other communicate better by e-mails. It could be said, that some
judges prefer more personal approach rather than e-mail communication.
However, when the case is complicated or something is problematic one of
the judges always suggest a meeting in person.

\emph{„E-mail corespodence at the beginning, followed by a phase when
the colleagues you have approached are asked to express themselves. So
they'll respond electronically again, actually send the text back after
editing, and some will ask for in-person meting. If that doesn't settle
the matter, then the three-judge panel meet all together.``}

During disscusion judges presents their opinion. It is this phase that
creates the embryo of dissent, if one of the judges decide to break the
majority opinion. Surely, judges are coming prepared for the discussion,
but it is unlikely, that the decision of dissent was already made.
Furthemore, the cost of dissent in chamber is higher sice it forces the
judge-rapporteur to do more work.

\emph{„If the Judge-Rapporteur proposes inadmissability by decision, and
I disagree, I will force the Judge-Rapporteur to invite the parties to
make a statement, others to reply\ldots{} I feel obliged to write the
dissent for the sake of it and not to let everything wear down the
judge-rapporteur, who suddenly has a lot more work to do, and so in that
case I feel obliged to also spit out some idea and explain to some
extent why the ruling is being made by way of a ruling.``}

It is not only the colegiality cost that judges consider in chamber
dissent. Strategical moves are also considerate, since the dissent can
transform from non-binding decision to binding judgment.

\emph{„It's just that in some cases, I've made, and I'll say this as a
frankly strategic consideration, that knowing that I would be in the
minority, you actually when you do a dissent in the Senate, you make a
denial of a resolution that is not intended to preclude a binding denied
finding. So I was making and even in this case I was making and even on
the floor a strategic consideration that it was better to be denied than
to be rejected.``}

Interestingly, strategist judges also consider with whom to they rule on
the decision. Since rotation of judges is available, one of the judges
revealed, that he strategically wanted to resolve the case with the
current composition, as the judge considered those more prominent in the
legal community and cared about the academic impact of the case.

\emph{„\ldots so actually by passing it in the more difficult Chamber
and gaining support from my colleagues, had a lot more public
legitimacy.``}

After the signing of voting protocol, judge-rapporteur waits for the
dissent and then announce the case. The situation in plenary cases is
quite different, since plenary meeting are mandatory in person and
planned on every tuesday. If judge-rapporteur want to discuss his case
on plenary meeting, he has to send by e-mail his draft with the
information on the case and his solution a week before. This is a
non-writen rules which is respected by all judges. After
judge-rapporteur send his draft some of the judges decide to react to
the draft by sending in reply their comments and any suggestions for
that particular case. These judges are called ``opinion leaders''.

\emph{„Usually it's that the judge who has some reservations about it
before the plenary session and writes out in advance what his
reservations are and sends it out to everybody else becomes the leader
of the dissenting group and then most people tend to sort of join in and
possibly write something of their own here and that's sort of the way
that the group is formed. Whoever starts dissenting.``}

It is not by any mean that the ``opinion leaders'' are always the same
judges. Opinion leader has a intrinsic reason why he is willing to
invest his free time in preparing and commenting on the case. Some of
the judges argue, that opinion leaders are based on the area of the law.
Some of them presentes themselves as an opinion leaders, since they
consider it their duty to always speak up when they have some arguments
against. Also it could happen that the opinion leader will arise during
the debate at plenary meeting. It is not a rule, that opinion leader are
only those, who comment on the case in advance. After all the e-mails
judges meet at the plenary meeting, when judge-rapporteur debates with
other judges. In this phase voting coalitions are emerging. Most of the
judges agree that they know from the debate, even before the voting, how
other judges will vote since most of the judges present their opinion or
at least announce with whom they agree.

\emph{„And then my experience is that after voting, when I know that
four colleagues vote against the judgment like me, I always offer them:
do you want to sign my dissent or not?{}``}

However, there are some exception, when judges do not speak up and then
quietly announce that they will express themselves in dissent. But that
really does happen in exceptional situations. After the debate, if the
chairman finds that the case has been sufficiently discussed, he will
open the vote. The non-written rule is that juges at this stage will
definitely confirm whether they will dissent or not. In the case of
dissent, it is necessary to give dissenters time to write their dissent.
In practice, judges are usually given a week to write the dissent. After
the vote, they decide when the case will be announced. The dissenters
have time until then to file a dissent. A curious case occurred in the
case of the repeal of the law on elections to the Chamber of Deputiesm
Pl. US 44/17. This case was mentioned by all of our nine intervenes in
this context. The problem that arised there was that the case held the
judge-rapporteur for three-years due to numerous reasons. After the vote
however chairman of CCC decided that the case will be announced the next
day, since it repealed part of the law three months before the elections
to the Chamber of Deputies. Dissenting judges criticzied in the dissent
the lack of time given which then lead to new agreement between judges
that a week should be guaranteed at minimum for writing a dissent.\\
The vote in plenary decision can also bring a situation, when
judge-rapporteur will lose his case. In that scenario, the chairman put
the case forward to the ``most louder opinion leader'' that becomes
judge-rapporteur. In this cases judge-rapportur uses his draft as a
basis for dissent. After the voting, opinion leader asks other their
colleagues if they want to join. Most of the judges perceive joint
dissent as an advantage. However even here we can spot some exceptions.

Researcher: \emph{\ldots. and has it ever happened to you that maybe a
judge wanted to have a solitary dissent and didn't want to invite you?}
Interviewee: \emph{Yeah as far as I recall there was a situation like
that with one the justices.}

After deciding who will write with whom the dissent, the judges get down
to writing. The Justices perceive greater literary freedom in the
possibility of a solitary dissent. Joint dissents are most often written
by the opinion leader. The other justices in the dissenting coalition
modify or comment on its text. In a few cases, there has also been joint
writing of dissents (each writing a specific portion).

\hypertarget{the-fan-hater-and-strategist}{%
\section{The fan, hater and
strategist}\label{the-fan-hater-and-strategist}}

There are three types of judges attitude toward dissent: there is the
fan, the hater and the strategist. Respondents revealed that those
attitudes toward dissenting represents a scale with to sides: One the
one end, there are ``haters'' who do not like disensts and dissent only
exceptionally; on the other side there are ``fans'' who like to dissents
and dissent everytime (as much as possible). In the middle of the scale,
there is the third group of judges who dissent strategically. They are
neither completely open to dissent nor completely closed to it, simply
put, they dissent where it suits them.

Example of hater: \emph{„I don't like them (dissents). {[}\ldots{]} I
believe that when a decision is taken by a majority, it is not to
comment further on some B that someone thought otherwise.``} Example of
fan: \emph{„I'm a big fan of dissent {[}\ldots{]} And I can safely say
that so far, I've been on the CCC for many years now, I've dissented
every time I've voted no. It's a sign of fairness to explain even
outwardly why I didn't support the majority opinion here.``}

After identification of the three types of judges attitude towards
dissent we need answer the question why they dissent? Even the judges
from the group of those who don't like dissent, dissent in some cases.
There is not a single judges without a dissent at CCC. One of the main
factor is the previous profession of constitutional judge. When the
constitutional judge is coming from the ordinary court system, he is not
used to dissent unles he was judge at the Supreme Court or Supreme
Administrative Court where is the possibility to dissent. For a carrier
judge one of the main aspect of judicial decision-making is to create
legal certainty for the public. Dissent however these certainties, from
their point of view, undermine.

\emph{„Basically, from my point of view, it undermines the authority of
the court to some extent. It was simply decided by majority vote,
period.``}

On the other hand, even those judges dissent sometimes. They explained
to us, that there are some cases where is it strongly principal opinion
for them and also a possibility to join in someone elses dissents, since
they wouldn't want to waste time on writing a dissent:

\emph{„Whether it's a values\ldots{} so yeah I made an agreement with a
colleague who felt the need to write the dissent. I would read it to
consider whether or not to join.``}

Furthemore judges who like to dissent and dissent as much as possible
have different reasons why they do so. They manage to dissent every
time, they vote against the resolution, however that do not mean that
they are always the authors of the dissents, since it would be nearly
impossible due to time management and the amount of workload at the CCC.
These reasons are in the end perceived as an obligation:

Interviewee: \emph{„it may not be a legal obligation, but I feel it's my
professional obligation. to write a dissent from a legal standpoint is a
judge's possibility not a duty, but I feel it's an ethical duty for a
judge to always write a dissent if they vote no.``} Interviewee:
\emph{„it should simply be the principle that the judge should reveal
why he was against.``}

There are four main aspects that further explains the decision of CCC
justice to dissent: 1) previous experience, 2) emotional valve, 3)
caseload and 4) importance of the case.

\hypertarget{previous-experience}{%
\subsection{Previous experience}\label{previous-experience}}

CCC presents a Legal Olymp and happens to be the highest rank that a
lawyer can obtain in the Czech Republic. President of the Czech Republic
nominates candidates for CCC Justices and Senate approves them. Only
experienced and prominent candidates are considered. As a rule, the
president tries to nominate as diverse a constitutional court as
possible - he seeks diversity not only in terms of gender but also in
terms of profession. The judges include academics, lawyers, but also
judges of higher and lower courts. Of course, for many it is not
possible to identify one profession (e.g.~a judge who also works in the
academy). This also influence the familiarity of CCC justices - some
have already met at distric court or department at university, some of
them dont know eeach other.

\emph{„Often those social relationships come from some previous work - I
don't know, I know some of my colleagues from the university
department.``}

Our interview revealed, that Justices perceive the previous profession
of their colleagues and at the same time transfer their knowledge from
the former profession to the current one.

Interviewee: \emph{„Then some academic comes along at CC and he's used
to writing academic papers and that's a bit different than judgments or
disent. In an academic paper you're completely free, you can just write
whatever you want and you're the only one behind it and it actually like
influences some of the other academic discussion, but it doesn't
directly affect people's fates.``} Interviewee: \emph{„Many colleagues
see the judgment as some kind of scientific work. For me, as a common
judge, it's a judgment\ldots.the basis of judicial work is to respect
the majority ruling. If the majority outvotes me, I click my heels and
secondly, it is equally basic to judicial work to respect the binding
legal opinion of a higher authority.``} Interviewee: \emph{„Obviously,
for example, if somebody is not a judge and now he's actually judging
here for the first time, it's a bit of a problem for him to learn the
procedures at court.``}

Three hypothesis arises from this. Firstly, judges from lower courts are
less like to dissent since they are usde to the binding nature of the
decision of the Court of Appeal and at the same time they are more aware
of the aspect of undermining the authority of the court by dissent.
Secondly, judges who come from the Supreme Court and Supreme
Administrative Court do not perceive this effect as they are at the top
of the hierarchy and their courts allow dissent to be exercised in
certain cases. Also judges at higher level have the power to determine
what direction the case law should take, while lower court judges must
learn to obey the higher courts. That situation was described by one of
our respondents as follows:

\emph{„If you're asking about the judicial career, it's related to the
fact that as you grow to the higher levels, then of course you're more
interested in influencing that jurisprudence by your rulings.``}

Thirdly, Justice who are academic and do not have any experience as
judges will have the ambition to take on an academic dimension in the
decision/judgment. At the same time, one can also expect more freedom in
writing dissent.

\hypertarget{emotional-vent}{%
\subsection{Emotional vent}\label{emotional-vent}}

Every Interviewee touched upon the topic of emotions and frustration
after the plenary or chamber discussion. A portion of the judges
admitted that frustration sometimes leads them to write dissents.
Dealing with emotions has proven to be a big issue for Constitutional
Court judges. The strong attitude that emerged viewed the dissent as a
positive tool for the judges' common ground. In a fundamental
disagreement, a judge do not need need to bottle up his emotions but he
can just vent them out through dissent.

Researcher: \emph{So if I may start - what is the meaning of a written
descent for you?} Interviewee: \emph{„It's kind of a relief, if we don't
agree that we do, in the reasoning or something else - it's a safety
valve to agree at all.``}

Also judges agreed that emotional dissent is easily written soliterly
than jointly. Since soliteraly dissent does not have to be approved by
anyone else: „Honestly when you write it yourself you are making less
compromises.`` The rest of Justices said, that they do not feel the need
to vent emotions. Interestingly, those who do not experience such
frustrations reminisced about their previous judicial careers:

\emph{„I'd have to hang myself if I was that emotional in the justice
system. Of course, when I was young and stupid, the court of appel
dissmised some of my cases. So I read it and wrote it like they wanted
from me, period.``}

Our hypothesis is, that those Justices are experienced vented their
emotions in a different way, as their previous judicial profession led
them to do so. Emotions also lead to strong personal expressions in
dissent. Most held the view that dissent should not interfere on a
personal level and argued that it is strictly unethical and
unprofesional. But the minority defended opposit approach with the
following argument:

\emph{„Someone said that simply the dissent is lost. yeah I just failed
to get a majority for my arguments, but when there is a loss it should
be as fair. So I recognize that if it's not fair, there's an opportunity
to argue back on a more personal level.``}

Those judges simply feel that something procedural was done unfair
against them, and the only way to draw attention to this problem is to
write a dissent that is intended to inform the public of some injustice
that has taken place in the Constitutional Court.

\hypertarget{caseload}{%
\subsection{Caseload}\label{caseload}}

One of the big factors that influences the decisions judges is caseload.
CCC is considered as one of the busiest court in the Czech Repubic. But
how does caseload affect individual judges at CCC? Dissenting haters are
more likely to spend time on their cases ratther than writing dissent.
However dissenting fans are more likely to overlook the caseload for
dissent. Strategist decide wheter he can aford to disent regarding the
amount of caseload he is dealing with at the moment.

Hater: \emph{„Well, I don't have time to write dissents and stuff like
that (laughs).``} Fan: \emph{„I dissent every time.``} Strategist:
\emph{„If I'm against it but it's not worth a dissent, either because I
don't consider it so fundamental, or that or just for completely prosaic
reasons that I just don't have the time.``}

Furthemore, we releaved collegiality effect regarding caseload. Judges
feels the obligation to dissent, if they by their dissent, assign extra
work to the Judge-Rapporteur. Extra work can be a situation where the
judge-rapporteur is forced to rewrite decision to judgment. Other
situation occurs when dissent is caused by delays in the proceedings.
The dissenting judge tries as much as possible to convince the
judge-reporter to the contrary, which ultimately delays the entire
process of the decision-making and subsequent announcement of the
judgment. Such time delays, which can lead to an increase in the
caseload of both judges, lead the dissenting judge to exercise dissent.
Collegiality effect was also revealed in joint dissent. Some judges are
agreed in advance to take turns writing dissents when voting together,
or simply ask each other to write the dissent for time reasons.
Long-term joint dissenting collaborations arise from frequent forms of
voting and consensus.

\emph{„We decided who would write the dissent and then we communicated
among ourselves and sent it to the other judges. We took the comments
into account, so like then it was a collective work, but like every
text, somebody just has to write the basis.``}

\hypertarget{importance-of-the-case}{%
\subsection{Importance of the case}\label{importance-of-the-case}}

All of the dissenting judges type have one thing in common, they
dissent. That means that for every judge there is a plausibility for
dissenting. Turning point for every judge stand on importance of the
case. This factor is however very subjective, since the boundaries of
importance are entirely dependent on the personality of the judge and
his perception:

\emph{„I think dissent is a fairly strong expression of disagreement,
and often one disagrees over less substantive things. I think that
dissent is meaningful if one is expressing a fundamental position or a
fundamental opinion.``} \emph{„It is really just supposed to be a
question of legal opinion, strong legal opinion, not some kind of
impressionology that I think something and it is a big question whether
to bring into it the way of, for example, making that decision.``}
\emph{„Really depends on the particular case. I distinguish the
dissenters by how I see them, of course. There may be someone else may
say it's just different it's not that important. I judge it by what I
think of it and if I then pay more attention to it, but that doesn't
mean I don't pay less attention to those at it as important. Of course
it's how they handle it. Because just like the decision you have it the
same that yeah I address that decision as more important so I give it
more attention.``}

\hypertarget{conclusion}{%
\section{Conclusion}\label{conclusion}}

\end{document}
