% Options for packages loaded elsewhere
\PassOptionsToPackage{unicode}{hyperref}
\PassOptionsToPackage{hyphens}{url}
%
\documentclass[
  11pt,
]{article}
\usepackage{amsmath,amssymb}
\usepackage{iftex}
\ifPDFTeX
  \usepackage[T1]{fontenc}
  \usepackage[utf8]{inputenc}
  \usepackage{textcomp} % provide euro and other symbols
\else % if luatex or xetex
  \usepackage{unicode-math} % this also loads fontspec
  \defaultfontfeatures{Scale=MatchLowercase}
  \defaultfontfeatures[\rmfamily]{Ligatures=TeX,Scale=1}
\fi
\usepackage{lmodern}
\ifPDFTeX\else
  % xetex/luatex font selection
\fi
% Use upquote if available, for straight quotes in verbatim environments
\IfFileExists{upquote.sty}{\usepackage{upquote}}{}
\IfFileExists{microtype.sty}{% use microtype if available
  \usepackage[]{microtype}
  \UseMicrotypeSet[protrusion]{basicmath} % disable protrusion for tt fonts
}{}
\usepackage{xcolor}
\usepackage[margin=1in]{geometry}
\usepackage{longtable,booktabs,array}
\usepackage{calc} % for calculating minipage widths
% Correct order of tables after \paragraph or \subparagraph
\usepackage{etoolbox}
\makeatletter
\patchcmd\longtable{\par}{\if@noskipsec\mbox{}\fi\par}{}{}
\makeatother
% Allow footnotes in longtable head/foot
\IfFileExists{footnotehyper.sty}{\usepackage{footnotehyper}}{\usepackage{footnote}}
\makesavenoteenv{longtable}
\usepackage{graphicx}
\makeatletter
\def\maxwidth{\ifdim\Gin@nat@width>\linewidth\linewidth\else\Gin@nat@width\fi}
\def\maxheight{\ifdim\Gin@nat@height>\textheight\textheight\else\Gin@nat@height\fi}
\makeatother
% Scale images if necessary, so that they will not overflow the page
% margins by default, and it is still possible to overwrite the defaults
% using explicit options in \includegraphics[width, height, ...]{}
\setkeys{Gin}{width=\maxwidth,height=\maxheight,keepaspectratio}
% Set default figure placement to htbp
\makeatletter
\def\fps@figure{htbp}
\makeatother
\setlength{\emergencystretch}{3em} % prevent overfull lines
\providecommand{\tightlist}{%
  \setlength{\itemsep}{0pt}\setlength{\parskip}{0pt}}
\setcounter{secnumdepth}{5}
\usepackage{longtable}
\renewcommand{\thesection}{\Alph{section}}
\LTcapwidth=.95\textwidth
\linespread{1.05}
\usepackage{hyperref}
\usepackage{booktabs}
\usepackage{longtable}
\usepackage{array}
\usepackage{multirow}
\usepackage{wrapfig}
\usepackage{float}
\usepackage{colortbl}
\usepackage{pdflscape}
\usepackage{tabu}
\usepackage{threeparttable}
\usepackage{threeparttablex}
\usepackage[normalem]{ulem}
\usepackage{makecell}
\usepackage{xcolor}
\ifLuaTeX
  \usepackage{selnolig}  % disable illegal ligatures
\fi
\usepackage{bookmark}
\IfFileExists{xurl.sty}{\usepackage{xurl}}{} % add URL line breaks if available
\urlstyle{same}
\hypersetup{
  pdftitle={„I have spoken and saved my soul: A Qualitative Analysis of Dissenting Behaviour of Czech Constitutional Judges Dissenting},
  pdfauthor={ANONYMIZED; ANONYMIZED},
  hidelinks,
  pdfcreator={LaTeX via pandoc}}

\title{„I have spoken and saved my soul\footnote{Respondent 9's statement on the importance of dissent.}: A Qualitative Analysis of Dissenting Behaviour of Czech Constitutional Judges Dissenting}
\author{ANONYMIZED \and ANONYMIZED}
\date{}

\begin{document}
\maketitle
\begin{abstract}
This paper presents a qualitative study on the dissenting behavior of judges at the Czech Constitutional Court (CCC). Building on previous quantitative research, the study investigates how judges' perceptions of their roles and the institutional framework influence their decisions to dissent. The research explores the motivations behind dissenting opinions, the impact of collegiality, and the role of unwritten rules and norms, particularly the `norm of consensus.' Through interviews with judges from the CCC (n=9), the study identifies different stances towards dissenting opinions, revealing the complex interplay between personal values, institutional pressures, and strategic considerations. The findings highlight the nuanced ways in which judges navigate the tension between expressing individual opinions and maintaining the court's legitimacy. The study also examines how the introduction of chamber rotation has altered the dynamics of dissent within the court. Overall, this research contributes to a deeper understanding of judicial behavior in the context of the CCC and offers insights that may be relevant to other constitutional courts operating within civil law traditions.
\end{abstract}

\section{Introduction}\label{introduction}

\emph{``I don't like them {[}separate opinions{]}. (\ldots) Because I am a routine judge and I am of the opinion that when a collegiate body makes a decision, a person X shouldn't further comment on it, just because they were of a differing view. It is undermining the authority of that court.''} The previous quote has been voiced by one of the Czech Constitutional Court (``CCC'') judges. It expresses one of many possible stances a judge may take towards exercising their right to dissent that's been shaped and influenced by their view of the role of a CCC judge and the CCC itself, a concept we later coin a norm-identification. The possible aspects influencing the judges' decision whether to dissent or not is manifold and has encompassed political considerations,\footnote{Chris Hanretty, \emph{Judicial {Disagreement} Need Not Be {Political}: {Dissent} on the {Estonian Supreme Court}}, 67 \textsc{Europe-Asia Studies} 970 (2015).} strategic considerations,\footnote{Lee Epstein, William M. Landes \& Richard A. Posner, \emph{Why ({And When}) {Judges Dissent}: {A Theoretical And Empirical Analysis}}, 3 \textsc{Journal of Legal Analysis} 101 (2011).} or institutional-systemic consideration.\footnote{Nuno Garoupa \& Peter Grajzl, \emph{Spurred by Legal Tradition or Contextual Politics? {Lessons} about Judicial Dissent from {Slovenia} and {Croatia}}, 63 \textsc{International Review of Law and Economics} 105912 (2020).}

This paper presents a qualitative empirical study on the dissenting behavior of judges of the CCC. More specifically, we pose the questions of how has their dissenting behavior been shaped and influenced by their perception of the role of the CCC shape their dissenting behavior and by the institutional setup of the CCC, and how do they interact with their peers during the process of writing a separate opinion. To answer the research questions, we conducted and evaluated interviews with the judges of the third decade of the CCC.

We build on our previous research effort, which employed quantitative methods to test hypotheses generated mainly within the rational-choice theory and within the framework of the identification-disagreement model\footnote{We leave the references to our paper ANONYMIZED}. In a quantitative large-n study, we found that there is not a strong norm of consensus operating at the CCC, that the disagreement potential of cases seems to be positively correlated with the dissenting behavior of CCC judges, that the CCC judges adapt to their workload, and finally that the judges pay attention to collegiality costs a separate opinion may involve.

In this study we choose depth over breadth.\footnote{Thomas Gschwend \& Frank Schimmelfennig, \emph{Introduction: {Designing Research} in {Political Science} --- {A Dialogue} Between {Theory} and {Data}}, \emph{in} \textsc{Research {Design} in {Political Science}: {How} to {Practice What They Preach}} 1 (Thomas Gschwend \& Frank Schimmelfennig eds., 2007).} We choose CCC for three main reasons: firstly, separate opinions are allowed at the CCC unlike at other Czech courts. Attaching a separate opinion is, however, not an obligation. The vote is not public, so if a judge votes against a decision but does not exercise the dissent, there is no way for the public to know that the judge voted against the majority. That opens up room to strategize and potentially decide not to dissent despite their disagreement with the majority outcome. This phenomenon has been termed as dissent aversion in the empirical legal scholarship.\footnote{Epstein, Landes, and Posner, \emph{supra} note 3; Lee Epstein, William M. Landes \& Richard A. Posner, \emph{The {Behavior} of {Federal Judges}: {A Theoretical} and {Empirical Study} of {Rational Choice}}, \emph{in} \textsc{The {Behavior} of {Federal Judges}} (2013); Richard A. Posner, \emph{What {Do Judges} and {Justices Maximize--}({The Same Thing Everybody Else Does})}, 3 \textsc{Sup. Ct. Econ. Rev.} 1 (1993).} Previous interviews with CCC judges suggest that they do indeed exercise dissent aversion.\footnote{\textsc{Jan Kysela, Kristýna Blažková \& Jan Chmel}, \textsc{Právnický {Olymp}: Portréty Vybraných Soudců {Ústavního} Soudu {ČR}} (2015); \textsc{Hubert Smekal et al.}, \textsc{Mimoprávní Vlivy Na Rozhodování Českého {Ústavního} Soudu:} (2021).} Secondly, the CCC judges can dissent alone or in group with other judges. The circumstances under which judges decide for either option have not been thoroughly researched. The relevance of such research exceeds the CCC: joint dissents are also allowed at other apex courts such as the SCOTUS. Lastly, it is hard to capture certain features solely with quantitative methods. This paper also concerns the role of, for example, unwritten rules and customs, and it results in a typology of stances of judges towards a dissent rather than inference.

The paper proceeds as follows. In \hyperref[theory-dissent]{section A}, we situate our research in the existing theories of dissenting behavior. In \hyperref[primer]{section B}, to give the reader sufficient context, we present back-ground information on the CCC and its Justices. In \hyperref[method]{section C}, we discuss the method we use to evaluate the interviews as well as how we conducted the interviews with the CCC judges. In \hyperref[empirics]{section D}, we evaluate the interviews in light of our research questions. \hyperref[conclusion]{Section E} concludes.

\section{Theories of Dissenting Behavior}\label{theory-dissent}

Lee Epstein \& Jack Knight\footnote{\emph{Toward a {Strategic Revolution} in {Judicial Politics}: {A Look Back}, {A Look Ahead}}, 53 \textsc{Political Research Quarterly} 625 (2000).} argue that judges as ``(1) social actors make choices in order to achieve certain goals, (2) social actors act strategically in the sense that their choices depend on their expectations about the choices of other actors, and (3) these choices are structured by the institutional setting in which they are made.'' Instead of holding judges to pursue political policy oriented goals, the judges' self-interest in terms of career progression, higher income, more leisure, or lesser workload takes the spotlight.\footnote{\emph{Id.}} In their empirical study on dissenting behavior on the Supreme Court of the USA (``SCOTUS''), Epstein, Landes, and Posner\footnote{\emph{Supra} note 3.} posit that ``a potential dissenter balances the costs and benefits of issuing a dissenting opinion.'' and that judges have ``leisure preferences, or, equivalently, effort aversion, which they trade off against their desire to have a good reputation and to express their legal and policy beliefs and preferences (and by doing so perhaps influence law and policy) by their vote, and by the judicial opinion explaining their vote, in the cases they hear.''

In the European context, Caroline Wittig\footnote{\textsc{The {Occurrence} of {Separate Opinions} at the {Federal Constitutional Court}} (2016).} summarizes in her study on separate opinions at the Federal Constitutional Court of Germany (``GFCC'') the potential utilities for judges to attach a separate opinion and, thus, to acquire additional costs: (1) potential of impacting future caselaw, (2) moral obligation to distance oneself from a decision that contradicts her values, (3) to convey certain image about oneself. These motivations also largely rely on the self-perceived stance towards separate opinions in general. The proponents of separate opinions view dissenting positively based on the separate opinions being able to enrich the legal debate, being a sign of judicial independence, increasing the legitimacy of any given decision for it makes the decision more accurate of the real discussion behind it. The opponents of separate opinions mainly argue that showing the inability to speak in one voice undermines a court's legitimacy or the reputation of the dissenting judge. Such a view is perfectly in line with the quote from the introduction. Moreover, judges seeking the appreciation from the general public or legal community may act in their personal interests instead of in the court's interests. Lastly, separate opinions come at collegiality costs and may harm the mutual relationships of judges. In our study, we pose the research question:

\textbf{RQ1:} \emph{What utilities do motivate the CCC judges to dissent or not to dissent?}

\subsection{Identification-disagreement model}\label{identification-disagreement}

Wittig argues that the traditional theories of dissent stemming from the common law context all have limited explanatory power within the civil law context, as judges therein are deciding in a different context, bound by different procedural rules, and, thus, given differing, sometimes broader, sometimes more limited, avenues to give way to their policy preferences or strategic considerations. To alleviate these issues, Wittig introduces a non-formal model of dissenting behavior, the identification-disagreement model. The model is made up of two dimensions: the disagreement level and judges' stance and degree of self-identification of their role as a judge, termed as \emph{norm of consensus}. Separate opinions are then ``a function of a judge's identification with the norm of consensus and the level of disagreement of judges.\footnote{\emph{Id.} at 74--75.} We base our theory on the identification-disagreement model because, institutionally, the CCC is akin to the GFCC and because the results of Wittig's empirical study corresponded to her theory.

Gregory A. Calderia \& Christopher J. W. Zorn\footnote{\emph{Of {Time} and {Consensual Norms} in the {Supreme Court}}, 42 \textsc{American Journal of Political Science} 874 (1998).}, p.~876-877 define a norm as ``a long-run equilibrium outcome, which underpins the interaction between individuals and reflects common understandings as to what is acceptable behavior in given circumstances.'' The norm of consensus in turn defines the level of dissent that is acceptable at any given court.\footnote{Paresh Kumar Narayan \& Russell Smyth, \emph{The {Consensual Norm} on the {High Court} of {Australia}: 1904-2001}, 26 \textsc{International Political Science Review} 147 (2005); \textsc{Wittig}, \emph{supra} note 12 at 75.} Wittig's argument is two-fold. First, in civil law traditions unlike its US counterpart, the prevailing notion of the norm of consensus is that a court should not display disagreement. Second, the extent of adherence to the norm varies among judges.\footnote{\textsc{Wittig}, \emph{supra} note 12 at 75.}

A dissonance between a proposed outcome for a case and any given judge's preferences are eventually bound to happen. In such a case, the judge can either express their sincere preferences by writing a separate opinion or they can adapt their behavior according to the norm of consensus and suppress the expression of her preferences. The second route has also been termed \emph{dissent aversion} and theoretically fleshed out by Epstein, Landes, and Posner\footnote{\emph{Supra} note 3.}. The decision of judge faced with such a conflict whether to attach a separate opinion or whether to avert their dissent is then a function of multiple potential utilities.

Wittig draws up three types of utility that dictate various levels of the adherence to the norm of consensus. Firstly, the intrinsic utility is maximized whenever a judge behaves in accordance with their true values and opinions, setting aside their strategic or political considerations. Secondly, expressive utility is harnessed when one displays individuality and counters the notion of conformism. Thirdly, the reputational utility arises when one adjusts their publicly displayed preferences to the expectations of others. Wittig argues that maximizing the former two forms of utilities in a situation of disagreement leads to separate opinions, whereas maximizing the reputational utility in such a situation gives way to the norm of consensus, as the judge would otherwise jeopardize the court's legitimacy as well as their reputation for not adhering to commonly accepted norms.\footnote{\textsc{Wittig}, \emph{supra} note 12 at 76.} We pose the second research question

\textbf{RQ2:} \emph{How do the CCC justices understand and identify with the norm of consensus at the CCC?}

Epstein, Landes, and Posner\footnote{\emph{Supra} note 3.} address the issue of collegiality costs arising for a dissenting judge: ``The effort involved in these revisions, and the resentment at criticism by the dissenting judge, may impose a collegiality cost on the dissenting judge by making it more difficult for him to persuade judges to join his majority opinions in future cases.'' Based on this theory, they predict and indeed empirically confirm that ``dissents will be less frequent in circuits that have fewer judges because any two of its judges will sit together more frequently and thus have a greater incentive to invest in collegiality.'' In the CCC context, collegiality plays one more important role. The judges may exercise a solitary or a joint dissent. The latter carries with itself the utility of a division of labor at the cost of having to reach a compromise with other dissenters. Because the circumstances under which a joint dissent occurs have not been studied, our fourth research question is

\textbf{RQ3:} \emph{What motivates judges to write a joint dissent rather than a solitary one?}

Moreover, since 2016, the judges do not stay in permanent 3-member chambers. Every two years (for more details see the ensuing section), the president of the chamber rotates to another chamber. That has an impact on the dynamics among the judges that do no longer have to expect to sit with the same judges for their whole terms and, therefore, do not have to pay the same attention to the collegiality costs. Therefore, our last research question is

\textbf{RQ4:} \emph{Has the introduction of chamber rotation changed the perception of dissent by judges?}

\subsection{Primer on the CCC}\label{primer}

The CCC consists of fifteen judges, out of which one is the president of the CCC, two are vice presidents and twelve associate judges.\footnote{Following the terminology of David Kosar, \emph{The {Constitutional Court} of {Czechia}}, (Pre-published 2020).} These fifteen judges are appointed by the president of the Czech republic upon approval of the Senate, the upper chamber of the Czech two-chamber Parliament. The judges enjoy 10 years terms with the possibility of re-election; there is no limit on the times a judge can be re-elected. The three CCC functionaries are unilaterally appointed by the Czech president.

The appointment procedure is similar to how the SCOTUS judges are appointed as the procedure lays in the hands of the president of the republic and the upper chamber. The minimal requirement for a CCC nominee are 40 years of age, a clean criminal record, a finished legal education and experience in the legal field. Other than that, the nomination is left to the consideration of the President of the Republic. After a nomination, the nominee is firstly interviewed in the constitutional law committee of the Senate, which produces an unbinding recommendation for the plenary Senate hearing. The final binding decision is then made by simple majority of the Senate plenary hearing. This procedure has lead to a situation, in which there is very little variance as to the nominating background of the judges. First, there is no nominating political party akin to the US context or the Spanish context.\footnote{Chris Hanretty, \emph{Dissent in {Iberia}: {The} Ideal Points of Justices on the {Spanish} and {Portuguese Constitutional Tribunals}}, 51 \textsc{European Journal of Political Research} 671 (2012).} Second, because the court was established in 1993 and filled within roughly a year of its establishment and because the term of the Czech president is 5 years and all the 3 Presidents, who'd finished their term at the time of writing this article, have been elected twice (for ten years it total), each president has had the chance to appoint all the fifteen members of ``their'' CCC. Therefore, the first term of the CCC has been termed the Václav Havel, the second the Václav Klaus and the third Miloš Zeman terms of the CCC.

Regarding the competences, the CCC is a typical Kelsenian court inspired mainly by the GFCC. The CCC enjoys the power of abstract constitutional review, including constitutional amendments. The abstract review procedure is initiated by political actors (for example MPs) and usually concerns political issues. Moreover, an ordinary court can initiate a concrete review procedure, if that court reaches the conclusion that a legal norm upon which its decision depends is not compatible with the constitution. Individuals can also lodge constitutional complaints before the CCC. Lastly, the CCC can also resolve separation-of-powers disputes, it can \emph{ex ante} review international treaties, decide on impeachment of the president of the republic, and it has additional ancillary powers.\footnote{For a complete overview, see Kosar, \emph{supra} note 20.}

From the perspective of the inner organization, the CCC can decide in four bodies: (1) individual judges in the role of judge rapporteur, (2) 3-member chambers (\emph{senáty}), (3) the plenum (\emph{plénum}), and (4) special disciplinary chamber. The 3-member chambers and the plenum play a crucial role. The plenum is composed of all judges, whereas the four 3-member chambers are composed of the associate judges. Neither the president of the CCC or her vice-presidents are permanents members of the 3-member chambers. Until 2016, the composition of the chambers was static. However, in 2016, a system of regular 2-yearly rotations was introduced, wherein the president of the chamber rotates to a different every 2 years.

In the chamber proceedings, decisions on admissibility must be unanimous, whereas decisions on merits need not be, therefore, a simple majority of two votes is necessary to pass a decision on merits. In the plenum, the general voting quorum is a simple majority and the plenum is quorate when there are ten judges present. The abstract review is one of the exceptions that sets the quorum higher, more specifically to 9 votes.

A judge rapporteur plays a crucial role.\footnote{Jan Chmel, \emph{Zpravodajové a senáty: Vliv složení senátu na rozhodování Ústavního soudu... České republiky o ústavních stížnostech}, 25 \textsc{Časopis pro právní vědu a praxi} 739 (2017); Jan Hořeňovský \& Jan Chmel, \emph{The Process of making the Constitutional Court Judgements}, 23 \textsc{Časopis pro právní vědu a praxi} 302 (2015) study the large influence of the judge rapporteurs at the CCC.} Each case of the CCC gets assigned to a judge rapporteur. The assignment is regulated by a case allocation plan. They are tasked with drafting the opinion, about which the body then votes. The president of the CCC (in plenary cases) or the president of the chamber (in chamber cases) may re-assign a case to a different judge rapporteur if the draft opinion by the original judge rapporteur did not receive a majority of votes. Unfortunately, the CCC does not keep track of these reassignments.

The act on the CCC allows for separate opinions. They can take two forms: dissenting or concurring opinions. Each judge has the right to author a separate opinion, which then gets published with the CCC decision. It follows that not every anti-majority vote implies a separate opinion, it is up to the judges to decide whether they want to attach a separate opinion with their vote. Vice-versa, not every separate opinion implies an anti-majority vote, as the judges can attach a concurring opinion. In contrast to dissenting opinion, when a judge attaches a concurring opinion, they voted with the majority but disagree with its argumentation.\footnote{Which makes it difficult to, for example, conduct the same point-estimation with data on dissenting behavior of judges as Hanretty, \emph{supra} note 21 has done on the Portuguese and Spanish Constitutional Courts.} Unfortunately, it is difficult to conduct research on the dissent aversion because the voting is kept secret. However, it opens up the possibility to examine why judges dissent alone or together.

Judges can dissent both in chambers or plenary decisions. The Venice Commission report on separate opinions of Constitutional Courts states that: \emph{„In the Czech Republic, the experience of the communist regime led to the introduction of separate opinions, which were seen as a means of protecting the personal integrity of individual judges. They continue to fulfill this role to this day. It is therefore important for a judge of the Czech Constitutional Court that a clear indication in the heading of each decision is included stating the name of the judge rapporteur who prepared the majority finding\ldots.The Czech doctrine claims that judges who draft separate opinions take off their mask of anonymity, because they have openly admitted that they do not agree with the majority and that the Court's decision was not reached unanimously. It also shows that the winning legal opinion was not accepted unequivocally, but that it was reached after difficult deliberations and after consideration of various arguments. Linking separate opinions to the name of a particular judge increases his or her responsibility for voting and content of the separate opinion.``}\footnote{Venice Commission, \emph{{REPORT ON SEPARATE OPINIONS OF CONSTITUTIONAL COURTS}}, (2018).}

It is important to note that a separate opinion in the 3-member chamber decisions almost at all time reveals that the vote was split. Out of 90154 3-member chamber decisions, 94.09 \% result in inadmissibility, in which an unanimous vote is required. An inadmissibility decision entails only a quasi-meritorious review that results into the conclusion that a case does not contain a question of constitutional law. A rejection already entails review on merits. It follows that if one of the judges decides to dissent in the 3-member chamber, the decisions has to be made on merits, which further makes it legally more binding and the judge rapporteur is typically forced to change the structure of the decision and make the argumentation more comprehensive. This further disincentivizes separate opinions in the 3-member chamber proceedings as the decision to dissent turns a non-binding decision containing a majority opinion, with which the dissenting justice disagrees, into a binding decision on merits and it imputes large effort costs on the judge rapporteur.

The room for the dissenting judge and the majority to address each other differs between the two bodies. There is less back and forth interplay between the judges, more akin to the SCOTUS context. Most of the communication is handled remotely in the chamber proceedings, whereas the plenum meets regularly to discuss the cases in person. Despite that, from the perspective of the procedure, the process of generating separate opinions is the same. In both cases, the rapporteurs are informed about the outcome of the vote, which is filed in the voting record. The separate opinion is then sent to the judge rapporteur before the decision is announced, as it cannot be added until after the announcement. It is important to note that judges have the possibility, not the obligation, to dissent. In other words, there is room for judges to give way to strategic considerations.

\section{Method and Data}\label{method}

To answer our research questions, we conducted semi-structured interviews, ``a qualitative data collection strategy in which the researcher asks informants a series of predetermined but open-ended questions''\footnote{\textsc{Lisa M. Given}, \textsc{The {SAGE Encyclopedia} of {Qualitative Research Methods}} (2008).} with justices of the third decade of the CCC.
We contacted all 15 CCC justices, out of which 9 agreed to participate in the research, 2 refused to participate and the rest did not respond to our repeated requests. The interviewee provided interviews with the condition of subsequent anonymization and approval. To unify anonymity, we present quotes from all interviewees in plural.
A written interview guide with a list of topics to be covered was developed in advance in accordance with the literature review. Once informed consent was obtained from the participants, interviews were conducted by one of the authors in an office of individual judges directly in the building of the CCC located in Brno from July 2023 to August 2023. The interviews took 50 minutes on average.

The interviews were recorded, transcribed verbatim, and analyzed using the ATLAS.ti software kit. Thematic analysis, ``a method for identifying, analyzing and reporting patterns (themes) within data,''\footnote{Virginia Braun \& Victoria Clarke, \emph{Using Thematic Analysis in Psychology}, 3 \textsc{Qualitative Research in Psychology} 77 (2006).} was used to organize and describe the dataset. The authors coded the interviews and consolidated codes into several content domains using the inductive approach.

\section{Interviews evalutation}\label{empirics}

We now separately discuss two main themes that appeared in the interviews. Firstly, in \hyperref[birth]{subsection 4.1.} we delve into the process of how a separate opinion comes about to answer RQ1 and RQ2. While there are some formal rules, over time, informal habits and unwritten rules have developed. Secondly, in \hyperref[stances]{subsection 4.2.} we analyse the various stances that the justices take towards dissent to answer RQ3 and RQ4.

\subsection{The birth of dissent at CCC}\label{birth}

Although the procedure at the CCC is governed by the act on the CCC, it is important to map the internal process that takes place before the CCC decides to explore the potential room that justices have to break away from the majority. We discuss the chamber and the plenary proceedings somewhat separately. We start with the former. First of all, when a case arrives at CCC a judge rapporteur is randomly assigned. The judge rapporteur mostly prepares the draft of the court decision after some familiarization with the file and sends it to their colleagues for further discussion. The judge rapporteur then discusses the case with other justices. The style of communication in chambers depends on prior agreement. Some of the justices prefer to meet in person, while others communicate per e-mail. Some justices prefer a more personal approach rather than e-mail communication. However, when the case is complicated or something is problematic one of the justices may suggest a meeting in person.

\emph{„E-mail correspondence at the beginning, followed by a phase when the colleagues you have approached are asked to express themselves. So they'll respond electronically again, actually send the draft back after editing, and some will ask for in-person meetings. If that doesn't settle the matter, then the three-judge chamber will meet all together.``}

Judges present their opinion during the deliberation. It is in this phase that an idea of a dissent comes about, if one of the judges decides to break away from the majority opinion. Although the judges come prepared for the deliberation, it is unlikely that the decision of dissent had already been made prior to it. Furthermore, a separate opinion in the 3-member chamber decisions presents an interesting paradox: as explained one of the interviewees, the costs of a dissent in the chamber proceedings are very high since it forces the judge rapporteur to do more work and since it transforms the decision into a binding \emph{nález}:

\emph{„If the judge rapporteur proposes a decision on inadmissibility, and I disagree, I will force the judge rapporteur to invite the parties to make a statement, others to reply\ldots{} I feel obliged to write the dissent for the sake of it and not to let everything wear down the judge rapporteur, who suddenly has a lot more work to do, and so in that case I feel obliged to also spit out some idea and explain to some extent why the ruling is being made by way of a judgment``}

It is not only the collegiality cost that judges consider in the chamber proceedings. Strategic considerations also come into play since the dissent can transform from a non-binding decision to a binding judgment. A non-binding decision on inadmissibility requires unanimous vote. Therefore, voting against the majority and exercising a dissent turns the non-binding majority opinion into a binding opinion contained in a decision on merits:

\emph{„It's just that in some cases, I've made, and I'll say this as a frankly strategic consideration. When you exercise a dissent in the chamber, you make a denial of a resolution that is not intended to preclude a binding denied finding. So I was making strategic consideration that it was better to have the case rejected on inadmissibility than to be rejected on merits.``}

Interestingly, the strategizing judges also consider with whom they sit on the bench. The rotation of judges at the CCC was introduced in 2016. Since then, once every two years, one member of a chamber ``rotates'' into another. One of the judges revealed that they strategically wanted to resolve the case with the current composition, as the justice considered those more prominent in the legal community and cared about the academic impact of the case:

\emph{„\ldots so actually by passing the decision in the more difficult chamber and gaining support from my colleagues, had a lot more public legitimacy.``}

After signing the voting protocol, the judge rapporteur waits for and collects separate opinions and then announces the case.

The situation in plenary cases is quite different: plenary sessions are mandatory in person and planned on every Tuesday. If a judge rapporteur wants to discuss their case at a plenary meeting, they have to send the draft of the decision, including facts of the case and their proposed outcome, per e-mail a week before the planned plenary session. This is an unwritten rule, which is respected by the judges:

\emph{„There is a sort of friendly rule that if you want your case to be discussed in the plenary meeting on Tuesday, you have to send it by 12:00 on the previous Tuesday and give your colleagues a week. If it's something more complicated it is better to send it up 14 days before the meeting or so\ldots{} and then of course it's up to the President of the CCC who sets up the plenary.``}

After the judge rapporteur sends their draft, some of the judges may react to the draft by sending their comments and any suggestions to that particular case. At this stage, the potential dissenting groups for the joint dissents are already starting to take shape:

\emph{„Usually it's the judge who has some reservations before the plenary session and writes out in advance what their reservations are and sends them out to everybody else becomes the leader of the dissenting group and then most people tend to sort of join in and possibly write something of their own here and that's sort of the way that the group is formed. Whoever starts dissenting.``}

One of the judges explained how they perceive ``opinion leaders'' within these dissenting groups. It is not by any means that the ``opinion leaders'' are always the same judges. Opinion leaders, as they were described by our respondent, are willing to invest their free time in preparing and commenting on the cases of others:

\emph{„Yeah, just type-wise and personality-wise you're going to have a judge who likes to debate more, who likes to argue, who doesn't resist, who always like goes into that opinion clash and then you're always going to have a judge who listens to that and then some even maybe join in or be persuaded. Which, of course, within that 15-member body, you have to have both groups of judges because if you have 15 opinion leaders, you're never going to have anything like a decision there. If you have 14 submissive judges who just vote according to the opinion leader, you're going to have a monocratic CCC.``}

Some of the judges argue that opinion leaders are selected based on their expertise in certain area of law. For example, in a criminal case, a judge who has previous experience in criminal law will be more forthcoming. Some of our respondents presented themselves as opinion leaders, since they consider it their duty to always speak up when they disagree with the majority opinion. It can happen that the opinion leader will arise during the debate at the plenary meeting. It is not a rule that opinion leaders are only those who comment on the case in advance. The voting coalitions are also emerging at this point. The judges agree that they can infer from the debate, even before the voting, how other judges will vote since most of the judges present their opinion or at least announce with whom they agree:

\emph{„So of course if there is a professor of criminal law in the plenary session and it is a matter that, for example, can repeal part of the Criminal Code you know that the factual influence of the particular professor is higher, because after all, its their specialization.``}

\emph{„And then my experience is that after voting, when I know that four colleagues vote against the judgment like me, I always offer them: do you want to sign my dissent or not?{}``}

However, there are some exceptions, under which the judges do not speak up and only then quietly announce that they will express themselves in a separate opinion. But that happens rarely in exceptional situations. After the debate, if the chamber president finds that the case has been sufficiently discussed, he will open the vote. The unwritten rule is that judges at this stage will definitely confirm whether they will dissent or not. In the plenary proceedings, the situation is different:

\emph{„In plenary matters, of course, the situation is different and there is an unwritten rule, that a dissenting opinion must be invoked at the latest when the voting record is signed.``}

In the case of dissent, it is necessary to give dissenters enough time to write their separate opinion. In practice, judges are usually given a week to write the dissent:

\emph{„Yes, and again, we don't have any rules of procedure, but {[}\ldots.{]}, the decision is generally published at the same time as the dissent, so that there is a practice of about one week between the adoption of the decision and its publication, including the preparation and publication of the separate opinion.``}

After the vote, it is decided when the case will be announced. The dissenters should have enough time to write the dissent. A curious case occurred in the case of the repeal of the law on elections to the Chamber of Deputies Pl. US 44/17. This case was mentioned by all of our nine interviewees. The problem that arose was that the case was withheld by the judge rapporteur for three-years before issuing the final decision. After the vote, however, the president of the CCC decided that the case will be announced the next day, since it repealed part of the law eight months before the elections to the Chamber of Deputies. The dissenting judges criticized the lack of time given to them. That then led to new agreement between judges that a week should be guaranteed at minimum for writing a dissent.

A judge rapporteur may lose their case. In that scenario, the chairman put the case forward to the „most convincing opinion leader'' (who got the most votes) that becomes judge rapporteur. In this case, the judge rapporteur may use their rejected draft as a basis for dissent and the opinion leader of the dissent group becomes the new judge rapporteur.

\emph{„It may be that those comments will convince the majority of the plenary to change the decision and also the rapporteur-judge, and therefore you will very likely be the new rapporteur, because the president will assume that you are prepared for this, since you presented all the arguments. You've probably already thought this through like you've been working on it.``}

To sum up, we identify these types of judges during the coalition making of a decision: (1) an original judge rapporteur, (2) opinion leaders(s), (3) president of the CCC (who presides over a session of the court) and (4) the remaining judges who join in the opinion of either judge rapporteur or one of the opinion leaders, or they express their own opinion, which no one joins. After the vote, the opinion leaders become dissent leaders or in case of a change of judge rapporteur the opinion leader becomes the new judge rapporteur, while the old judge rapporteur may become the dissenting leader. After the vote, dissent leader(s) ask other colleagues whether they want to join. Most of the judges perceive joint dissent as an advantage.

\emph{„I am of the opinion that the dissents are stronger and more understandable if the judges, who share the same opinion, write it jointly. I think it should be done in such a way that they process it together, precisely so that everyone projects a part of their own sub-argument. It is much more understandable for the public and experts rather than having 6 separate dissents. ``}

However even here we can spot some exceptions.

\emph{Researcher: \ldots. and has it ever happened to you that maybe a judge wanted to have a solitary dissent and didn't want to invite you?}
\emph{Interviewee: Yeah as far as I recall there was a situation like that with one of the judges.}

After deciding who will write with whom the dissent, the judges get down to writing. The judges perceive greater literary freedom in the possibility of a solitary dissent, which aligns with the finding of Tom S. Clark \& Benjamin Lauderdale\footnote{\emph{Locating {Supreme Court Opinions} in {Doctrine Space}}, 54 \textsc{American Journal of Political Science} 871 (2010).} that separate opinions correspond the most to the political position of SCOTUS judges. Joint dissents are most often written by the dissent leader. The remaining judges in the dissenting coalition mainly comment on it or make suggestions. In a few cases, there has also been joint writing of dissents (each writing a specific portion).

\emph{„Sometimes there is simply such a time pressure that I rely on other judges to express their opinion in the dissent. And then I join their dissent.``}

Researcher: \emph{„How does the arrangement for writing joint dissents take place?{}``}

Interviewee: \emph{„This happens in such a way that basically the judge who is the most active writes the dissent. And then, from my experience, the practice is that when they then write a dissent and four colleagues voted alike, they are usually offered, at least I always offer if they would like to sign the dissent. You can recognize the main author of the dissent by the order of the names in the dissent.``}

\subsection{The fan, hater and strategist}\label{stances}

In the interviews, we were able to identify three overarching types of attitude towards dissent among judges: the fan, the hater and the strategist. The main difference between these three groups lies in the degree of their norm-identification, or their internal attitude on how the court should act externally. Haters of dissent argue that the CCC should not outwardly appear disunited by any dissents and instead should try to appear as united and unanimous as possible. These judges adhere to the norm of consensus to full extent. It is possible to find dissents even from the haters. They differ in the degree of disagreement required for them to dissent: simply put a hater needs a much greater degree of disagreement on a bench to exercise dissent, whereas the threshold of the fans is lower, since a disunited court externally is not perceived as a major problem - on the contrary, they perceives the plurality of opinions positively. Strategists are somewhere in the middle of both these views. They are neither completely open to dissent nor completely closed to it. Simply put, they dissent where it strategically suits them. We now present an example for each type and then delve deeper into each type.

An example of a hater: \emph{„I don't like them (dissents). {[}\ldots{]} I believe that when a decision is taken by a majority, it is not to comment further on some B that someone thought otherwise.``}

An example of a fan: \emph{„I'm a big fan of dissent {[}\ldots{]} And I can safely say that so far, I've been on the CCC for many years now, I've dissented every time I've voted no. It's a sign of fairness to explain even outwardly why I didn't support the majority opinion here.``}

An example of a strategist: \emph{„I think that dissent makes sense if one expresses a fundamental position or a fundamental opinion. I am motivated by a different legal opinion or the hope that it may lead to a change in either case law or legislation in the future.``}

Indeed we are aware that these three types are a simplified model that does not explain all the variance in the dissenting behavior at the CCC. Even the judges from the group of those who don't like dissent, dissent in some cases. There is not a single judge without a dissent at CCC. One of the factors that could explain that is the previous profession of constitutional judge. When the CCC judge is coming from the ordinary court system, they are not used to dissent unless they were a judge at the Supreme Court or Supreme Administrative Court for only the SAC allows for dissents under specific and narrow circumstances, otherwise judges in Czechia are not legally allowed to attach separate opinions. For a career judge one of the main aspects of judicial decision-making is to create legal certainty for the public. A dissenting opinion, from their point of view, undermines this certainty.

\emph{„Basically, from my point of view, it undermines the authority of the court to some extent. It was simply decided by majority vote, period.``}

So how is it possible that even these judges sometimes exercise a dissent? In one response, two factors played a large role for the decision to overcome their aversion towards dissenting: 1) a high level of disagreement and 2) the possibility to join the dissent of another judge who shares a similar view. One of the judges explained to us:

\emph{„Whether it's a value\ldots{} so yeah I made an agreement with a colleague who felt the need to write the dissent. I would read it to consider whether or not to join.``}

Furthermore judges who like to dissent and dissent as much as possible have different reasons why they do so. They manage to dissent every time that they vote against the majority opinion. However, that does not imply that they are always the authors of the dissents, since it would be nearly impossible due to time management and the amount of workload at the CCC. It's almost as if the option to explain their differing view is perceived as an obligation:

Interviewee: \emph{„It may not be a legal obligation, but I feel it's my professional obligation to write a dissent; from a legal standpoint it is a judge's possibility not a duty, but I feel it's an ethical duty for a judge to always write a dissent if they vote no.``}

Interviewee: \emph{„It should simply be the principle that a judge reveals what they were against.``}

There are four main aspects that further explain the decision of CCC judges whether to dissent: 1) previous experience, 2) emotional valve, 3) caseload and 4) importance of the case. Previous experience, caseload, as well as the importance of the case flow from our theory and we have already tested them empirically\footnote{ANONYMIZED}, on the other hand, the need to release emotions has not been thematized in the literature on dissent.

\subsubsection{Previous experience}\label{previous-experience}

The CCC as an apex court represents for a share of Czech lawyers the pinnacle of their careers. The President of the Czech Republic nominates candidates for CCC judges and the Senate approves them. As a rule, the president should nominate as diverse a constitutional court as possible - they seek diversity not only in terms of gender but also in terms of profession. The CCC judges have in the past encompassed academics, lawyers, but also judges of higher and lower courts. The professional background of CCC judges shapes their stance and views towards the role and functioning of the CCC and it also influences the familiarity among CCC judges - some have already met at a district court or at a department at university, some of them don't know each other. That opens up an interesting dynamic that develops over time as the newly appointed justices have to get accustomed to their new position as well as to their new colleagues:

\emph{„Often those social relationships come from some previous work - I don't know, I know some of my colleagues from the university department.``}

One interviewee revealed that judges perceive the previous profession of their colleagues and at the same time transfer their knowledge from the former profession to the current one.

Interviewee: \emph{„Then an academic comes along at CCC and they are used to writing academic papers and that's a bit different than writing judgments or dissent. In an academic paper you're completely free, you can just write whatever you want and you're the only signed under it and it actually influences the academic discussion, but it doesn't directly affect people's fates.``}

Interviewee: \emph{„Many colleagues see the judgment as some kind of scientific work. For me, as a common judge, it's a judgment\ldots{} the basis of judicial work is to respect the majority ruling. If the majority outvotes me, I obey and secondly, it is equally basic to judicial work to respect the binding legal opinion of a higher authority.``}

Interviewee: \emph{„Obviously, for example, if somebody is not a judge and now they're actually judging here for the first time, it's a bit of a problem for them to learn the procedures at a court.``}

There are three additional sub-themes that surfaced during our interviews and that we believe deserve further inquiry that is beyond the scope of our paper. First, it appears that judges from lower courts are less likely to dissent since they are used to the binding nature of the decision of an appellate court and at the same time they are more aware of the aspect of undermining the authority of the court by dissent. Since this is very difficult to verify quantitatively in the Czech context, because the number of judges from non-supreme courts is minimal, the in-depth interviews suggest that that may indeed be the case:

\emph{„Because I was a routine judge I believe that if a decision was made by a majority, no one should further comment further on some other solutions or that he thought otherwise.``}

Secondly, it appears that the judges who come from the Supreme Court and Supreme Administrative Court perceive this aspect differently as they come from a court at the top of the hierarchy. Moreover, judges at higher court have the power to determine what direction the case law should take, while lower court judges must learn to obey the higher courts. That situation was described by one of our respondents as follows:

\emph{„If you're asking about the judicial career, it's related to the fact that as you grow to the higher levels, then of course you're more interested in influencing the jurisprudence with your judgments``}

Lastly, it appears that judges with background in the academia and without experience as a judge have the ambition to take on an academic dimension in the decision/judgment. At the same time, one can also expect more freedom in the writing of dissents:

\emph{„Academics have it differently. They're able to sort of overlook the verdict and have the ambition to sort of show like in the judicial opinion or in the dissents, express ideas beyond the court's decision.``}

\subsubsection{Emotional vent}\label{emotional-vent}

Every interviewee touched upon the topic of emotions and frustration resulting from a plenary or chamber discussion. A portion of the CCC judges admitted that frustration sometimes leads them to writing a separate opinion. Dealing with emotions has proven to be an important aspect of their stance towards dissents. In a fundamental disagreement, a judge does not need to bottle up their emotions but they can just vent them out through dissent.

Researcher: ``\emph{So if I may start - what is the meaning of a written dissent for you?}
Interviewee: \emph{„It's a kind of relief when we don't agree, for example, on the reasoning --- it's a safety valve so that we can all work together.``}

The CCC judges agreed that a highly emotionally charged separate opinion is more readily and easily written alone than together with others because a solitary separate opinion does not have to be approved by anyone else:

\emph{„Honestly when you write it yourself you are making less compromises.``}

Some judges revealed that they do not experience the need to vent emotions. Interestingly, those who do not experience such frustrations reminisced about their previous judicial careers:

\emph{„I'd have to hang myself if I was that emotional in the justice system. Of course, when I was young and stupid, the court of appeal dismissed some of my cases. So I read it and wrote it like they wanted from me, period.``}

Interesting research questions emerge from the interviews: To what extent is the dissenting behavior influenced by the perceived role of dissents as an avenue for venting emotions? What factors influence the variance of the role of emotions venting through a dissent? To our knowledge, such a question has not been thoroughly empirically researched and opens up an interesting research avenue. Does it depend on the previous professional experience as some interviewees seem to hint? We present a couple of fragments that enlighten this aspect of a dissent. It follows from the interviews that emotions may lead to strong personal expressions in dissent. Most held the view that dissent should not interfere on a personal level and argued that that would be strictly unethical and unprofessional:

Researcher: \emph{„You said that it can improve the relationships in that court, and often there are dissents that seem to me to attack the judge rapporteur rather than the the decision itself``}
Interviewee: \emph{„I'll stop you there, that's a total ethical mistake. You don't write dissent at all by saying what they did was wrong. Dissent to me is always offering a different way of dealing with it. I mean, I've never, I don't think any dissent of mine has ever been as impolite, because I've actually just offered an alternative solution. And even I've avoided that sort of ``my colleagues are a disappointment to me'' and so on and so on, maybe at some point like that, but it's certainly not what I was primarily writing, but just that maybe I was joining someone.``}

But some defended opposite approach with the following argument:

\emph{„Someone said that simply the dissent is lost. Yeah I just failed to get a majority for my arguments, but when there is a loss it should be as fair. So I recognize that if it's not fair, there's an opportunity to argue back on a more personal level.``}

Those judges simply feel that a procedural injustice was done against them; that the only way to draw attention to this issue is to write a separate opinion that is intended to inform the public of some injustice that has taken place in the Constitutional Court. Therefore, a less calculated, less strategic dimension of the emotional plane emerges. To the best of our knowledge, no previous empirical study on dissenting behavior has considered this dimension.\footnote{Benjamin Bricker, \emph{Breaking the {Principle} of {Secrecy}: {An Examination} of {Judicial Dissent} in the {European Constitutional Courts}}, 39 \textsc{Law \& Policy} 170 (2017); Hanretty, \emph{supra} note 21; Hanretty, \emph{supra} note 2; Epstein, Landes, and Posner, \emph{supra} note 3; Garoupa and Grajzl, \emph{supra} note 4; Nuno Garoupa \& Catarina Santos Botelho, \emph{Judicial {Dissent} in {Collegial Courts}: {Theory} and {Evidence}}, \emph{in} \textsc{Oxford {Research Encyclopedia} of {Politics}} (2022); Nuno Garoupa, Laura Salamero-Teixidó \& Adrián Segura, \emph{Disagreeing in Private or Dissenting in Public: An Empirical Exploration of Possible Motivations}, 53 \textsc{Eur J Law Econ} 147 (2022); \textsc{Wittig}, \emph{supra} note 12; Henrik Litleré Bentsen, Mark Jonathan McKenzie \& Jon Kåre Skiple, \emph{Explaining {Dissent Rates} on a {Consensual Danish Supreme Court}}, \textsc{Open Judicial Politics} (2021).}

\subsubsection{Workload}\label{workload}

The importance of leisure and workload was already discussed in the \hyperref[theory-dissent]{section 2.1.}. The question remains how is the role of workload perceived by individual judges at CCC? The haters of dissent are more likely to spend time on their cases rather than writing dissent. In contrast to that, the fans of dissent are more likely to ignore their workload. The strategists weigh and decide whether they can afford to dissent regarding the amount of caseload he is dealing with at the moment.

A hater: \emph{„Well, I don't have time to write dissents and stuff like that (laughs).``}
A fan: \emph{„I dissent every time.``}
A strategist: \emph{„If I'm against it but it's not worth a dissent, either because I don't consider it so fundamental, or that or just for completely prosaic reasons that I just don't have the time.``}

We therefore reveal that the typically researched effect of workload may vary across certain clusters of judges. To our knowledge this heterogeneity of the effect has not been studied at all. Our interviews therefore open another potential research avenue.

Furthermore, we revealed a collegiality effect regarding caseload. The CCC judges take into account whether their dissent imputes extra work on the judge rapporteur. Extra work can be a situation, in which the judge rapporteur is forced to rewrite an inadmissibility decision to a judgment as the law requires the admissibility decisions to be unanimous. Another situation occurs when a dissent causes delays in the proceedings. The dissenting judge tries as much as possible to convince the judge-reporter of their view, which ultimately delays the entire process of the decision-making and subsequent announcement of the judgment. Such time delays, which can lead to an increase in the workload of both judges, lead the dissenting judge to exercise dissent as their reputation among their colleagues could suffer.

\emph{„If I force the judge-rapporteur to rewrite an inadmissibility decision I feel obliged to write the dissent just for the sake of it. He suddenly has far more work, and in that case I feel obliged to explain to a certain extent why I chose to vote differently than the majority.``}

The collegiality and effort costs also come into play in joint separate opinions. Some judges admitted to having been agreed in advance to take turns writing dissents when voting together or they simply ask each other to write the separate opinion for time reasons. Long-term joint dissenting collaborations arise from frequent forms of voting and consensus.\footnote{Gor Vartazaryan, \emph{Sít'ová Analỳza Disentujících Ústavních Soudců}, \textsc{Pravnik} (2022).}

\emph{„We decided who would write the dissent and then we communicated among ourselves and sent it to the other judges. We took the comments into account, so like then it was a collective work, but like every text, somebody just has to write the basis.``}

\subsubsection{Importance of the case}\label{importance-of-the-case}

There is not a single judge of CCC with zero separate opinions. Tab. \ref{tab:dissent-rate} reveals the number of times that the 3rd term CCC judges exercised the dissent.\footnote{The data are obtained from the CCC database presented in Štěpán Paulík, \emph{The {Czech Constitutional Court Database}}, Forthcoming \textsc{Journal of Law and Courts} (2024).}

\begin{table}

\caption{\label{tab:dissent-rate}A table showing the number of times a judge dissented and the number of times they were part of the bench deciding a case. The data encompass only the 2013-2023 third term of the CCC.}
\centering
\begin{tabular}[t]{l|r|r}
\hline
Name & Dissents & Decisions\\
\hline
David Uhlíř & 24 & 7894\\
\hline
Jan Filip & 61 & 9004\\
\hline
Jan Musil & 17 & 5548\\
\hline
Jaromír Jirsa & 16 & 7769\\
\hline
Jaroslav Fenyk & 18 & 2897\\
\hline
Jiří Zemánek & 27 & 9552\\
\hline
Josef Fiala & 50 & 7454\\
\hline
Kateřina Šimáčková & 53 & 8038\\
\hline
Ludvík David & 43 & 8836\\
\hline
Milada Tomková & 13 & 2350\\
\hline
Pavel Rychetský & 14 & 2773\\
\hline
Radovan Suchánek & 73 & 9185\\
\hline
Tomáš Lichovník & 6 & 8530\\
\hline
Vladimír Sládeček & 46 & 8882\\
\hline
Vojtěch Šimíček & 46 & 8398\\
\hline
\end{tabular}
\end{table}

It appears that the importance of a case is another key aspect behind the decision whether to dissent. The factor of importance of a case is rather subjective, since the boundaries of importance of the individual case are partly dependent on the perception of the individual judges. There is not an objective definition that unites the undecided judges. Their views on which case is important differ:

\emph{„I think dissent is a fairly strong expression of disagreement, and often one disagrees over less substantive things. I think that dissent is meaningful if one is expressing a fundamental position or a fundamental opinion.``}

\emph{„It is really just supposed to be a question of legal opinion, strong legal opinion, not some kind of impressionology that I think something and it is a big question whether to bring into it the way of, for example, making that decision.``}

\emph{„Really depends on the particular case. I distinguish the cases by how I percieve them, of course. There may be someone else who will say that this is more important than that. I judge it by what I think of it and if I then pay more attention to it, but that doesn't mean I pay less attention to less important cases.``}

\emph{``Well, that's it, of course it's less important. Even there they can create important things. That's hard to say. I distinguish the dissents according to how I see them, of course. It can be someone else, they can simply say differently that it is not so important. I judge it by what I think about it and if I pay more attention to it, but that doesn't mean that I don't pay less attention to it.''}

Despite the impossibility of discerning a pattern from the interviews that would reveal what judges consider as an important case, one insight is often repeated. That is that plenary cases are more influential in terms of social impact. As already mentioned, the CCC may repeal part of a law or even the entire law in a plenary decision. In contrast to that in the 3-member chamber cases the CCC decides on individual constitutional complaints.

\emph{„When it's some principled matter of value especially in plenary cases. In the chamber decision I never dissent. Whether it's about plenary or some core constitutional values, yeah I agreed with a colleague who felt the need to write the dissent that I would read it to consider whether or not to join in.``}

\emph{„The plenary cases have a much greater significance and reach.''}

\section{Conclusion}\label{conclusion}

Our study has provided a deeper understanding of the dissenting behavior of the CCC judges, drawing on a qualitative analysis of interviews with nine judges. We revealed that the decision to dissent is shaped by multiple factors, including professional background, emotional regulation, caseload, and the perceived significance of the case. Based on the identification-disagreement model, we grouped the judges into 3 clusters depending on their stance towards exercising a dissent: ``haters,'' ``fans,'' and ``strategists,'' each exhibiting distinct approaches to dissent based on their alignment with or opposition to the norm of consensus.

Our findings suggest that while some judges are guided by intrinsic and expressive utilities---motivated by personal values or the desire to showcase individuality---others prioritize reputational utility, aligning their behavior with the court's norm of consensus and their understanding of the role of a court. Additionally, we have discussed the dynamics of joint versus solitary dissent, with strategic considerations influencing the choice between collaborative and individual separate opinions. The introduction of chamber rotation has also altered the institutional context, by which the dissent is shaped at the CCC, reducing the long-term impact of collegiality costs.

In summary, this study contributes to the broader discourse on judicial dissent and adds a more detailed insight to our previous large-n quantitative study. The main potential research avenue that the interviews opened is the question of what role does the emotional venting play in the decision whether to exercise or to avert a dissent.

\section*{Funding Statement}\label{funding-statement}
\addcontentsline{toc}{section}{Funding Statement}

The study was supported by ANONYMIZED.

\end{document}
