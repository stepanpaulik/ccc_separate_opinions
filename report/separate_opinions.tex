% ARTICLE 2 ----
% This is just here so I know exactly what I'm looking at in Rstudio when messing with stuff.
% Options for packages loaded elsewhere
\PassOptionsToPackage{unicode}{hyperref}
\PassOptionsToPackage{hyphens}{url}
%
\documentclass[
  11pt,
]{article}
\usepackage{lmodern}
\usepackage{amssymb,amsmath}
\usepackage{ifxetex,ifluatex}
\ifnum 0\ifxetex 1\fi\ifluatex 1\fi=0 % if pdftex
  \usepackage[T1]{fontenc}
  \usepackage[utf8]{inputenc}
  \usepackage{textcomp} % provide euro and other symbols
\else % if luatex or xetex
  \usepackage{unicode-math}
  \defaultfontfeatures{Scale=MatchLowercase}
  \defaultfontfeatures[\rmfamily]{Ligatures=TeX,Scale=1}
\fi
% Use upquote if available, for straight quotes in verbatim environments
\IfFileExists{upquote.sty}{\usepackage{upquote}}{}
\IfFileExists{microtype.sty}{% use microtype if available
  \usepackage[]{microtype}
  \UseMicrotypeSet[protrusion]{basicmath} % disable protrusion for tt fonts
}{}
\makeatletter
\@ifundefined{KOMAClassName}{% if non-KOMA class
  \IfFileExists{parskip.sty}{%
    \usepackage{parskip}
  }{% else
    \setlength{\parindent}{0pt}
    \setlength{\parskip}{6pt plus 2pt minus 1pt}
    }
}{% if KOMA class
  \KOMAoptions{parskip=half}}
\makeatother
\usepackage{xcolor}
\IfFileExists{xurl.sty}{\usepackage{xurl}}{} % add URL line breaks if available
\urlstyle{same} % disable monospaced font for URLs
\usepackage[margin=1in]{geometry}
\usepackage{graphicx}
\makeatletter
\def\maxwidth{\ifdim\Gin@nat@width>\linewidth\linewidth\else\Gin@nat@width\fi}
\def\maxheight{\ifdim\Gin@nat@height>\textheight\textheight\else\Gin@nat@height\fi}
\makeatother
% Scale images if necessary, so that they will not overflow the page
% margins by default, and it is still possible to overwrite the defaults
% using explicit options in \includegraphics[width, height, ...]{}
\setkeys{Gin}{width=\maxwidth,height=\maxheight,keepaspectratio}
% Set default figure placement to htbp
\makeatletter
\def\fps@figure{htbp}
\makeatother
\setlength{\emergencystretch}{3em} % prevent overfull lines
\providecommand{\tightlist}{%
  \setlength{\itemsep}{0pt}\setlength{\parskip}{0pt}}
\setcounter{secnumdepth}{5}

\ifluatex
  \usepackage{selnolig}  % disable illegal ligatures
\fi
\newlength{\cslhangindent}
\setlength{\cslhangindent}{1.5em}
\newlength{\csllabelwidth}
\setlength{\csllabelwidth}{3em}
\newenvironment{CSLReferences}[2] % #1 hanging-ident, #2 entry spacing
 {% don't indent paragraphs
  \setlength{\parindent}{0pt}
  % turn on hanging indent if param 1 is 1
  \ifodd #1 \everypar{\setlength{\hangindent}{\cslhangindent}}\ignorespaces\fi
  % set entry spacing
  \ifnum #2 > 0
  \setlength{\parskip}{#2\baselineskip}
  \fi
 }%
 {}
\usepackage{calc}
\newcommand{\CSLBlock}[1]{#1\hfill\break}
\newcommand{\CSLLeftMargin}[1]{\parbox[t]{\csllabelwidth}{#1}}
\newcommand{\CSLRightInline}[1]{\parbox[t]{\linewidth - \csllabelwidth}{#1}\break}
\newcommand{\CSLIndent}[1]{\hspace{\cslhangindent}#1}


\title{Disagreement and dissent on a bench: a quantitative empirical
analysis of the Czech Constitutional Court\thanks{Replication files are
available on the author's Github account
(\url{https://github.com/stepanpaulik/apex_courts_dataset/}).
\textbf{Current version}: January 19, 2024}}
\author{true \and true}
\date{January 19, 2024}

% Jesus, okay, everything above this comment is default Pandoc LaTeX template. -----
% ----------------------------------------------------------------------------------
% I think I had assumed beamer and LaTex were somehow different templates.


\usepackage{kantlipsum}

\usepackage{abstract}
\renewcommand{\abstractname}{}    % clear the title
\renewcommand{\absnamepos}{empty} % originally center

\renewenvironment{abstract}
 {{%
    \setlength{\leftmargin}{0mm}
    \setlength{\rightmargin}{\leftmargin}%
  }%
  \relax}
 {\endlist}

\makeatletter
\def\@maketitle{%
  \newpage
%  \null
%  \vskip 2em%
%  \begin{center}%
  \let \footnote \thanks
      {\fontsize{18}{20}\selectfont\raggedright  \setlength{\parindent}{0pt} \@title \par}
    }
%\fi
\makeatother


\title{Disagreement and dissent on a bench: a quantitative empirical
analysis of the Czech Constitutional Court\thanks{Replication files are
available on the author's Github account
(\url{https://github.com/stepanpaulik/apex_courts_dataset/}).
\textbf{Current version}: January 19, 2024}  }

\date{}

\usepackage{titlesec}

% 
\titleformat*{\section}{\large\bfseries}
\titleformat*{\subsection}{\normalsize\itshape} % \small\uppercase
\titleformat*{\subsubsection}{\normalsize\itshape}
\titleformat*{\paragraph}{\normalsize\itshape}
\titleformat*{\subparagraph}{\normalsize\itshape}

% add some other packages ----------

% \usepackage{multicol}
% This should regulate where figures float
% See: https://tex.stackexchange.com/questions/2275/keeping-tables-figures-close-to-where-they-are-mentioned
\usepackage[section]{placeins}



\makeatletter
\@ifpackageloaded{hyperref}{}{%
\ifxetex
  \PassOptionsToPackage{hyphens}{url}\usepackage[setpagesize=false, % page size defined by xetex
              unicode=false, % unicode breaks when used with xetex
              xetex]{hyperref}
\else
  \PassOptionsToPackage{hyphens}{url}\usepackage[draft,unicode=true]{hyperref}
\fi
}

\@ifpackageloaded{color}{
    \PassOptionsToPackage{usenames,dvipsnames}{color}
}{%
    \usepackage[usenames,dvipsnames]{color}
}
\makeatother
\hypersetup{breaklinks=true,
            bookmarks=true,
            pdfauthor={Štěpán Paulík (Humboldt Universität zu Berlin,
\href{mailto:stepan.paulik.1@hu-berlin.de}{\nolinkurl{stepan.paulik.1@hu-berlin.de}}) and Gor
Vartazaryan (Charles University,
\href{mailto:gorike2000@gmail.com}{\nolinkurl{gorike2000@gmail.com}})},
             pdfkeywords = {empirical legal research, courts, dissents,
judicial behavior, political science, regression analysis},
            pdftitle={Disagreement and dissent on a bench: a
quantitative empirical analysis of the Czech Constitutional Court},
            colorlinks=true,
            citecolor=blue,
            urlcolor=blue,
            linkcolor=magenta,
            pdfborder={0 0 0}}
\urlstyle{same}  % don't use monospace font for urls

% Add an option for endnotes. -----



% This will better treat References as a section when using natbib
% https://tex.stackexchange.com/questions/49962/bibliography-title-fontsize-problem-with-bibtex-and-the-natbib-package

% set default figure placement to htbp
\makeatletter
\def\fps@figure{htbp}
\makeatother



\usepackage{longtable}
\LTcapwidth=.95\textwidth
\linespread{1.05}
\usepackage{hyperref}
\usepackage{float}
\usepackage{booktabs}
\usepackage{longtable}
\usepackage{array}
\usepackage{multirow}
\usepackage{wrapfig}
\usepackage{float}
\usepackage{colortbl}
\usepackage{pdflscape}
\usepackage{tabu}
\usepackage{threeparttable}
\usepackage{threeparttablex}
\usepackage[normalem]{ulem}
\usepackage{makecell}
\usepackage{xcolor}

\newtheorem{hypothesis}{Hypothesis}

\usepackage{setspace}

% trick for moving figures to back of document
% really wish we'd knock this shit off with moving tables/figures to back of document
% but, alas...

% 
% Optional code chunks ------
% SOURCE: https://stackoverflow.com/questions/50702942/does-rmarkdown-allow-captions-and-references-for-code-chunks



\begin{document}

% \textsf{\textbf{This is sans-serif bold text.}}
% \textbf{\textsf{This is bold sans-serif text.}}


% \maketitle

{% \usefont{T1}{pnc}{m}{n}
\setlength{\parindent}{0pt}
\thispagestyle{plain}
{%\fontsize{18}{20}\selectfont\raggedright
\maketitle  % title \par

}




{
   \vskip 13.5pt\relax \normalsize\fontsize{11}{12}
   \MakeUppercase{Štěpán Paulík}, \small{Humboldt Universität zu Berlin,
\href{mailto:stepan.paulik.1@hu-berlin.de}{\nolinkurl{stepan.paulik.1@hu-berlin.de}}}   \par \vskip -3.5pt \MakeUppercase{Gor
Vartazaryan}, \small{Charles University,
\href{mailto:gorike2000@gmail.com}{\nolinkurl{gorike2000@gmail.com}}}   

}

}








\begin{abstract}

%    \hbox{\vrule height .2pt width 39.14pc}

    \vskip 8.5pt % \small

\noindent \small{XXX}


\vskip 8.5pt \noindent \emph{Keywords}: empirical legal research,
courts, dissents, judicial behavior, political science, regression
analysis \par

%    \hbox{\vrule height .2pt width 39.14pc}



\end{abstract}


\vskip -8.5pt

{
\hypersetup{linkcolor=black}
\setcounter{tocdepth}{2}
\tableofcontents
}

 % removetitleabstract

{
\setcounter{tocdepth}{2}
\tableofcontents
}

\setlength{\parindent}{16pt}
\setlength{\parskip}{0pt}

% We'll put doublespacing here
\doublespacing
% Remember to cut it out later before bib
\hypertarget{introduction}{%
\section{Introduction}\label{introduction}}

Empirical legal research has been slowly but surely finding it's outside
the predominant US context. Historically though most of the empirical
studies have been conducted in the US, especially the Supreme Court,
context (such as
\protect\hyperlink{ref-boydUntanglingCausalEffects2010}{Boyd, Epstein,
and Martin 2010};
\protect\hyperlink{ref-carrubbaWhoControlsContent2012}{Carrubba et al.
2012}; \protect\hyperlink{ref-epsteinWhyWhenJudges2011}{Epstein, Landes,
and Posner 2011}). We now know that judgments are what judges had for a
breakfast. Put less pompously, there are many theories and approaches
for explanation of judicial behavior
(\protect\hyperlink{ref-posnerHowJudgesThink2010}{Posner 2010}). What we
do not know is the extent to which these theories and explanations carry
over to other legal systems and context.

Although it has been traditionally espoused that there has been a divide
between the empirically oriented US legal scholarship, stemming from a
different perception of the role of courts and judges, and the rest of
the world (\protect\hyperlink{ref-hamannGermanFederalCourts2019}{Hamann
2019, 416}). Therein the judges empirically researched whether and to
what extent they behave as for example political
(\protect\hyperlink{ref-carrubbaWhoControlsContent2012}{Carrubba et al.
2012}; \protect\hyperlink{ref-clarkLocatingSupremeCourt2010}{Clark and
Lauderdale 2010};
\protect\hyperlink{ref-epsteinChoicesJusticesMake1997}{Epstein and
Knight 1997};
\protect\hyperlink{ref-lauderdaleScalingPoliticallyMeaningful2014}{Lauderdale
and Clark 2014};
\protect\hyperlink{ref-sunsteinAreJudgesPolitical2006}{Sunstein et al.
2006}) or strategic
(\protect\hyperlink{ref-cameronChapterWhatJudges2017}{Cameron and
Kornhauser 2017};
\protect\hyperlink{ref-clarkEstimatingEffectLeisure2018}{Clark, Engst,
and Staton 2018};
\protect\hyperlink{ref-epsteinWhyWhenJudges2011}{Epstein, Landes, and
Posner 2011};
\protect\hyperlink{ref-epsteinStrategicRevolutionJudicial2000}{Epstein
and Knight 2000};
\protect\hyperlink{ref-kornhauserModelingCollegialCourts1992}{Kornhauser
1992b},
\protect\hyperlink{ref-kornhauserModelingCollegialCourts1992a}{1992a};
\protect\hyperlink{ref-posnerWhatJudgesJustices1993}{Posner 1993},
\protect\hyperlink{ref-posnerHowJudgesThink2010}{2010};
\protect\hyperlink{ref-rousseyOverburdenedJudges2018}{Roussey and
Soubeyran 2018}) actors.

In contrast to, especially in European legal systems, such as the one at
hand - Czechia, judges have been perceived as ``proclaimers of law'' and
the law handed down by them
(\protect\hyperlink{ref-hamannGermanFederalCourts2019}{Hamann 2019,
417}). Such a view had hindered empirical legal research in Europe. The
lack of empirical legal research can be partially blamed on lack of high
quality data, a prerequisite for any quantitative empirical research. At
least so the story goes until recently. The interest in empirical legal
studies has picked up in the last years across the whole continent,
including studies on plethora of topics within Germany
(\protect\hyperlink{ref-arnoldScalingCourtDecisions2023}{Arnold, Engst,
and Gschwend 2023};
\protect\hyperlink{ref-coupetteQuantitativeRechtswissenschaft2018}{Coupette
and Fleckner 2018};
\protect\hyperlink{ref-engstEinflussParteinaheAuf2017}{Engst et al.
2017};
\protect\hyperlink{ref-wittigOccurrenceSeparateOpinions2016}{Wittig
2016}), Spain and Portugal
(\protect\hyperlink{ref-hanrettyDissentIberiaIdeal2012}{Hanretty 2012}),
the UK
(\protect\hyperlink{ref-hanrettyCourtSpecialistsJudicial2020}{Hanretty
2020}) or the EU institutions
(\protect\hyperlink{ref-bielenBacklogsLitigationRates2018}{Bielen et al.
2018}; \protect\hyperlink{ref-brekkeThatOrderHow2023}{Brekke et al.
2023}; \protect\hyperlink{ref-fjelstulHowChamberSystem2023}{J. Fjelstul
2023}; \protect\hyperlink{ref-fjelstulEvolutionEuropeanUnion2019}{Joshua
C. Fjelstul 2019};
\protect\hyperlink{ref-fjelstulTimelyAdministrationJustice2022}{Joshua
C. Fjelstul, Gabel, and Carrubba 2022}).

In our article, we set out to conduct an empirical research into the
circumstances of disagreement on a court bench, more specifically
whether Czech constitutional justices behave strategically in when and
under what circumstances they dissent and whether there is an interplay
between the behavior at different institutional level within the Czech
Constitutional Court (``CCC'').

Our research is loosely inspired by a similar research by Epstein,
Landes, and Posner
(\protect\hyperlink{ref-epsteinWhyWhenJudges2011}{2011}), who studied
under which circumstances do US judges generally dissent. More
specifically, they built a formal economic model based on the strategic
account of judicial behavior. In particular, they tested the dependence
of dissent rate on workload, the dependence of dissent rate on size of
courts, the dependence of dissent rate on the ideological distance, and
the dependence of length of majority argumentation on the presence of a
dissenting opinion. In our study, we test our hypotheses adopted to the
CCC context that are nonetheless based on similar theoretical grounds.

We adapt the theories constructed in the US context to the civil law and
Czech judiciary contexts. We test whether the length of majority
argumentation depends on the presence of one or more dissents, whether
the workload of a judge affects their dissenting behavior, whether the
dissenting behavior of judges changes at the start and end of their
terms, and, lastly, whether relationships formed during the plenary
sessions, as posited by the Czech legal scholarship, carry over to
3-member panel proceedings.

We find that a dissent imputes costs on the majority that produces
longer arguments to address a dissent. The effect is stronger the more
disagreement there is on the bench. We find that the workload of a judge
does decrease the likelihood of dissent. Moreover, our analysis
corresponds to the theory that dissents bring about significant
collegiality costs for the dissenter. Lastly, we reveal similar trends
in behavior of judicial coalitions from plenary proceedings also in the
3-member panel proceedings.

Our article proceeds as follows. We start out with a theory. We explain
the main differences between the expectations based on the theory in the
CCC context in comparison to the SCOTUS context and based on that we
draw the hypotheses for the empirical part. We briefly explain the
choice of our broad methodological framework: the Bayesian statistics.
We proceed to test the hypotheses in empirical part divided into
sections one per each hypothesis. We discuss the pitfalls of our
research and potential room for improvement afterwards. Lastly, we
conclude with a summary of our findings.

A separate opinion is a statement following the main ruling, in which a
judge expresses the dissatisfacation with the decision by elaborating on
points of the majority decision they disagree with
(\protect\hyperlink{ref-wittigOccurrenceSeparateOpinions2016}{Wittig
2016, 57}).

\hypertarget{theory}{%
\section{Theory}\label{theory}}

In general, there are multiple accounts of behavior of judges'. The
first that had dominated until \textasciitilde the end of 20th century
posited that judges are policy oriented. A lot of research has been
conducted on whether, how and to what extent do judges indeed seek to
advance the policies they desire
(\protect\hyperlink{ref-berdejoElectoralCyclesUS2017}{Berdejó and Chen
2017}; \protect\hyperlink{ref-clarkLocatingSupremeCourt2010}{Clark and
Lauderdale 2010};
\protect\hyperlink{ref-dworkinPoliticalJudgesRule1980}{Dworkin 1980};
\protect\hyperlink{ref-kastellecEmpiricallyEvaluatingCountermajoritarian2016}{Kastellec
2016}; \protect\hyperlink{ref-moyerJudicialInnovationSexual2012}{Moyer
and Tankersley 2012}).

However, as of recently, the perspective on judges has shifted. Judges
are now allegedly strategic and rational actors. One of the early
pioneers of this approach Posner
(\protect\hyperlink{ref-posnerWhatJudgesJustices1993}{1993}) presents a
simple model of judicial utility as function mainly of income, leisure
and judicial voting. Further research followed the Posner mode and
presented alternative models of judicial utility (based on economic
psychology Foxall
(\protect\hyperlink{ref-foxallWhatJudgesMaximize2004}{2004})). Replacing
the policy oriented approaches, which hold judges to pursue political
policy oriented goals, researchers now focus more on their self-interest
in terms of career progression, higher income, or lesser workload
(\protect\hyperlink{ref-carrubbaWhoControlsContent2012}{Carrubba et al.
2012};
\protect\hyperlink{ref-epsteinStrategicRevolutionJudicial2000}{Epstein
and Knight 2000}). Posner
(\protect\hyperlink{ref-posnerHowJudgesThink2010}{2010}) presents nine
theories of approach for judicial behaviour, from which we mostly draw
on the economical and sociological theory. Economic theory of judicial
behaviour treats the judges as a rational, self-interested, utility
maximizer and sociological theory of judicial behaviour incorporates
factors of strategic calculation, emotion, and group polarization.

Epstein, Landes, and Posner
(\protect\hyperlink{ref-epsteinWhyWhenJudges2011}{2011}) based their
theory of dissents on the strategic-economic framework of
self-interested strategically motivated judges. They presume that judges
``leisure preferences, or, equivalently, effort aversion, which they
trade off against their desire to have a good reputation and to express
their legal and policy beliefs and preferences (and by doing so perhaps
influence law and policy) by their vote, and by the judicial opinion
explaining their vote, in the cases they hear.'' The benefits of a
dissenting opinion are the potential to undermine the majority opinion
when the dissent is influential and the enhanced reputation that the
judge enjoys. The dissenting opinion may be cited in the future by other
judges or publicly analysed by legal scholars.

The theories they presume and hypotheses they test rest on this
framework: in the policy-oriented framework, it would not make sense to
expect judges to dissent less as their workload increases. They would
still seek a way to advance their political agenda and research has
shown that dissenting opinions usually correspond to exactly just that
(\protect\hyperlink{ref-clarkLocatingSupremeCourt2010}{Clark and
Lauderdale 2010}). However, in the strategic account, the higher the
workload of a judge, the more pressing the effort costs of a dissent.
Similarly, if a dissenting opinion imputes costs on the majority, we can
theoretically expect it to respond to the dissent with a more thorough
or detailed argumentation in the majority opinion.

Wittig summarises the potential motivations for judges to attach a
separate opinion and, thus, to acquire additional costs: (1) potential
of impacting future caselaw, (2) moral obligation to distance oneself
from a decision that contradicts her values, (3) to convey certain image
about oneself.

These motivations also largely rely on the self-perceived stance towards
separate opinions in general.

The proponents of separate opinions view dissenting positively based on
the separate opinions being able to enrich the legal debate, being a
sign of judicial independence, increasing the legitimacy of any given
decision for it makes the decision more accurate of the real discussion
behind it.

The opponents mainly argue that showing the inability to speak in one
voice undermines a court's legitimacy or the reputation of the
dissenting judge. Moreover, judges seeking the appreciation from the
general public or legal community may act in their personal interests
instead of in the court's interests. Lastly, separate opinions come at
collegiality costs and may harm the mutual relationships of judges.

Wittig comes up with a model of separate opinions much better suited for
the civil law context of the CCC.

``Another part of the literature claims that the time judges have been
in office plays a crucial role in their behavior at the court. This
often called freshmen or acclimation effect draws on the argument that
new judges undergo a period of adjustment until they get used to the
workload and the procedures at the court. Brenner and Hagle describe it
as follows: ``The essence of an acclimation effect is that justices'
normal behavior patterns are temporarily disrupted while adjusting to
the Court's procedures and the workload'' (1996, 239). Hence, in their
earlier years at the court the judges are expected to write less
separate opinions than later in their term in office (Lanier 2011; Boyea
2010; Hurwitz and Stefko 2004; Hettinger, Lindquist, and Martinek 2003;
Brenner and Hagle 1996; Hagle 1993).''

\hypertarget{the-model}{%
\subsection{The model}\label{the-model}}

Wittig introduces a non-formal model of separate opinions, the
identification-disagreement model. We build on the
identification-disagreement model theoretically and we use it to
generate hypotheses for the CCC. Wittig amalgamates all the previously
introduced potential motivations of judges for writing separate opinions
into one cohesive and comprehensive model.

The model is made up of two dimensions. The first dimensions of the
model covers the disagreement level. The second dimension concerns the
judges' stance and degree of self-identification of their role as a
judge, Wittig terms this as a \emph{norm of consensus}. Separate
opinions are then ``a function of a judge's identification with the norm
of consensus and the level of disagreement of judges
(\protect\hyperlink{ref-wittigOccurrenceSeparateOpinions2016}{Wittig
2016, 74--75}).

\hypertarget{the-norm-of-consensus}{%
\subsection{The norm of consensus}\label{the-norm-of-consensus}}

Calderia and Zorn
(\protect\hyperlink{ref-calderiaTimeConsensualNorms1998}{1998}),
p.~876-877 define a norm as ``a long-run equilibrium outcome, which
underpins the interaction between individuals and reflects common
understandings as to what is acceptable behavior in given
circumstances.'' The norm of consensus in turn defines the level of
dissent that is acceptable at any given court
(\protect\hyperlink{ref-narayanConsensualNormHigh2005}{Narayan and Smyth
2005};
\protect\hyperlink{ref-wittigOccurrenceSeparateOpinions2016}{Wittig
2016, 75}.).

The argument of Wittig is two-fold. First, in civil law traditions, the
prevailing notion of the norm of consensus is that a court should not
display disagreement. Second, the extent of adherence to the norm varies
among judges,\footnote{This is corroborated by interviews we conducted
  with the third term CCC justices.} depending on how they weight the
costs and benefits they receive from following it
(\protect\hyperlink{ref-wittigOccurrenceSeparateOpinions2016}{Wittig
2016, 75}.).

A disonance between a proposed outcome for a case and any given judge's
preferences are eventually bound to happen. In such a case, the judge
can either express their sincere preferences by writing a separate
opinion or they can adapt their behavior according to the norm of
consensus and suppress the expression of her preferences. The second
route has also been termed \emph{dissent aversion}
(\protect\hyperlink{ref-epsteinWhyWhenJudges2011}{Epstein, Landes, and
Posner 2011})

Wittig draws up three types of utility that dictate various levels of
the adherence to the norm of consensus. Firstly, the intrinsic utility
is maximazed whenever a judge behaves in accordance with their true
values and opinions, setting aside their strategic or political
considerations. Secondly, expressive utility is harnessed when one
displays individuality and counters the notion of conformism. Thirdly,
the reputational utility arises when one adjusts their publicly
displayed preferences to the expectations of others. Wittig argues that
maximizing the former two forms of utilities in a situation of
disagreement leads to separate opinions, whereas maximazing the
reputational utility in such a situation gives way to the norm of
consensus, as the judge would otherwise jeopardize the court's
legitimacy as well as their reputation for not adhering to commonly
accepted norms
(\protect\hyperlink{ref-wittigOccurrenceSeparateOpinions2016}{Wittig
2016, 76}).\footnote{To some extent, even the third utility may lead to
  dissenting opinions insofar the individual reputation of a judge can
  in any way be linked to their non-conformity with the majority. An
  example that springs into mind is the late Justice Scalia, whose
  individual reputation among conservative circles would've been likely
  more jeopardized by siding with the liberal majority rather than with
  not adhering to the norm of consensus. SEHNAT ZDROJ}. The decision to
dissent or not to dissent then is a result of an weighting between costs
and benefits of these three types of utilities a judge derives from
either dissenting or not dissenting. The utilities in turn run across
two dimensions, disagreement across judges and adherence to the norm of
consensus.

\hypertarget{disagreement-on-the-bench}{%
\subsection{Disagreement on the bench}\label{disagreement-on-the-bench}}

A disagreement on a bench arises when the opinions on the matter diverge
during a discussion and a judge has a reason to object the majority
view. The sources of disagreement are manifold. A major source is that
of judge's individuality, each judge has varying preferences regarding
the legal rules, dispositions of cases or simply their moral values. On
top of that, case characteristics play an important role. Cases with
more value-laden or controversial topics may give raise to more
disagreement, similarly highly complex cases leave more space for
disagreement.

Within the CCC, we can observe a special example of circumstances giving
rise to higher level of disagreement. Lastly, the research on judicial
coalitions at the CCC has revealed that the third period of CCC between
2013-2023 is rather polarized and that there are two big coalitions of
judges that clash against each other
(\protect\hyperlink{ref-chmelCoOvlivnujeUstavni2021}{Chmel 2021};
\protect\hyperlink{ref-smekalMimopravniVlivyNa2021}{Smekal et al. 2021};
\protect\hyperlink{ref-vartazaryanSitOvaAnalyza2022}{Vartazaryan
2022}).\footnote{The Smekal et al.~book goes so far to coin the first
  coalition as a more left-leaning and the second as a more
  right-leaning, whereas we are not convinced by this label.} The
articles rely primarily on network analysis of the dissenting opinions
in the plenary proceedings and make inferential conclusions based on a
rather superficial descriptive analysis. We hypothesize that should the
relationship from the plenary sessions indeed exist, they should also
carry over to the 3-member panel hearings. Our research question is
whether having a 3-member panel composed of justices from both
coalitions creates a fertile ground for more disagreement. If this shows
to be true, it would provide further evidence to the two coalition
theory of the CCC
(\protect\hyperlink{ref-chmelCoOvlivnujeUstavni2021}{Chmel 2021};
\protect\hyperlink{ref-vartazaryanSitOvaAnalyza2022}{Vartazaryan 2022};
\protect\hyperlink{ref-smekalMimopravniVlivyNa2021}{Smekal et al. 2021})
as well as for the Wittig's identification-disagreement model.

We test whether the presumable existence of the coalitions carry over to
and have any effect on the dissenting behavior of judges in the panels.
Consistent with our theoretical part, we believe that such an situation
is theoretically a special case of circumstances with higher level of
disagreement. Our intuition suggests that if indeed there are two
coalitions in the plenary proceedings, which strongly disagree between
each other, such a disagreement should carry over to the panel level.

\hypertarget{a-brief-primer-on-the-ccc}{%
\section{A brief primer on the CCC}\label{a-brief-primer-on-the-ccc}}

The CCC consists of fifteen justices, out of which one is the president
of the CCC, two are vice presidents and twelve associate justices
(following the terminology of
\protect\hyperlink{ref-kosarConstitutionalCourtCzechia2020}{Kosař and
Vyhnánek 2020}). These fifteen justices are appointed by the president
of the Czech republic upon approval of the Senate. The justices enjoy 10
years terms with the possibility of re-election; there is no limit on
the times a justice can be re-elected. The three CCC functionaries are
unilaterally appointed by the Czech president.

Regarding the competences, the CCC is a typical Kelsenian court inspired
mainly by the German Federal Constitutional Court. The CCC enjoys the
power of abstract constitutional review, including constitutional
amendments. The abstract review procedure is initiated by political
actors (for example MPs) and usually concerns political issues.
Moreover, an ordinary court can initiate a concrete review procedure, if
that court reaches the conclusion that a legal norm upon which its
decision depends is not compatible with the constitution. Individuals
can also lodge constitutional complaints before the CCC. Lastly, the CCC
can also resolve separation-of-powers disputes, it can \emph{ex ante}
review international treaties, decide on impeachment of the president of
the republic, and it has additional ancillary powers (for a complete
overview, see
\protect\hyperlink{ref-kosarConstitutionalCourtCzechia2020}{Kosař and
Vyhnánek 2020}).

The CCC is an example of a collegial court. From the perspective of the
inner, the CCC can decide in four bodies: (1) individual justices, (2)
3-member chambers (\emph{senáty}), (3) the plenum (\emph{plénum}), and
(4) special disciplinary chambers. However, the 3-member chambers and
the plenum play a crucial role. The plenum is composed of all justices,
whereas the four 3-member chambers are composed of the associate
justices. Neither the president of the CCC or her vice-presidents are
permanents members of the 3-member chambers. Until 2016, the composition
of the chambers was static. However, in 2016, a system of regular
2-yearly rotations was introduced, wherein the president of the chamber
rotates to a different every 2 years. I am of the view that such a
institutional change opens up potential for quasi-experimental research
similar to the Gschwend, Sternberg, and Zittlau
(\protect\hyperlink{ref-gschwendAreJudgesPolitical2016}{2016}) study
utilizing judge absences within the 3-member panels of the German
federal constitutional court. In general, the plenum is responsible for
the abstract review, whereas the 3-member chambers are responsible for
the individual constitutional complaints.

In the chamber proceedings, decisions on admissibility must be
unanimous, decisions on merits need not be, therefore, two votes are
necessary.\footnote{Which enables the attachment of separate opinions}
In the plenum, the general voting quorum is a simple majority and the
plenum is quorate when there are ten justices present. The abstract
review is one of the exceptions that sets the quorum higher, more
specifically to 9 votes.

A judge rapporteur plays a crucial role
(\protect\hyperlink{ref-chmelZpravodajoveSenatyVliv2017}{Chmel 2017};
\protect\hyperlink{ref-horenovskyProcessMakingConstitutional2015}{Hořeňovský
and Chmel 2015} study the large influence of the judge rapporteurs at
the CCC). Each case of the CCC gets assigned to a judge raporteur. The
assignement is regulated by a case allocation plan.\footnote{the
  original term is \emph{rozvrh práce}, which is usually translated as a
  \emph{work schedule}, however, I borrow the term \emph{case allocation
  plan} from Hamann
  (\protect\hyperlink{ref-hamannGermanFederalCourts2019}{2019}), p.~673}
She is tasked with drafting the opinion, about which the body then
votes. The president of the CCC (in plenary cases) or the president of
the chamber (in chamber cases) may re-assign a case to a different judge
rapporteur if the draft opinion by the original judge rapporteur did not
receive a majority of votes. Unfortunately, the CCC does not keep track
of these reassignments.\footnote{I unsuccessfully attempted to retrieve
  the information with the right to information}

The CCC allows for separate opinions. They can take two forms:
dissenting or concurring opinions. Each justice has the right to author
a separate opinion, which then gets published with the CCC decision. It
follows that not every anti-majority vote implies a separate opinion, it
is up to the justices to decide whether they want to attach a separate
opinion with their vote.

The room for the dissenting judge and the majority to address each other
differs between the two bodies. Based on our internal insight, there is
less back and forth interplay between the judges, more akin to the
SCOTUS context, and most of the communication is handled remotely in the
panel proceedings, whereas the plenum meets regularly to discuss the
cases in person. Despite that, procedurally speaking, the process of
generating separate opinions is the same. In both cases, the rapporteurs
are informed about the outcome of the vote, which is filed in the voting
record. The separate opinion is then sent to the judge rapporteur before
the decision is announced, as it cannot be added until after the
announcement. It is important to note that judges have the possibility,
not the obligation, to dissent. In other words, there is room for judges
to give way to strategic considerations.

It may be concluded that the CCC takes after

\hypertarget{hypotheses}{%
\section{Hypotheses}\label{hypotheses}}

Following the identification-disagreement model, the likelihood of a
separate opinion depends on judges' adherence to the norm of consensus
and the level of disagreement. Therofore, the first two hypotheses are
as follows:

\textbf{H\textsubscript{1}:} \emph{The probability of observing a
separate opinion is higher for judges with low norm-identification than
for judges with high norm-identification.}

\textbf{H\textsubscript{2}:} \emph{The probability of observing a
separate opinion is higher for cases with a higher level of disagreement
than in cases with a lower level of disagreement.}

According to Epstein, Landes, and Posner
(\protect\hyperlink{ref-epsteinWhyWhenJudges2011}{2011}): ``{[}t{]}he
economic theory of judicial behavior predicts that a decline in the
judicial workload would lower the opportunity cost of dissenting and
increase the frequency of separate opinions, and also that the greater
the ideological heterogeneity among judges the more likely they are to
disagree and so the higher the dissent rate will be.'' The authors find
a positive relationship between the dissent rate, i.e., number of
dissents divided by the number of cases, and caseload. Using the
language of the identification-disagreement model, we believe leisure to
be an example of the individual utility a judge may consider. We believe
individual utility may also pull the other way: against a separate
opinion. Therefore, our hypothesis 3 suggests:

\textbf{H\textsubscript{3}:} \emph{The higher the workload of a judge,
the lower their likelihood of dissent.}

Epstein, Landes, and Posner
(\protect\hyperlink{ref-epsteinWhyWhenJudges2011}{2011}) address the
issue of collegiality costs arising for a dissenting judge: ``The effort
involved in these revisions, and the resentment at criticism by the
dissenting judge, may impose a collegiality cost on the dissenting judge
by making it more difficult for him to persuade judges to join his
majority opinions in future cases.'' Based on this theory, they predict
and indeed empirically confirm that ``dissents will be less frequent in
circuits that have fewer judges because any two of its judges will sit
together more frequently and thus have a greater incentive to invest in
collegiality.'' Put simply, the researchers compare the dissent rates
between courts with differing number of members.

While it is hard for us to see how a variation between the number of
members in the plenary session and 3-member panels could be isolated
from a plethora of potential confounding variables, we are able to make
use of the limited term of CCC judges. We test whether judges that are
at the start of their term, and thus are aware that they will ``sit
together more frequently'' invest in collegiality by averting separate
opinions and whether when their term draws to an end, they give way to
their disagreement. This presumes that the outlook of sharing the 10
year term with your colleagues at the beginning of judges' terms
increases the collegiality costs of dissenting, whereas at the end of
their terms, the collegiality costs decrease with the end of the shared
term looming on the horizon.

Moreover, Wittig argues that the adherence to the norm of consensus
varies across professions the justices enter into after their term: the
closer they are to the end of their terms, the wider the gap between the
professions. Justices that stay in the judiciary or go into scholarship
are theoretically expected to adhere to the norm of consensus stronger,
especially at the end of their terms. For reasons discussed bellow, such
an approach does not fit well the CCC as its justices are rather old
when they leave their office. We replace that with the profession that
the justices held when they entered the office.

We pose the following research question: does the judges' likelihood of
separate opinions across their terms as a result of differing
collegiality costs and as a result of their professional history. We
test the following hypotheses:

\textbf{H\textsubscript{4}:} \emph{The closer the date of the decision
to the date at which the judge entered the office, the lower likelihood
of a separate opinion, whereas the closer the date of the decision to
the date at which the judge left the office, the higher the likelihood
of a separate opinion.}

\textbf{H\textsubscript{5}:} \emph{The closer the date of the decision
to the date at which the judge will enter or leave the office, the
larger the difference between the professions.}

Lastly, we test whether the 3-member chambers with members from both
judicial coalitions formed at the plenum make up a special case of
circumstances of higher level of disagreement. Our research question is
thus whether judicial coalitions formed in the plenary proceedings
affect the amount of disagreement and, in turn, the likelihood of
dissent of a judge in 3-member panels. Our hypothesis is as follows:

\textbf{H\textsubscript{6}:} \emph{Having a 3-member panel composed of
members of both judicial coalitions increases judges' likelihood of a
dissent.}

\hypertarget{empirical-analysis}{%
\section{Empirical analysis}\label{empirical-analysis}}

Wittig's model:

\begin{itemize}
\tightlist
\item
  Dependent variable

  \begin{itemize}
  \tightlist
  \item
    A Judge's Vote - done
  \end{itemize}
\item
  Explanatory variables

  \begin{itemize}
  \tightlist
  \item
    Norm identification

    \begin{itemize}
    \tightlist
    \item
      a profession the judge has pursued after their time in the office
      - done as before
    \end{itemize}
  \item
    time in office - months until judges exit from the office - done
  \item
    potential for disagreement

    \begin{itemize}
    \tightlist
    \item
      paragraphs - number of paragraphs in the facts of the
      case/procedure history part - complexity in the form of number of
      references to other law/CCC caselaw
    \item
      controversial topics - done
    \end{itemize}
  \end{itemize}
\item
  Control variables

  \begin{itemize}
  \tightlist
  \item
    institutional rules

    \begin{itemize}
    \tightlist
    \item
      admissibility - whether a case was admissible or not - can be
      added
    \item
      number of parties - cannot be added
    \item
      political procedure - if it's initiated by a political actor
    \item
      president of the senate / judge rapportuer - don't understand
    \end{itemize}
  \item
    role of law

    \begin{itemize}
    \tightlist
    \item
      number of references to scholarly literature per paragraph
    \end{itemize}
  \end{itemize}
\end{itemize}

\hypertarget{data-description}{%
\subsection{Data description}\label{data-description}}

The data is based on the CCC dataset, which contains all decisions
published by the CCC since its foundation, complete text corpus as well
as plenty of metadata. We narrow our cases to all plenum decisions and
to all 3-member chamber decisions on merits up until the end of 2022.
The admissibility decisions of the 3-member chambers must be made
unanimously, concurring decisions therein are a rarity.\footnote{On top
  of that out of the 39 separate opinions in admissibility 3-chamber
  decisions, 25 of that are a copypasta from justice Jan Filip and 6 are
  a copypasta from justice Josef Fiala in alike cases. Thus, there is
  only a few left for a meaningful analysis. The class imbalance of the
  remaining \textasciitilde10 decisions would be too large against the
  XXX of all chamber decisions on admissibility}. We skip the first
decade of the CCC as the data on it are rather incomplete and
inconsistent: For example, very few decisions contain the information on
the composition in the text and many do not contain the name of the
dissenting judge. We also limit our analysis until the end of 2022 as
the CCC entered its 4th term in 2023 and started to undergo a complete
personal change.

The final dataset for analysis contains 4669 decisions of the CCC. Out
of them, 81 \% are decisions by the 3-member chambers on merits, around
9 \% are plenum decisions on merits and the remaining 9 \% are plenum
decisions on admissibility. At least one separate opinion contain 4 \%
decisions out of the 3-member decisions, 11 \% decisions of the plenum
decisions on admissibility, and 39 \% decisions of the plenum decisions
on merits.

\begin{table}

\caption{\label{tab:unnamed-chunk-1}Summary statistics for the whole dataset}
\centering
\begin{tabular}[t]{l|l|r|l|l}
\hline
\textbf{Formation} & \textbf{Admissibility} & \textbf{Count} & \textbf{Ratio - Total} & \textbf{Dissents - Ratio}\\
\hline
Chamber & merits & 3862 & 81.2\% & 4.3\%\\
\hline
Plenum & admissibility & 456 & 9.6\% & 11.2\%\\
\hline
Plenum & merits & 436 & 9.2\% & 39.4\%\\
\hline
\end{tabular}
\end{table}

\hypertarget{operationalization}{%
\subsection{Operationalization}\label{operationalization}}

\hypertarget{dependent-variable-a-separate-opinion}{%
\subsubsection{Dependent variable: a separate
opinion}\label{dependent-variable-a-separate-opinion}}

Our dependent variable is whether a justice attached a separate opinion
to a decision or not.\footnote{Unlike Wittig we do not call our
  dependent variable a judge's vote, as that refers to a slightly
  different thing within the CCC context. A judge may vote against the
  majority opinion but since they are not mandated to write a separate
  opinion, these do not necessarily overlap. Similarly, a judge may vote
  for a disposition of a case and still attach a concurrence separate
  opinion.} Our dependent variable is thus a binomial variable that has
two categories: either a justice did attach a separate opinion or she
did not in any given case.

We do not distinguish between a concurrence and a dissent for one
reason. The difference between the two lays only in the disposition of
the case. The justices may equally disagree on the interpretation of
legal rules, thus, in the case-space model terms
(\protect\hyperlink{ref-landaDisagreementsCollegialCourts2007}{Landa and
Lax 2007--2008}; \protect\hyperlink{ref-laxNewJudicialPolitics2011}{Lax
2011}), the judge cut points in any given case differ even when a
justice attaches ``only'' concurrence. The difference is that in the
cases containing concurrence, the case facts may have completely
accidentally fallen on the same side both of the concurring judge as
well as the majority, whereas in the cases containing a dissenting
opinion, the case fell in between the cut points. We do not consider
this phenomenon as theoretically interesting.

\hypertarget{explanatory-variables}{%
\subsubsection{Explanatory variables}\label{explanatory-variables}}

\hypertarget{disagreement-potential}{%
\paragraph{Disagreement potential}\label{disagreement-potential}}

From the theoretical perspective it may be expected that the potential
for disagreement varies across cases. In some cases, the disagreement
potential is higher and, thus, the likelihood of a separate opinion is
higher than in the cases with lower potential for disagreement. More
specifically, we expect the potential for disagreement to be captured by
two characteristics of any given case: (1) its complexity and (2) its
controversy.

As Corley, Steigerwalt, and Ward
(\protect\hyperlink{ref-corleyPuzzleUnanimityConsensus2013}{2013}), p70
argue and empirically measure, complexity of a case leads to less
certainty and more ambiguity for the justices, which leads to a higher
likelihood of disagreement. The authors define legal complexity as the
number of legal issues a case has to address. True to the Corley,
Steigerwalt, and Ward
(\protect\hyperlink{ref-corleyPuzzleUnanimityConsensus2013}{2013})
study, our operationalization of complexity relies on the assumption
that the more legal issues there are in any given case, the higher the
number of references to other laws and caselaw in the text of the
corresponding decision. The variable \emph{concerned acts} captures the
number of concerned ordinary legal acts on the legal-act level, the
variable \emph{concerned constitutional acts} captures the number of
articles of the constitutional legal acts (mostly the Czech Constitution
and the Charter of Fundamental Rights and Freedoms) and variable
\emph{caselaw} captures the number of citations to its own caselaw. The
information on the first two variables is based on the metadata provided
by the CCC, the last is based on the regular expressions search of the
text of the decisions.

The first two we believe to be sufficiently different from each other: a
typical right to fair proceedings case may concern only one
constitutional article (the article 36 of the Charter) but many legal
acts, whereas a typical separation of powers case concerning the
Parliament concerns many consitutional articles but only few ordinary
laws. On the other hand, the number of (concerned) constitutional acts
and references to the CCC caselaw may be correlated. We therefore run
diagnostics, which reveal that the citations to CCC caselaw and the
number of references to constitutional acts are rather correlated.

\begin{table}

\caption{\label{tab:unnamed-chunk-2}Correlations between Independent Variables of Case Complexity}
\centering
\begin{tabular}[t]{l|l|l|l}
\hline
\textbf{Variable} & \textbf{1} & \textbf{2} & \textbf{3}\\
\hline
1. \# of Ordinary Acts & — & — & —\\
\hline
2. \# of Constitutional Articles & .35 & — & —\\
\hline
3. \# of CCC References & .32 & .48 & —\\
\hline
\multicolumn{4}{l}{\rule{0pt}{1em}\textit{Note.} Correlation was calculated using the Spearman Correlation Coefficient.}\\
\end{tabular}
\end{table}

In a similar vein, certain typically value-laden topics may generate
more disagreement even if they raise only one or few legal questions.
Typically, the restitution cases or cases concerning fundamental human
rights have been coined as rather controversial. Wittig identifies a
number of topics in the FCC database as prone for controversy. The CCC
dataset contains a variable \emph{subject\_proceedings}, which roughly
corresponds to the FCC topics data.

\hypertarget{norm-identification}{%
\paragraph{Norm-identification}\label{norm-identification}}

Wittig, we believe rightfully, draws a relationship between the
adherence to norm of consensus and one's career choices. Theoretically,
the actors socialized within the judiciary and its values are more
likely to adhere to them.

Wittig thus operationalizes the norm-identification as the justices'
career choice after their term. The justices that chose to stay within
judiciary or go back to being scholars are expected to more strongly
identify with the norm of consensus, whereas the justices' that made
different career choice are less likely to identify with the norm of
consensus.

Unfortunately, such a measure does not fit the CCC context. As the data
shows, CCC justices start their term usually at the end of their career
and a considerable part of them ends their term in their 70s, well past
the retirement age in Czechia. Therefore, instead of operationalizing
the norm-identification as the profession after their term, we
operationalize it as the profession before they entered the CCC.

\begin{figure}
\centering
\includegraphics{separate_opinions_files/figure-latex/unnamed-chunk-3-1.pdf}
\caption{Kernel density of justices' ages at the start and and the end
of their terms}
\end{figure}

The variable \emph{profession} contains the information on the justices'
previous career choices. We can observe that the distribution of the
professions has changed across time. While the 1st and 2nd terms of the
CCC were quite balanced in terms of the professions, its 3rd term is
heavily skewed towards the more to the norm of consensus adherent
professions.

\begin{figure}
\centering
\includegraphics{separate_opinions_files/figure-latex/unnamed-chunk-4-1.pdf}
\caption{The distribution of professions of the CCC justices across its
3 terms.}
\end{figure}

\hypertarget{time-in-office}{%
\paragraph{Time in office}\label{time-in-office}}

We code the time in office variables as the number of days a justice has
left until their end of mandate. That allows us to account both for the
collegiality costs hypothesis as well as the difference between
professions hypothesis. We also operationalize the time in the office
variable as a number of months between the month of the decision and the
month at which the justice is expected to leave their office. The
variable ranges from XXX to XXX.

\hypertarget{workload}{%
\paragraph{Workload}\label{workload}}

We operationalize workload as the number of unfinished cases that any
given judge has in the moment of any given decision as a judge
rapporteur. We firstly mined the compositions of panels as well as the
plenary from the text of the decision. We then calculated the number of
unfinished cases each judge had at the time of any given decision as a
judge rapporteur using the date of submission and of decision of a case.
We believe such a measure captures the perceived workload of a judge
better than the original EPL measure of caseload of the whole court: a
judge knowing that they have, for example, 20 in comparison to 100
decisions to draft as a judge rapporteur is how they would perceive
having had more workload.

\hypertarget{mixed-coalition}{%
\paragraph{Mixed coalition}\label{mixed-coalition}}

Lastly, we include a dummy variable mixed coalitions, which takes up the
value 1 when two conditions are met: (1) the decision was made in a
3-member chamber and (2) the composition of the chamber was made up of
judges from both coalitions.

\hypertarget{identification-strategy}{%
\subsection{Identification strategy}\label{identification-strategy}}

\hypertarget{results}{%
\subsection{Results}\label{results}}

The results reveal quite an interesting trend at the CCC.

\% Table created by stargazer v.5.2.3 by Marek Hlavac, Social Policy
Institute. E-mail: marek.hlavac at gmail.com \% Date and time: Fri, Jan
19, 2024 - 21:57:42

\begin{table}[!htbp] \centering 
  \caption{Results from the Logit Model} 
  \label{} 
\begin{tabular}{@{\extracolsep{5pt}}lc} 
\\[-1.8ex]\hline 
\hline \\[-1.8ex] 
 & \multicolumn{1}{c}{\textit{Dependent variable:}} \\ 
\cline{2-2} 
\\[-1.8ex] & dissenting\_opinion \\ 
\hline \\[-1.8ex] 
 n\_concerned\_acts & 0.054$^{***}$ \\ 
  & (0.010) \\ 
  & \\ 
 n\_concerned\_constitutional\_acts & 0.111$^{***}$ \\ 
  & (0.010) \\ 
  & \\ 
 n\_citations & 0.032$^{***}$ \\ 
  & (0.003) \\ 
  & \\ 
 merits\_admissibilitymerits & $-$0.228$^{**}$ \\ 
  & (0.105) \\ 
  & \\ 
 judge\_professionlawyer & $-$0.349 \\ 
  & (0.259) \\ 
  & \\ 
 judge\_professionpolitician & 0.551$^{**}$ \\ 
  & (0.270) \\ 
  & \\ 
 judge\_professionscholar & 0.253$^{*}$ \\ 
  & (0.152) \\ 
  & \\ 
 time\_in\_office & $-$0.002 \\ 
  & (0.002) \\ 
  & \\ 
 controversial & 0.636$^{***}$ \\ 
  & (0.114) \\ 
  & \\ 
 judge\_professionlawyer:time\_in\_office & 0.003 \\ 
  & (0.004) \\ 
  & \\ 
 judge\_professionpolitician:time\_in\_office & $-$0.014$^{***}$ \\ 
  & (0.004) \\ 
  & \\ 
 judge\_professionscholar:time\_in\_office & 0.001 \\ 
  & (0.002) \\ 
  & \\ 
 Constant & $-$3.997$^{***}$ \\ 
  & (0.146) \\ 
  & \\ 
\hline \\[-1.8ex] 
Observations & 21,741 \\ 
Log Likelihood & $-$3,328.673 \\ 
Akaike Inf. Crit. & 6,683.347 \\ 
\hline 
\hline \\[-1.8ex] 
\textit{Note:}  & \multicolumn{1}{r}{$^{*}$p$<$0.1; $^{**}$p$<$0.05; $^{***}$p$<$0.01} \\ 
\end{tabular} 
\end{table}

\hypertarget{conclusions}{%
\section{Conclusions}\label{conclusions}}

\vspace{30pt}

\hypertarget{literature}{%
\section*{Literature}\label{literature}}
\addcontentsline{toc}{section}{Literature}

\hypertarget{refs}{}
\begin{CSLReferences}{1}{0}
\leavevmode\vadjust pre{\hypertarget{ref-arnoldScalingCourtDecisions2023}{}}%
Arnold, Christian, Benjamin G. Engst, and Thomas Gschwend. 2023.
{``Scaling {Court Decisions} with {Citation Networks}.''} \emph{Journal
of Law and Courts} 11 (1): 25--44. \url{https://doi.org/10.1086/717420}.

\leavevmode\vadjust pre{\hypertarget{ref-berdejoElectoralCyclesUS2017}{}}%
Berdejó, Carlos, and Daniel L. Chen. 2017. {``Electoral {Cycles} Among
{US Courts} of {Appeals Judges}.''} \emph{The Journal of Law and
Economics} 60 (3): 479--96. \url{https://doi.org/10.1086/696237}.

\leavevmode\vadjust pre{\hypertarget{ref-bielenBacklogsLitigationRates2018}{}}%
Bielen, Samantha, Ludo Peeters, Wim Marneffe, and Lode Vereeck. 2018.
{``Backlogs and Litigation Rates: {Testing} Congestion Equilibrium
Across {European} Judiciaries.''} \emph{International Review of Law and
Economics} 53 (March): 9--22.
\url{https://doi.org/10.1016/j.irle.2017.09.002}.

\leavevmode\vadjust pre{\hypertarget{ref-boydUntanglingCausalEffects2010}{}}%
Boyd, Christina L., Lee Epstein, and Andrew D. Martin. 2010.
{``Untangling the {Causal Effects} of {Sex} on {Judging}.''}
\emph{American Journal of Political Science} 54 (2): 389--411.
\url{https://www.jstor.org/stable/25652213}.

\leavevmode\vadjust pre{\hypertarget{ref-brekkeThatOrderHow2023}{}}%
Brekke, Stein Arne, Daniel Naurin, Urška Šadl, and Lucía López-Zurita.
2023. {``That's an {Order}! {How} the {Quest} for {Efficiency Is
Transforming Judicial Cooperation} in {Europe}.''} \emph{JCMS: Journal
of Common Market Studies} 61 (1): 58--75.
\url{https://doi.org/10.1111/jcms.13346}.

\leavevmode\vadjust pre{\hypertarget{ref-calderiaTimeConsensualNorms1998}{}}%
Calderia, Gregory A., and Christopher J. W. Zorn. 1998. {``Of {Time} and
{Consensual Norms} in the {Supreme Court}.''} \emph{American Journal of
Political Science} 42 (3): 874--902.
\url{https://doi.org/10.2307/2991733}.

\leavevmode\vadjust pre{\hypertarget{ref-cameronChapterWhatJudges2017}{}}%
Cameron, Charles M., and Lewis A. Kornhauser. 2017. {``Chapter 3: {What
Do Judges Want}? {How} to {Model Judicial Preferences}.''} SSRN
Scholarly Paper. {Rochester, NY}. June 2, 2017.
\url{https://doi.org/10.2139/ssrn.2979419}.

\leavevmode\vadjust pre{\hypertarget{ref-carrubbaWhoControlsContent2012}{}}%
Carrubba, Cliff, Barry Friedman, Andrew D. Martin, and Georg Vanberg.
2012. {``Who {Controls} the {Content} of {Supreme Court Opinions}?''}
\emph{American Journal of Political Science} 56 (2): 400--412.
\url{https://doi.org/10.1111/j.1540-5907.2011.00557.x}.

\leavevmode\vadjust pre{\hypertarget{ref-chmelZpravodajoveSenatyVliv2017}{}}%
Chmel, Jan. 2017. {``Zpravodajové a Senáty: {Vliv} Složení Senátu Na
Rozhodování {Ústavního} Soudu {České} Republiky o Ústavních
Stížnostech.''} \emph{Časopis Pro Právní Vědu a Praxi} 25 (4): 739.
\url{https://doi.org/10.5817/CPVP2017-4-9}.

\leavevmode\vadjust pre{\hypertarget{ref-chmelCoOvlivnujeUstavni2021}{}}%
---------. 2021. \emph{Co Ovlivňuje {Ústavní} Soud a Jeho Soudce? /}.
Vydání první. Teoretik ({Leges}). {Leges,}.

\leavevmode\vadjust pre{\hypertarget{ref-clarkEstimatingEffectLeisure2018}{}}%
Clark, Tom S., Benjamin G. Engst, and Jeffrey K. Staton. 2018.
{``Estimating the {Effect} of {Leisure} on {Judicial Performance}.''}
\emph{The Journal of Legal Studies} 47 (2): 349--90.
\url{https://doi.org/10.1086/699150}.

\leavevmode\vadjust pre{\hypertarget{ref-clarkLocatingSupremeCourt2010}{}}%
Clark, Tom S., and Benjamin Lauderdale. 2010. {``Locating {Supreme Court
Opinions} in {Doctrine Space}.''} \emph{American Journal of Political
Science} 54 (4): 871--90.
\url{https://doi.org/10.1111/j.1540-5907.2010.00470.x}.

\leavevmode\vadjust pre{\hypertarget{ref-corleyPuzzleUnanimityConsensus2013}{}}%
Corley, Pamela C., Amy Steigerwalt, and Artemus Ward. 2013. \emph{The
{Puzzle} of {Unanimity}: {Consensus} on the {United States Supreme
Court}}. {Redwood City, UNITED STATES}: {Stanford University Press}.
\url{http://ebookcentral.proquest.com/lib/huberlin-ebooks/detail.action?docID=1180198}.

\leavevmode\vadjust pre{\hypertarget{ref-coupetteQuantitativeRechtswissenschaft2018}{}}%
Coupette, Corinna, and Andreas M. Fleckner. 2018. {``Quantitative
{Rechtswissenschaft}.''} \emph{JuristenZeitung (JZ)} 73 (8): 379--89.
\url{https://doi.org/10.1628/jz-2018-0020}.

\leavevmode\vadjust pre{\hypertarget{ref-dworkinPoliticalJudgesRule1980}{}}%
Dworkin, Ronald M. 1980. \emph{Political Judges and the Rule of Law}.
{London}: {British Academy}.

\leavevmode\vadjust pre{\hypertarget{ref-engstEinflussParteinaheAuf2017}{}}%
Engst, Benjamin G., Thomas Gschwend, Nils Schaks, Sebastian Sternberg,
and Caroline Wittig. 2017. {``Zum {Einfluss} Der {Parteinähe} Auf Das
{Abstimmungsverhalten} Der {Bundesverfassungsrichter} -- Eine
Quantitative {Untersuchung}.''} \emph{JuristenZeitung} 72 (17): 816--26.
\url{https://www.jstor.org/stable/44867374}.

\leavevmode\vadjust pre{\hypertarget{ref-epsteinChoicesJusticesMake1997}{}}%
Epstein, Lee, and Jack Knight. 1997. \emph{The {Choices Justices Make}}.
{SAGE}. \url{https://books.google.com?id=hSnom2h2_zUC}.

\leavevmode\vadjust pre{\hypertarget{ref-epsteinStrategicRevolutionJudicial2000}{}}%
---------. 2000. {``Toward a {Strategic Revolution} in {Judicial
Politics}: {A Look Back}, {A Look Ahead}.''} \emph{Political Research
Quarterly} 53 (3): 625--61.
\url{https://doi.org/10.1177/106591290005300309}.

\leavevmode\vadjust pre{\hypertarget{ref-epsteinWhyWhenJudges2011}{}}%
Epstein, Lee, William M. Landes, and Richard A. Posner. 2011. {``Why
({And When}) {Judges Dissent}: {A Theoretical And Empirical
Analysis}.''} \emph{Journal of Legal Analysis} 3 (1): 101--37.
\url{https://doi.org/10.1093/jla/3.1.101}.

\leavevmode\vadjust pre{\hypertarget{ref-fjelstulHowChamberSystem2023}{}}%
Fjelstul, Joshua. 2023. {``How the {Chamber System} at the {CJEU
Undermines} the {Consistency} of the {Court}'s {Application} of {EU
Law}.''} \emph{Journal of Law and Courts}, 717422.
\url{https://doi.org/10.1086/717422}.

\leavevmode\vadjust pre{\hypertarget{ref-fjelstulEvolutionEuropeanUnion2019}{}}%
Fjelstul, Joshua C. 2019. {``The Evolution of {European Union} Law: {A}
New Data Set on the {\emph{Acquis Communautaire}}.''} \emph{European
Union Politics} 20 (4): 670--91.
\url{https://doi.org/10.1177/1465116519842947}.

\leavevmode\vadjust pre{\hypertarget{ref-fjelstulTimelyAdministrationJustice2022}{}}%
Fjelstul, Joshua C., Matthew Gabel, and Clifford J. Carrubba. 2022.
{``The Timely Administration of Justice: Using Computational Simulations
to Evaluate Institutional Reforms at the {CJEU}.''} \emph{Journal of
European Public Policy}, August, 1--22.
\url{https://doi.org/10.1080/13501763.2022.2113115}.

\leavevmode\vadjust pre{\hypertarget{ref-foxallWhatJudgesMaximize2004}{}}%
Foxall, Gordon R. 2004. {``What Judges Maximize:toward an Economic
Psychology of the Judicial Utility Function.''} \emph{Liverpool Law
Review} 25 (3): 177--94.
\url{https://doi.org/10.1007/s10991-004-2877-9}.

\leavevmode\vadjust pre{\hypertarget{ref-gschwendAreJudgesPolitical2016}{}}%
Gschwend, Thomas, Sebastian Sternberg, and Steffen Zittlau. 2016. {``Are
{Judges Political Animals} After {All}? {Quasi-Experimental Evidence}
from the {German Federal Constitutional Court}.''} SSRN Scholarly Paper.
{Rochester, NY}. February 26, 2016.
\url{https://doi.org/10.2139/ssrn.2738512}.

\leavevmode\vadjust pre{\hypertarget{ref-hamannGermanFederalCourts2019}{}}%
Hamann, Hanjo. 2019. {``The {German Federal Courts Dataset} 1950--2019:
{From Paper Archives} to {Linked Open Data}.''} \emph{Journal of
Empirical Legal Studies} 16 (3): 671--88.
\url{https://doi.org/10.1111/jels.12230}.

\leavevmode\vadjust pre{\hypertarget{ref-hanrettyDissentIberiaIdeal2012}{}}%
Hanretty, Chris. 2012. {``Dissent in {Iberia}: {The} Ideal Points of
Justices on the {Spanish} and {Portuguese Constitutional Tribunals}.''}
\emph{European Journal of Political Research} 51 (5): 671--92.
\url{https://doi.org/10.1111/j.1475-6765.2012.02056.x}.

\leavevmode\vadjust pre{\hypertarget{ref-hanrettyCourtSpecialistsJudicial2020}{}}%
---------. 2020. \emph{A {Court} of {Specialists}: {Judicial Behavior}
on the {UK Supreme Court}}. {Oxford University Press}.
\url{https://doi.org/10.1093/oso/9780197509234.001.0001}.

\leavevmode\vadjust pre{\hypertarget{ref-horenovskyProcessMakingConstitutional2015}{}}%
Hořeňovský, Jan, and Jan Chmel. 2015. {``The Process of making the
Constitutional Court Judgements.''} \emph{Časopis pro právní vědu a
praxi} 23 (3): 302--11.
\url{https://www.ceeol.com/search/article-detail?id=780150}.

\leavevmode\vadjust pre{\hypertarget{ref-kastellecEmpiricallyEvaluatingCountermajoritarian2016}{}}%
Kastellec, Jonathan P. 2016. {``Empirically {Evaluating} the
{Countermajoritarian Difficulty}: {Public Opinion}, {State Policy}, and
{Judicial Review} Before {\emph{Roe}}{ \emph{v.} }{\emph{Wade}}.''}
\emph{Journal of Law and Courts} 4 (1): 1--42.
\url{https://doi.org/10.1086/683466}.

\leavevmode\vadjust pre{\hypertarget{ref-kornhauserModelingCollegialCourts1992a}{}}%
Kornhauser, Lewis A. 1992a. {``Modeling {Collegial Courts}. {II}. {Legal
Doctrine}.''} \emph{Journal of Law, Economics and Organization} 8: 441.
\url{https://heinonline.org/HOL/Page?handle=hein.journals/jleo8&id=449&div=&collection=}.

\leavevmode\vadjust pre{\hypertarget{ref-kornhauserModelingCollegialCourts1992}{}}%
---------. 1992b. {``Modeling Collegial Courts {I}:
{Path-dependence}.''} \emph{International Review of Law and Economics}
12 (2): 169--85. \url{https://doi.org/10.1016/0144-8188(92)90034-O}.

\leavevmode\vadjust pre{\hypertarget{ref-kosarConstitutionalCourtCzechia2020}{}}%
Kosař, David, and Ladislav Vyhnánek. 2020. {``The {Constitutional Court}
of {Czechia}.''} In \emph{The {Max Planck Handbooks} in {European Public
Law}: {Volume III}: {Constitutional Adjudication}: {Institutions}},
edited by Armin von Bogdandy, Peter Huber, and Christoph Grabenwarter,
0. {Oxford University Press}.
\url{https://doi.org/10.1093/oso/9780198726418.003.0004}.

\leavevmode\vadjust pre{\hypertarget{ref-landaDisagreementsCollegialCourts2007}{}}%
Landa, Dimitri, and Jeffrey R. Lax. 2007--2008. {``Disagreements on
{Collegial Courts}: {A Case-Space Approach}.''} \emph{University of
Pennsylvania Journal of Constitutional Law} 10: 305.
\url{https://heinonline.org/HOL/Page?handle=hein.journals/upjcl10&id=315&div=&collection=}.

\leavevmode\vadjust pre{\hypertarget{ref-lauderdaleScalingPoliticallyMeaningful2014}{}}%
Lauderdale, Benjamin E., and Tom S. Clark. 2014. {``Scaling {Politically
Meaningful Dimensions Using Texts} and {Votes}: {SCALING POLITICALLY
MEANINGFUL DIMENSIONS}.''} \emph{American Journal of Political Science}
58 (3): 754--71. \url{https://doi.org/10.1111/ajps.12085}.

\leavevmode\vadjust pre{\hypertarget{ref-laxNewJudicialPolitics2011}{}}%
Lax, Jeffrey R. 2011. {``The {New Judicial Politics} of {Legal
Doctrine}.''} \emph{Annual Review of Political Science} 14 (1): 131--57.
\url{https://doi.org/10.1146/annurev.polisci.042108.134842}.

\leavevmode\vadjust pre{\hypertarget{ref-moyerJudicialInnovationSexual2012}{}}%
Moyer, Laura P., and Holley Tankersley. 2012. {``Judicial {Innovation}
and {Sexual Harassment Doctrine} in the {U}.{S}. {Courts} of
{Appeals}.''} \emph{Political Research Quarterly} 65 (4): 784--98.
\url{https://doi.org/10.1177/1065912911411097}.

\leavevmode\vadjust pre{\hypertarget{ref-narayanConsensualNormHigh2005}{}}%
Narayan, Paresh Kumar, and Russell Smyth. 2005. {``The {Consensual Norm}
on the {High Court} of {Australia}: 1904-2001.''} \emph{International
Political Science Review} 26 (2): 147--68.
\url{https://doi.org/10.1177/0192512105050379}.

\leavevmode\vadjust pre{\hypertarget{ref-posnerWhatJudgesJustices1993}{}}%
Posner, Richard A. 1993. \emph{What {Do Judges} and {Justices
Maximize}?: (The {Same Thing Everyone Else Does})}. {Law School,
University of Chicago}. \url{https://books.google.com?id=ciFUHQAACAAJ}.

\leavevmode\vadjust pre{\hypertarget{ref-posnerHowJudgesThink2010}{}}%
---------. 2010. \emph{How {Judges Think}}. {Harvard University Press}.
\url{https://books.google.com?id=ZVUC8riEVPQC}.

\leavevmode\vadjust pre{\hypertarget{ref-rousseyOverburdenedJudges2018}{}}%
Roussey, Ludivine, and Raphael Soubeyran. 2018. {``Overburdened
Judges.''} \emph{International Review of Law and Economics} 55
(September): 21--32. \url{https://doi.org/10.1016/j.irle.2018.02.003}.

\leavevmode\vadjust pre{\hypertarget{ref-smekalMimopravniVlivyNa2021}{}}%
Smekal, Hubert, Jaroslav Benák, Monika Hanych, Ladislav Vyhnánek, and
Štěpán Janků. 2021. \emph{Mimoprávní Vlivy Na Rozhodování Českého
{Ústavního} Soudu:} {Brno}: {Masaryk University Press}.
\url{https://doi.org/10.5817/CZ.MUNI.M210-9884-2021}.

\leavevmode\vadjust pre{\hypertarget{ref-sunsteinAreJudgesPolitical2006}{}}%
Sunstein, Cass R., David Schkade, Lisa M. Ellman, and Andres Sawicki.
2006. \emph{Are {Judges Political}? {An Empirical Analysis} of the
{Federal Judiciary}}. {Brookings Institution Press}.
\url{https://www.jstor.org/stable/10.7864/j.ctt12879t7}.

\leavevmode\vadjust pre{\hypertarget{ref-vartazaryanSitOvaAnalyza2022}{}}%
Vartazaryan, Gor. 2022. {``Sít'ová Analỳza Disentujících Ústavních
Soudců.''} \emph{Pravnik}, no. 12.

\leavevmode\vadjust pre{\hypertarget{ref-wittigOccurrenceSeparateOpinions2016}{}}%
Wittig, Caroline. 2016. \emph{The {Occurrence} of {Separate Opinions} at
the {Federal Constitutional Court}}. {Logos Verlag Berlin}.
\url{https://doi.org/10.30819/4411}.

\end{CSLReferences}

\end{document}
