% ARTICLE 2 ----
% This is just here so I know exactly what I'm looking at in Rstudio when messing with stuff.
% Options for packages loaded elsewhere
\PassOptionsToPackage{unicode}{hyperref}
\PassOptionsToPackage{hyphens}{url}
%
\documentclass[
  11pt,
]{article}
\usepackage{lmodern}
\usepackage{amssymb,amsmath}
\usepackage{ifxetex,ifluatex}
\ifnum 0\ifxetex 1\fi\ifluatex 1\fi=0 % if pdftex
  \usepackage[T1]{fontenc}
  \usepackage[utf8]{inputenc}
  \usepackage{textcomp} % provide euro and other symbols
\else % if luatex or xetex
  \usepackage{unicode-math}
  \defaultfontfeatures{Scale=MatchLowercase}
  \defaultfontfeatures[\rmfamily]{Ligatures=TeX,Scale=1}
\fi
% Use upquote if available, for straight quotes in verbatim environments
\IfFileExists{upquote.sty}{\usepackage{upquote}}{}
\IfFileExists{microtype.sty}{% use microtype if available
  \usepackage[]{microtype}
  \UseMicrotypeSet[protrusion]{basicmath} % disable protrusion for tt fonts
}{}
\makeatletter
\@ifundefined{KOMAClassName}{% if non-KOMA class
  \IfFileExists{parskip.sty}{%
    \usepackage{parskip}
  }{% else
    \setlength{\parindent}{0pt}
    \setlength{\parskip}{6pt plus 2pt minus 1pt}
    }
}{% if KOMA class
  \KOMAoptions{parskip=half}}
\makeatother
\usepackage{xcolor}
\IfFileExists{xurl.sty}{\usepackage{xurl}}{} % add URL line breaks if available
\urlstyle{same} % disable monospaced font for URLs
\usepackage[margin=1in]{geometry}
\usepackage{longtable,booktabs}
% Correct order of tables after \paragraph or \subparagraph
\usepackage{etoolbox}
\makeatletter
\patchcmd\longtable{\par}{\if@noskipsec\mbox{}\fi\par}{}{}
\makeatother
% Allow footnotes in longtable head/foot
\IfFileExists{footnotehyper.sty}{\usepackage{footnotehyper}}{\usepackage{footnote}}
\makesavenoteenv{longtable}
\usepackage{graphicx}
\makeatletter
\def\maxwidth{\ifdim\Gin@nat@width>\linewidth\linewidth\else\Gin@nat@width\fi}
\def\maxheight{\ifdim\Gin@nat@height>\textheight\textheight\else\Gin@nat@height\fi}
\makeatother
% Scale images if necessary, so that they will not overflow the page
% margins by default, and it is still possible to overwrite the defaults
% using explicit options in \includegraphics[width, height, ...]{}
\setkeys{Gin}{width=\maxwidth,height=\maxheight,keepaspectratio}
% Set default figure placement to htbp
\makeatletter
\def\fps@figure{htbp}
\makeatother
\setlength{\emergencystretch}{3em} % prevent overfull lines
\providecommand{\tightlist}{%
  \setlength{\itemsep}{0pt}\setlength{\parskip}{0pt}}
\setcounter{secnumdepth}{5}

\ifluatex
  \usepackage{selnolig}  % disable illegal ligatures
\fi
\newlength{\cslhangindent}
\setlength{\cslhangindent}{1.5em}
\newlength{\csllabelwidth}
\setlength{\csllabelwidth}{3em}
\newenvironment{CSLReferences}[2] % #1 hanging-ident, #2 entry spacing
 {% don't indent paragraphs
  \setlength{\parindent}{0pt}
  % turn on hanging indent if param 1 is 1
  \ifodd #1 \everypar{\setlength{\hangindent}{\cslhangindent}}\ignorespaces\fi
  % set entry spacing
  \ifnum #2 > 0
  \setlength{\parskip}{#2\baselineskip}
  \fi
 }%
 {}
\usepackage{calc}
\newcommand{\CSLBlock}[1]{#1\hfill\break}
\newcommand{\CSLLeftMargin}[1]{\parbox[t]{\csllabelwidth}{#1}}
\newcommand{\CSLRightInline}[1]{\parbox[t]{\linewidth - \csllabelwidth}{#1}\break}
\newcommand{\CSLIndent}[1]{\hspace{\cslhangindent}#1}


\title{Why and when do (Czech) judges dissent: an empirical analysis of
the Czech Constitutional Court}
\author{true \and true}
\date{August 28, 2023}

% Jesus, okay, everything above this comment is default Pandoc LaTeX template. -----
% ----------------------------------------------------------------------------------
% I think I had assumed beamer and LaTex were somehow different templates.


\usepackage{kantlipsum}

\usepackage{abstract}
\renewcommand{\abstractname}{}    % clear the title
\renewcommand{\absnamepos}{empty} % originally center

\renewenvironment{abstract}
 {{%
    \setlength{\leftmargin}{0mm}
    \setlength{\rightmargin}{\leftmargin}%
  }%
  \relax}
 {\endlist}

\makeatletter
\def\@maketitle{%
  \newpage
%  \null
%  \vskip 2em%
%  \begin{center}%
  \let \footnote \thanks
      {\fontsize{18}{20}\selectfont\raggedright  \setlength{\parindent}{0pt} \@title \par}
    }
%\fi
\makeatother


\title{Why and when do (Czech) judges dissent: an empirical analysis of
the Czech Constitutional Court }

\date{}

\usepackage{titlesec}

% 
\titleformat*{\section}{\large\bfseries}
\titleformat*{\subsection}{\normalsize\itshape} % \small\uppercase
\titleformat*{\subsubsection}{\normalsize\itshape}
\titleformat*{\paragraph}{\normalsize\itshape}
\titleformat*{\subparagraph}{\normalsize\itshape}

% add some other packages ----------

% \usepackage{multicol}
% This should regulate where figures float
% See: https://tex.stackexchange.com/questions/2275/keeping-tables-figures-close-to-where-they-are-mentioned
\usepackage[section]{placeins}



\makeatletter
\@ifpackageloaded{hyperref}{}{%
\ifxetex
  \PassOptionsToPackage{hyphens}{url}\usepackage[setpagesize=false, % page size defined by xetex
              unicode=false, % unicode breaks when used with xetex
              xetex]{hyperref}
\else
  \PassOptionsToPackage{hyphens}{url}\usepackage[draft,unicode=true]{hyperref}
\fi
}

\@ifpackageloaded{color}{
    \PassOptionsToPackage{usenames,dvipsnames}{color}
}{%
    \usepackage[usenames,dvipsnames]{color}
}
\makeatother
\hypersetup{breaklinks=true,
            bookmarks=true,
            pdfauthor={Štěpán Paulík (Humboldt Universität zu Berlin,
\href{mailto:stepan.paulik.1@hu-berlin.de}{\nolinkurl{stepan.paulik.1@hu-berlin.de}}) and Gor
Vartazaryan (Charles University,
\href{mailto:gorike2000@gmail.com}{\nolinkurl{gorike2000@gmail.com}})},
             pdfkeywords = {empirical legal research, courts, dissents,
judicial behavior, political science, Bayesian statistics, regression
analysis},
            pdftitle={Why and when do (Czech) judges dissent: an
empirical analysis of the Czech Constitutional Court},
            colorlinks=true,
            citecolor=blue,
            urlcolor=blue,
            linkcolor=magenta,
            pdfborder={0 0 0}}
\urlstyle{same}  % don't use monospace font for urls

% Add an option for endnotes. -----



% This will better treat References as a section when using natbib
% https://tex.stackexchange.com/questions/49962/bibliography-title-fontsize-problem-with-bibtex-and-the-natbib-package

% set default figure placement to htbp
\makeatletter
\def\fps@figure{htbp}
\makeatother



\usepackage{longtable}
\LTcapwidth=.95\textwidth
\linespread{1.05}
\usepackage{hyperref}
\usepackage{float}

\newtheorem{hypothesis}{Hypothesis}

\usepackage{setspace}

% trick for moving figures to back of document
% really wish we'd knock this shit off with moving tables/figures to back of document
% but, alas...

% 
% Optional code chunks ------
% SOURCE: https://stackoverflow.com/questions/50702942/does-rmarkdown-allow-captions-and-references-for-code-chunks



\begin{document}

% \textsf{\textbf{This is sans-serif bold text.}}
% \textbf{\textsf{This is bold sans-serif text.}}


% \maketitle

{% \usefont{T1}{pnc}{m}{n}
\setlength{\parindent}{0pt}
\thispagestyle{plain}
{%\fontsize{18}{20}\selectfont\raggedright
\maketitle  % title \par

}




{
   \vskip 13.5pt\relax \normalsize\fontsize{11}{12}
   \MakeUppercase{Štěpán Paulík}, \small{Humboldt Universität zu Berlin,
\href{mailto:stepan.paulik.1@hu-berlin.de}{\nolinkurl{stepan.paulik.1@hu-berlin.de}}}   \par \vskip -3.5pt \MakeUppercase{Gor
Vartazaryan}, \small{Charles University,
\href{mailto:gorike2000@gmail.com}{\nolinkurl{gorike2000@gmail.com}}}   

}

}








\begin{abstract}

%    \hbox{\vrule height .2pt width 39.14pc}

    \vskip 8.5pt % \small

\noindent \small{The decision of a judge to dissent or not to dissent
opens up avenue for strategical considerations. Building on the
economic-strategic account of judicial behavior developed by Lee
Epstein, Richard A. Posner and William M. Landes, we develop and test
multiple hyptotheses on the Czech Constitutional Court. To test the
hypotheses, we utilize Bayesian regression analyses. We find that the
workload of a judge does not affect their dissenting behavior as
previous research in the US context suggests. We also find that a
dissent imputes substantial costs on the majority that produces longer
arguments to address a dissent. The effect is stronger the more
disagreement there is on the bench. Lastly, we cast doubt upon the
theory that dissents bring about significant collegiality costs to the
dissenter}


\vskip 8.5pt \noindent \emph{Keywords}: empirical legal research,
courts, dissents, judicial behavior, political science, Bayesian
statistics, regression analysis \par

%    \hbox{\vrule height .2pt width 39.14pc}



\end{abstract}


\vskip -8.5pt

{
\hypersetup{linkcolor=black}
\setcounter{tocdepth}{2}
\tableofcontents
}

 % removetitleabstract

{
\setcounter{tocdepth}{2}
\tableofcontents
}

\setlength{\parindent}{16pt}
\setlength{\parskip}{0pt}

% We'll put doublespacing here
\doublespacing
% Remember to cut it out later before bib
\vspace{30pt}

\hypertarget{introduction}{%
\section{Introduction}\label{introduction}}

Empirical legal research has been slowly but surely finding it's outside
the predominant US context. Historically though most of the empirical
studies have been conducted in the US, especially the Supreme Court,
context (such as Boyd, Epstein, and Martin 2010; Carrubba et al. 2012;
Epstein, Landes, and Posner 2011). We now know that judgments are what
judges had for a breakfast. Put less pompously, there are many theories
and approaches for explanation of judicial decisions (Posner 2010) e.
What we do not know is the extent to which these theories and
explanations carry over to other legal systems and context.

In our article, we set out to replicate one of the empirical legal
research studies conducted in the US by Epstein, Posner and Landes
(Epstein, Landes, and Posner 2011) on the on the Czech Constitutional
Court (``CCC'') and to test whether the same theory and same empirical
conclusions hold in a different context.

Epstein et al.~study under which circumstances do US judges generally
dissent. More specifically, they build a formal economic model based on
the strategic account of judicial behavior. The study's empirical part
is firmly rooted in their theoretical part: they draw and test multiple
hypotheses, which flow from their theory. In particular, they test the
dependence of dissent rate on workload, the dependence of dissent rate
on size of courts, the dependence of dissent rate on the ideological
distance, and the dependence of length of majority argumentation on the
presence of a dissenting opinion. We will test similar hypotheses on the
CCC based on the same theories.

We adapt the theories constructed in the US context to the civil law and
Czech judiciary contexts. Based on the adaption, we draw hypotheses that
resemble the original ones. We test whether the length of majority
argumentation depends on the presence of one or more dissents due to,
whether the workload of a judge affects their dissenting behavior,
whether the dissenting behavior of judges changes at the start and end
of their terms, and, lastly, whether relationships formed during the
plenary sessions, as posited by the Czech legal scholarship, carry over
to 3-member panel proceedings.

We find that the workload of a judge does not affect their dissenting
behavior as previous research in the US context suggests. We also find
that a dissent imputes substantial costs on the majority that produces
longer arguments to address a dissent. The effect is stronger the more
disagreement there is on the bench. Lastly, we cast doubt upon the
theory that dissents bring about significant collegiality costs for the
dissenter.\footnote{All the code and the link to data for replication is
  available at \url{https://github.com/stepanpaulik/court_dissents}}

In this article, we utilize Bayesian regression analyses. In other
words, we rely on quantitative methods of social science research. That
is not without its limits. To not rely on one methodological approach,
we set this article into a broader mixed effects effort. We conducted
semi-structured interviews with the CCC judges to gain deeper
understanding of the judicial behavior, to paint more details to our
quantitative effort, and to help with further development of our theory.
Parts of the interviews mirror our findings in this article. To name an
example, in the interviews, we inquire into the effect of the coalitions
at the CCC, whose effect we measure in the last part of this article. In
terms of the output of our effort, we divide it into two articles: one
utilizing the quantitative methods, the other utilizing qualitative
methods.

Our article proceeds as follows. We start out with a theory. We explain
the main differences between the expectations based on the theory in the
CCC context in comparison to the SCOTUS context and based on that we
draw the hypotheses for the empirical part. We then explain the choice
of our broad methodological framework: the Bayesian statistics. We
proceed to test the hypotheses in empirical part divided into sections
one per each research question. Lastly, we discuss the results.

\hypertarget{theory}{%
\section{Theory}\label{theory}}

\hypertarget{current-accounts-of-judicial-behavior}{%
\subsection{Current accounts of judicial
behavior}\label{current-accounts-of-judicial-behavior}}

In general, there are multiple accounts of behavior of judges'. The
first that had dominated until \textasciitilde the end of 20th century
posited that judges are policy oriented. A lot of research has been
conducted on whether, how and to what extent do judges indeed seek to
advance the policies they desire (Berdejó and Chen 2017; Clark and
Lauderdale 2010; Dworkin 1980; Kastellec 2016; Moyer and Tankersley
2012).

However, as of recently, the perspective on judges has shifted. Judges
are now allegedly strategic and rational actors. One of the early
pioneers of this approach Posner (1993) presents a simple model of
judicial utility as function mainly of income, leisure and judicial
voting. Further research followed the Posner mode and presented
alternative models of judicial utility (based on economic psychology
Foxall (2004)). Replacing the policy oriented approaches, which hold
judges to pursue political policy oriented goals, researchers now focus
more on their self-interest in terms of career progression, higher
income, or lesser workload (Carrubba et al. 2012; Epstein and Knight
2000). Posner (2010) presents nine theories of approach for judicial
behaviour, from which we mostly, similarly to Epstein et al., draw on
the economical and sociological theory. Economic theory of judicial
behaviour treats the judges as a rational, self-interested, utility
maximizer and sociological theory of judicial behaviour incorporates
factors of strategic calculation, emotion, and group polarization.

In their article Epstein et al.~test multiple hypotheses about the
circumstances under which do the judges of the US courts dissent or
under which they decide to not dissent (dissent aversion). Their
analysis is based on the strategic-economic framework of self-interested
strategically motivated judges. They presume that judges ``leisure
preferences, or, equivalently, effort aversion, which they trade off
against their desire to have a good reputation and to express their
legal and policy beliefs and preferences (and by doing so perhaps
influence law and policy) by their vote, and by the judicial opinion
explaining their vote, in the cases they hear.''

The authors find the strategic aspect of dissenting in how a judge
squares their decision to either dissent or to avert a dissent based on
the costs and benefits of a dissent. The authors claim that
``{[}s{]}ince writing a dissenting opinion requires effort, which is a
cost, a judge will not dissent unless he anticipates a benefit from
dissenting that offsets his cost.'' The majority also accrues costs from
dissenting. In the words of Epstein et al.: ``{[}d{]}issenting imposes
an effort cost on the majority as well and sometimes a reputation cost
too, if the dissenting opinion criticizes the majority force- fully. To
minimize the dissenter's criticisms and retain the vote of the other
judge in the majority (in a panel of three judges, the normal number of
judges who decide a case in the federal courts of appeals), the author
of the majority opinion often will revise his opinion to meet, whether
explicitly or implicitly, the points made by the dissent.''

The benefits of a dissenting opinion are the potential to undermine the
majority opinion when the dissent is influential and the enhanced
reputation that the judge enjoys. The dissenting opinion may be cited in
the future by other judges or publicly analysed by legal scholars.

The theories they presume and hypotheses they test rest on this
framework: in the policy-oriented framework, it would not make sense to
expect judges less as their workload increases. They would still seek a
way to advance their political agenda and research has shown that
dissenting opinions usually correspond to exactly just that (Clark and
Lauderdale 2010). However, in the strategic account, the higher the
workload of a judge, the more pressing the effort costs of a dissent.
Similarly, if a dissenting opinion imputes costs on the majority, we can
theoretically expect it to respond to the dissent with a more thorough
or detailed argumentation in the majority opinion. It is safe to say
that the research questions are grounded in theory.

\hypertarget{disagreement-in-the-drivers-seat}{%
\subsection{Disagreement in the driver's
seat}\label{disagreement-in-the-drivers-seat}}

Our hypotheses and research design slightly differ. Our article focuses
only on a single court, the CCC, rather than focusing on multiple
courts. It also does not replicate the political distance research
question. Our research questions had to be adapted mainly for two
reasons.

Firstly, a major obstacle in conducting and carrying the US research
elsewhere is data availability. We narrow our object of analysis to the
CCC because there is the largest variation in the dissenting behavior
(unlike on the Supreme Administrative Court). While there is not yet a
full fledged dataset on the CCC (like the SCOTUS or CJEU Brekke et al.
2023), we have managed to build a complete yet unreleased dataset on the
CCC, which includes the metadata about the cases as well as the text
corpus. However, Epstein, Posner and Landes include in their analysis
the ideological distance between judges. The ideological distance serves
as one of the explanatory variables for dissent aversion. The measures
of ideological position of judges mainly rely on information about their
voting behavior. Regrettably such information is in continental legal
systems typically not made public: the votes in cases are kept secret.
Therefore, it is near impossible to construct a measure of the political
position of judges without knowing how they voted in each case.

Secondly, we utilize variation of different institutional settings,
which allows us to answer very similar questions. While Epstein et
al.~explore the variation between SCOTUS and Federal Courts, we're able
to explore variation within the CCC. The CCC is structured so that it
can either decide cases in 3-member panels or a plenary session.
Moreover, we are able to use the variation of a judge rapporteur. To
apprehend the institutional variation, we interpose a short section on
the CCC institutional setup and the difference in comparison to the
SCOTUS, which has repercussions on our theory.

The CCC consists of 15 judges: a Chairman, two Vice-chairmains and
twelve judges who are members of the permanent 3- member panels
consisting of three judges. The CCC justices are elected for 10 years
and the appointment process is akin to that of SCOTUS: the President
proposes a candidate that is confirmed by the Czech Senate.

The CCC is currently entering its fourth decade, having been established
in 1993, with 3 ``generations'' of judges having been rotated so far
with the fourth term of the CCC being just around the corner. Most
importantly, the CCC can decide a case in two formations: there are four
3-member panels and a plenum, which attracts procedurally specified.
Thus, the size of the deciding body varies within the court. So does the
type of cases that get assigned to either type of body. We address this
issue later on.

The room for the dissenting judge and the majority to address each other
differs between the two bodies. Based on our internal insight, there is
less back and forth interplay between the judges, more akin to the
SCOTUS context, and most of the communication is handled remotely,
whereas the plenum meets regularly to discuss the cases in person.
Despite that the process of generating dissents is the same. In both
cases, the rapporteurs are informed about the outcome of the vote, which
is filed in the voting record. The dissenting opinion is then sent to
the judge rapporteur before the decision is announced, as it cannot be
added until after the announcement.\footnote{It is interesting to note
  that in some cases the rapporteur judge can dissent against their own
  decision. Such a behavior mostly occurs in cases where the rapporteur
  is forced to omit an idea in the reasoning they would've otherwise
  included. The rapporteur then appends the omitted part as a concurring
  opinion.} It is important to note that judges have the possibility,
not the obligation, to dissent. In other words, there is room for judges
to give way to strategic considerations.

It may then seem that it would be futile to measure the impact of the
dissenting opinion on the majority opinion when the majority may have
not even be given a chance to familiarize itself with it before its vote
on the draft opinion.

However, to advance the theory further, what we believe that a presence
of a dissenting opinion truly captures is the expressed disagreement
among CCC judges. Since the individual cases are debated among the
judges, whether in person or remotely, it is possible to observe which
side a judge takes during the discussion before the final vote. The
judges have in either deciding body ample room to voice their
disagreement, even if they write the dissenting opinion last minute.
There is no evidence that would suggest otherwise - that the normal
behavior would be to remain silent until the vote and then present the
majority with a dissenting opinion.\footnote{We are aware of one judge
  whose behavior resembles the description, the rest voice their
  disagreement openly.} Thus, technically speaking, the true explanatory
variable of our theory is at all times the varying disagreement among
judges. We, in turn, capture that with the presence and the yearly
number of dissenting opinions.

Thus, on the one hand, we narrow our inquiry in comparison to Epstein et
al.~and omit the research questions that require the knowledge of the
ideological position of the judges. On the other hand, we are able to
utilize the variation at the CCC that's not present at the SCOTUS:
namely the division of the CCC into 3-member panels and one Plenum and
the limited terms of CCC judges. As a result of that we are able to add
two additional research questions.

Epstein et al.~base their theory on an economic model of dissenting, in
which a decision to dissent imposes collegiality costs on the dissenter.
They then test whether the variation of panel sizes impacts the
likelihood of dissent, as bigger panels impose lower collegiality costs.
Similarly, one of the profits of a dissent may be increased reputation
costs (as explained below). To test the collegiality costs theory, we
measure whether the dissenting behavior of CCC judges changes at the
start and end of their term. The CCC judges enjoy a 10 years long
limited term, which they may repeat if they go through a re-election
before the Senate. Our hypothesis is that as the collegiality costs
decrease before the end of the judges' terms, their likelihood to
dissent may go up. We will also analyse whether judges changed their
dissenting behavior depending on whether they decided to face a
re-election or not.

Lastly, the research on judicial coalitions at the CCC has revealed that
the third period of CCC is the most polarized, and there are clearly two
big coalitions of judges that clash against each other. However this
literature focuses only on plenary decisions and is of rather
superficial descriptive nature. We hypothesize that the relationship
from the plenary sessions carry over to the 3-member panel hearings. Our
hypothesis is that panels composed of judges from both coalitions will
be more likely to show disagreement in the form of dissenting opinion.
If this shows to be true, it would provide further evidence to the two
coalition theory of the CCC (Chmel 2021; Vartazaryan 2022; Smekal et al.
2021).

We therefore test the following hypotheses:

RQ1: The presence of dissent positively affects the length of the
majority argumentation.

RQ2: The higher the workload of a judge, the lower their dissent rate as
a result of higher effort costs.

RQ3: The judges are more likely to dissent at the end of their terms as
the collegiality costs decrease.

RQ4: Judicial coalitions formed in the plenary proceedings affect the
likelihood to dissent in 3-member panels. Having a panel composed of
members of both judicial coalitions increases the disagreement on the
bench and, thus, the likelihood of a dissent.

\hypertarget{method}{%
\section{Method}\label{method}}

Our goal, as we have noted, is to replicate the Lee Epstein et al.~study
in the context of the CCC. The original study employs quantitative
methods, namely a linear regression (including a log transformation in
one case) on observational data.

We will deviate in some aspects from the original study. Firstly, we
utilize the \emph{Bayesian} rather than \emph{frequentist} framework of
statistics. Secondly, we adapt the statistical models to a different
regulatory and institutional setting of the CCC. Thirdly, we change the
statistical models so that only relevant explanatory and confounding
variables remain. Despite that, regression models lay at the core of our
research. We now discuss the reasoning behind opting for Bayesian
statistics as the framework for our quantitative analysis. We address
the adaptation of the models and the selection of variables for each
research question separately.

\hypertarget{bayesian-framework}{%
\subsection{Bayesian framework}\label{bayesian-framework}}

Without delving too much into the Bayesian versus frequentist
statistics, we opt for the Bayesian framework for it, we believe,
reflects better our understanding of probability and scientific inquiry.
There are two major differences in understanding of concepts between the
two approaches towards statistics: that of role of prior knowledge and
that of probability.

In the frequentist framework, prior knowledge does not play too much of
a role and the inference is shaped solely by the observed data, whereas
in the Bayesian framework prior knowledge is updated with new data to
form new posterior conclusions. In other words, the Bayesian
statistician concerns themselves not with the uncertainty of the data
but also with how it fits into his prior knowledge.

That is reflected in different understandings of probability. The
frequentist understanding of probability refers to the long-run relative
frequency of a repeatable event. In other words, the main concern of
frequentist statistics is what would the frequency of any event be if we
could repeat it as many times as possible. The Bayesian probability
measures the relative plausibility of an event (Johnson, Ott, and Dogucu
2022).

Science in general is based on the frequentist framework. The typical
quantitative studies are driven by finding a low enough p-value, i.e.,
the measure of probability of having observed data as or more extreme
than the observed data if in fact the original null hypothesis is
incorrect. In simple terms, the search for statistical significance is a
search for data so unlikely to have occurred due to chance, even if we
could gather them again and again.

The Bayesian framework rather than measuring the uncertainty about
observed data measures the uncertainty of the parameters of interests,
given the observed data and our prior knowledge. In simple terms, the
Bayesian statistician puts into doubt their conclusions about parameters
of a certain model, given the observed data and their prior knowledge.
Mathematically, the uncertainty is reflected in the fact that the
posterior parameters are drawn from a posterior distribution of the
model and are just an approximation of thereof in the form of
probability density function rather than a single value. The posterior
distribution of a parameter comes from simulating in our case 40000 (4
chains*10000 simulations) possible posterior models via the Monte Carlo
Markov Chain simulations.

The Lee Epstein et al.~original study employs the frequentist framework
and hence they report statistically significant relationships between
the variables of interest. Our approach will be Bayesian. We will
firstly see whether the employed models actually make sense for the data
by running posterior predictive checks and afterwards we will draw the
parameters from the posterior distribution to see whether their
distribution indeed looks similar to the ones reached by Epstein et al.

\hypertarget{effect-of-presence-of-dissenting-opinion-on-the-length-of-majority-argumentation}{%
\section{Effect of presence of dissenting opinion on the length of
majority
argumentation}\label{effect-of-presence-of-dissenting-opinion-on-the-length-of-majority-argumentation}}

According to Epstein, Landes and Posner, ``{[}a{]} dissent imposes an
effort cost on the majority because the author of the majority opinion
is likely to revise his opinion to address the objections raised by the
dissent. This suggests that the majority opinions will be longer when
there is a dissent.''

To test this hypothesis, Epstein, Landes and Posner collected roughly
446 SCOTUS and 1025 US court of appeals decisions. They then create
regression models to see whether the presence of at least one dissenting
opinion affected the length of the majority opinion. The regression
model of Epstein et al.~naturally controls for multiple covariates, so
that bias is eliminated and a causal interpretation of the result is
made possible. The outcome variable of their model was the number of
words of the majority opinion. The explanatory and control variables in
their model were:

\begin{enumerate}
\def\labelenumi{(\arabic{enumi})}
\tightlist
\item
  whether the decision included oral hearing,
\item
  and (3) a dummy variable for presence of 1 or 2 and more dissents (the
  explanatory variable),
\item
  whether the majority mentioned the dissent,
\item
  a presence of concurring opinion,
\item
  a dummy for subject matter,
\item
  a dummy for the term of the court,
\item
  importance of the case, proxied by the number of references to the
  case in caselaw of SCOTUS and courts of appeal
\end{enumerate}

Based on application of their model to the SCOTUS data, the trio of
authors found a statistically significant and positive relationship
between presence of at least two dissenting opinions and the length of
the majority decision. Unsurprisingly, the authors also found a
statistically significant positive relationship between the outcome
variable and the importance of the case.

We test a very similar hypothesis with a slight tweak. We assume that
what becomes longer as a result of a presence of a dissent is not the
majority decision as such but its argumentation part, i.e., we are not
including, for example, the heading or the procedural history, which
both make up the majority decision. A following hypotheses can be
distilled:

RQ\textsubscript{1}: The presence of dissent positively affects the
length of the majority argumentation.

\hypertarget{adapting-the-epstein-et-al.-regression-model-to-the-ccc-context}{%
\subsection{Adapting the Epstein et al.~regression model to the CCC
context}\label{adapting-the-epstein-et-al.-regression-model-to-the-ccc-context}}

There are multiple obstacles we had to overcome to conduct a similar
model on the CCC. Firstly, while we are aware that the importance or
salience of a case is probably the key confounding variable, measuring
as a post-treatment amount of references of that decision might
introduce further bias. Secondly, we think inclusion of some of the
covariates is unnecessary and may even constitute a bad control
variable, insofar we do not think there is any potential avenue for
confounding. Here we present a brief overview of issues we faced with
the original model and how we solved them.

\hypertarget{conceptualizing-importance-of-a-case-as-a-control-variable}{%
\subsubsection{Conceptualizing importance of a case as a control
variable}\label{conceptualizing-importance-of-a-case-as-a-control-variable}}

In general, when utilizing a regression design with observational data,
as is our case, a researcher must satisfy certain conditions to be able
to interpret regression results as a causal relationship. The first is
the stable unit treatment value assumption (SUTVA) and the other is
usually referred to as \emph{conditional independence assumption}. This
condition requires that the assignment of treatment T to unit \emph{i}
is independent of any covariates \emph{X} of the unit \emph{i} that also
influence their outcome \emph{Y}. The CIA may be formalized as follows:

\[
{Y_{1i}, Y_{0i}} {\bot} T_{i}|X_{i}
\]

The notation follows the potential outcomes framework.
\{Y\textsubscript{1i}, Y\textsubscript{0i}\} refer to the outcome of a
unit i with or without treatment, in our case the presence of at least
one dissenting opinion. There are, in general, two types of causes of
bias: a confounding variable, which breaks the CIA, and reverse
causality. A confounding variable is such that

\begin{enumerate}
\def\labelenumi{(\arabic{enumi})}
\tightlist
\item
  has an effect on treatment status,
\item
  has an effect on the outcome over and above its effect on the
  treatment status.
\end{enumerate}

Not controlling for confounding variables causes an omitted variable
bias that in turn precludes causal interpretation of the regression.

While at first glance it may thus seem that researchers should throw in
as many covariates as possible, that is in reality not the case. There
are examples of bad or unnecessary controls that are themselves an
outcome of the treatment (for a detailed discussion see Angrist and
Pischke 2009, 2014; Montgomery, Nyhan, and Torres 2018). Among them are
post-treatment variables, which imply that all control variables must
occur before the treatment takes place.

We believe that importance of a decision is a potential confounding
variable as it clearly may impact both the length of a judgment as well
as the likelihood of a dissent. However, proxying it by the number of
citations in ensuing caselaw may present a bad post-treatment control
variable because it occurs and is measured after the decision to dissent
or not to dissent has been made. We believe that in our context, the
\emph{formation of the CCC} and the \emph{type of the decision} are
better pre-treatment proxies for the importance of a decision.

Institutionally, the CCC can decide cases either in 3-member panels or
in the whole plenary session. Put simply, we assume that more important
cases are decided in plenary rather than 3-member panel formation. The
plenary is more likely to rule on merits and its decisions are,
therefore, on average longer. Moreover, the dissent rate in the plenary
decisions is also higher. Thus, the formation of the CCC reflects the
importance of the case being decided, which has a confounding potential.

We confirm our intuition by comparing our metric of importance of a case
to Epstein's metric in the CCC context. While plenary decisions make up
only 1.5 \% of all CCC decisions, they make up 15 \% references in the
CCC caselaw. Unlike the Epstein metric, our metric is in any case
determined before the decision to dissent or not to dissent is made.
Thus, in our model, we're including a dummy variable for the formation
of a court, i.e., whether the decision was made in a 3 member panel or a
full court plenary. We find similar numbers regarding the distribution
of decisions on the merits and on the procedure.

\begin{longtable}[]{@{}lrlrl@{}}
\toprule\noalign{}
formation & Count\_total & Percent\_total & Count\_cited &
Percent\_cited \\
\midrule\noalign{}
\endhead
\bottomrule\noalign{}
\endlastfoot
Panel & 91208 & 98.4 \% & 1329 & 92 \% \\
Plenary & 1454 & 1.6 \% & 116 & 8 \% \\
\end{longtable}

Similarly, the type of decision affects both the length of argumentation
as well as the likelihood of dissent. The CCC can decide either
procedural (``usnesení'') or on merits (``nález''). The latter type of
decisions are on average longer and contain disproportionately more
dissents than procedural decisions. We assume that the type of decision
to some extent also reflects the importance of a case.

\hypertarget{unnecessary-or-untransferrable-control-variables}{%
\subsubsection{Unnecessary or untransferrable control
variables}\label{unnecessary-or-untransferrable-control-variables}}

Secondly, we believe not all variables in the Epstein et al.~model have
potential for confounding or are transferable to the CCC context. The
majority mentioning the dissent does not impact the judge's decision to
or not to dissent. What's more, a mention of a dissent by the majority
can only follow after a decision to dissent was made. Thus, we omit it
from our model. Oral hearings are few and far between in the CCC
context. It therefore makes no sense to control for the presence of oral
hearing in the context of the CCC.

\hypertarget{data-collection-and-method}{%
\subsection{Data collection and
method}\label{data-collection-and-method}}

The data used for this analysis includes the dataset CCC decisions. The
Czech Apex Court dataset was built by Štěpán Paulík and includes
complete database of decisions of the CCC, SAC and the Czech Supreme
Court, including comprehensive metadata, text corpus, as well as
additional information mined from the texts or publicly available
sources, such as case references, compositions, or background
information of the judges. The analysis was limited up until the end of
2022. We filtered only those decisions, where a variation in the
dissenting behavior could procedurally be observed: procedural decisions
were filtered out as they require unanimity amongs the judges and do not
leave any space for disdagreement.

While the metadata of decisions contain information about a presence of
a dissenting opinion, regrettably, the CCC decisions are not neatly
split up like their SCOTUS counterparts. The text of the decision
contains both the majority opinion as well as the dissent without any
clear boundary from other parts of the structure. That is why we relied
on machine learning to extract the information about presence, position
of dissenting opinion as well as length of the majority argumentation in
the CCC decisions.

Utilizing machine learning unlocked one more avenue to further improve
the Epstein et al.~model. Instead of conceptualizing the length of the
majority opinion as the whole majority opinion, we narrowed our inquiry
only to the majority argumentation rather than including, for example,
the heading or the facts of the case in the length of a majority opinion
variable. Therefore, we believe our model better reflects the
relationship between the presence of a dissent and the legal
argumentation of the majority.

To extract the length of dissents and length of majority argumentation,
multiple supervised classification algorithms were trained following
similar structure-mining attempts within the Czech context (Eliášek,
Kól, and Švaňa 2020; Harašta et al. 2019; and elsewhere Lüders and
Stohlman 2023). A sample of 200 decisions was manually annotated on a
paragraph level. The paragraphs were then represented either as dense
doc2vec vectors based on word2vec model of the whole CCC text corpus
(Mikolov et al. 2013) or as sparse document-term-matrix with the td-idf
values for each word. Positional encoding of a paragraph was added to
both representations of text. Because there was a large unbalance
between the classes, oversampling algorithm SMOTE was applied to balance
out the dataset, which is a standard practice when working with smaller
datasets (Fjelstul 2021).

Our classification followed in two stages. In the first step, the
dissenting opinions were classified from the rest of the decision so
that the positional encoding remained consistent across all cases.
Otherwise, in the decision without a dissenting opinion, usually a
conclusion or court argumentation was at the end of the decision,
whereas in the decisions with a dissent, the dissenting opinion was at
the end of the decision. That created confusion with the positional
encoding. Our first stage classification allowed us to separate the
dissenting opinion from the rest of the decision and then recalculate
the positional encoding for the remaining paragraphs. In the second
step, the remaining text as well as the decisions that did not contain a
dissenting opinion were classified into an inner structure consisting
of:

\begin{enumerate}
\def\labelenumi{(\arabic{enumi})}
\tightlist
\item
  heading
\item
  verdict
\item
  procedure history
\item
  complainant arguments
\item
  court arguments
\item
  conclusion
\item
  information on further legal remedies
\item
  signature
\end{enumerate}

In line with the findings of Eliasek and Lüders articles, we tested and
compared three classification algorithms: - Support Vector Machines
(Gandhi 2018) with the sparse td-idf representation and Gradient Boosted
Decision Trees (Maklin 2019) and Random Forests with the dense doc2vec
embeddings. More complex algorithms did not provide any improvement in
accuracy at the cost of higher computing costs. The benchmark for both
classifiers was the zero rule, i.e., the proportion of the majority
class as a zero rule classifier would presume that all occurrences are
of that class and would be right the proportion amount of the time.

In the end, the XGBoost algorithm combined with the doc2vec embeddings
boasted the highest benchmark values (precision, accuracy, FScore). The
precision of the first stage classifier was \textasciitilde86 \%, above
the 75 \% zero rule benchmark. The precision of the second stage
classifier was \textasciitilde82 \%, well above the 37 \% zero rule
benchmark. Afterwards, both classification models were trained on all
annotated data and used to predict classes of the whole dataset.

\hypertarget{result}{%
\subsection{Result}\label{result}}

\hypertarget{the-model}{%
\subsubsection{The model}\label{the-model}}

Building on the theory and the Epstein model, in our model, we included
the number of words of the court argument part of a decision as the
outcome variable. Regarding the explanatory variables, we opted for

\begin{enumerate}
\def\labelenumi{(\arabic{enumi})}
\tightlist
\item
  a dummy variable signifying a presence of one dissent as an
  explanatory variable,
\item
  a dummy variable signifying a presence of two or more dissents as an
  explanatory variable,
\item
  the formation of the CCC as a control variable,
\item
  the type of the decision as a control variable,\\
\item
  year of the decision as a control variable.
\end{enumerate}

We conducted our analysis employing Bayesian regression implemented by
the software Stan in R. We opted for a completely pooled model as the
data did not contain any inherent structure (there were no clusters). At
first glance, we assumed that

\[
Y | \lambda \sim Pois(\lambda)
\]

because our outcome variable of interest is a discrete count and the
density plot of the length of court argumentation suggests so.

\includegraphics{dissents_article_files/figure-latex/negbinom-1.pdf}

However, as the posterior checking revealed, the Y was actually
overdispersed and the Poisson regression was not able to capture the
overdispersion. Therefore, we instead opted for the Negative Binomial
model, which allows for relaxing the assumption of equality of variance
of Y to its expected value. Thus, the explanatory variable, the number
of words of argumentation of the CCC \emph{Y}

\[
Y_{words} | \mu, r \sim NegBin(\mu, r)
\] As for the priors, we based the priors on the Epstein results as well
as a cursory exploratory peak into the data.

\hypertarget{diagnosis}{%
\subsubsection{Diagnosis}\label{diagnosis}}

We ran the model via Stan with 4 Monte Carlo Markov Chains (MCMC) of
20000 iterations each, the first 10000 warm up iterations being
discarded. The trace plot shows that the chains were stable and probed
plausible parameter values, the density plots of the MCMC show that all
4 chains exhibited similar behavior, and the autocorrelation between the
iterations always dropped quickly and that the chains were moving around
the potential parameter values quickly.

The posterior diagnosis confirms that although the simulations are not
perfect, they do reasonably capture the features of the observed number
of words of court arguments. In other words, we selected the correct
model and the priors are not too off either. Thus, our Negative Binomial
regression assumptions are reasonable.

\includegraphics{dissents_article_files/figure-latex/pp_check_negbinom-1.pdf}

\hypertarget{posterior-interpretation}{%
\subsubsection{Posterior
interpretation}\label{posterior-interpretation}}

Parameters of all variables of interest are significantly different from
0 as revealed by the density and whisker plots with 80 \% and 95 \%
posterior credible intervals.

\includegraphics{dissents_article_files/figure-latex/length_parameter_plots-1.pdf}
\includegraphics{dissents_article_files/figure-latex/length_parameter_plots-2.pdf}

Even after controlling for all potential observable confounding
variables, the regression table looks as follows

\begin{longtable}[]{@{}lrrrr@{}}
\toprule\noalign{}
term & estimate & std.error & conf.low & conf.high \\
\midrule\noalign{}
\endhead
\bottomrule\noalign{}
\endlastfoot
(Intercept) & 2528.12 & 1.01 & 2496.75 & 2560.25 \\
one\_dissent & 1.77 & 1.04 & 1.68 & 1.87 \\
more\_dissent & 2.99 & 1.05 & 2.82 & 3.18 \\
formationPlenum & 1.65 & 1.03 & 1.58 & 1.71 \\
type\_decisionUsnesení & 0.25 & 1.04 & 0.24 & 0.26 \\
\end{longtable}

In other words, the presence of one dissent implies an average
\emph{e}\textsuperscript{0.57} increase of words in the court arguments
part of judgment. Put in terms of percentage, a presence of one dissent
increases the length of the argumentation by 77 \%. The presence of two
or more dissents implies an average \emph{e}\textsuperscript{1.10}
increase of words in the court arguments part of judgment. That is a
staggering \textasciitilde200 \% increase in the length as a result of
presence of two or more dissenting opinions. To this end, the result of
our study is in line with that of Epstein et al.: a presence of
dissenting opinion increases the length of the majority opinion
argumentation considerably.

We believe there are two potential ways to explain this behavior. Either
the majority simply takes the dissenting opinion seriously and addresses
the arguments raised in them or the presence of a dissenting opinion
reflects a deeper disagreement between judges that would have taken
place during the deliberation. Based on our knowledge of the inner
organisation of the court, the deeper disagreement explanation would fit
the plenary proceedings more accurately as a more thorough debate
usually takes place than in the 3-member panel proceedings. Such a
substantive explanation for our findings is supported by the fact that
decisions originating in the plenary proceedings are disproportionately
over-represented among decisions that contain at least one dissenting
opinion. This explanation is further supported by the larger effect of
having 2 or more dissents over just the 1 dissent. More dissenting
judges simply imply higher degree of disagreement on the bench. The
latter explanation fits within our theory of conceptualizing dissent as
a result of differing degree of disagreement on the bench. Be it as it
may, we conclude that the majority takes the disagreement seriously. Our
finding is in line with that of Epstein et al.

\hypertarget{effect-of-workload-on-the-dissenting-behavior}{%
\section{Effect of workload on the dissenting
behavior}\label{effect-of-workload-on-the-dissenting-behavior}}

\hypertarget{adapting-the-epstein-et-al.-regression-model-to-the-ccc-context-1}{%
\subsection{Adapting the Epstein et al.~regression model to the CCC
context}\label{adapting-the-epstein-et-al.-regression-model-to-the-ccc-context-1}}

According to Epstein et al.~``{[}t{]}he economic theory of judicial
behavior predicts that a decline in the judicial workload would lower
the opportunity cost of dissenting and increase the frequency of
dissents, and also that the greater the ideological heterogeneity among
judges the more likely they are to disagree and so the higher the
dissent rate will be.'' The authors then test both of these hypotheses -
whether the workload and the political distance between judges affects
dissent rate.

We cannot regrettably measure the ideological distance among CCC
judges.\footnote{At least not yet. One of the authors is currently
  working on a study that estimates the positions of opinions and, by
  extension, of judges. The extension to judges is built on the
  assumption that dissenting opinions correspond the most to the judges'
  ideological or political positions.} We omit that from our study and
stick only to the effect of changes in the workload of a judge. Our
second research question may be formulated as

RQ\textsubscript{2}: The higher the workload of a judge, the lower their
dissent rate as a result of higher effort costs.

The authors find a positive relationship between the log of dissent
rate, i.e., number of dissents divided by the number of cases and log of
caseload, i.e., the log of total number of cases decided after oral
arguments. The authors find that on the SCOTUS, ``a 10 percent decrease
in the caseload increases statistically significantly the dissent rate
by about 3.3 percent'' at a p-value of less than 0.05 Epstein, Landes,
and Posner (2011).

The authors again control for multiple variables, which we believe to be
unnecessary. They control for ideological differences between judges. It
is hard to see how the ideological difference between judges could
affect caseload, although it undoubtedly carries an explanatory value
for dissent rate variance, as the authors indeed conclude.

\hypertarget{model}{%
\subsection{Model}\label{model}}

Our model is built slightly differently. We utilize the variance in the
caseload among judge rapporteurs to capture the workload of a judge. We
could conceptualize the workloaf of a judge in multiple ways:

\begin{enumerate}
\def\labelenumi{\arabic{enumi}.}
\tightlist
\item
  the number of cases submitted and assigned to each judge rapporteur
  per year (we refer to this option as the caseload)
\item
  the number of unfinished cases as a judge rapporteur of any judge at
  the time of decision in any given decision (we refer to this option as
  workload), 3 and 4. as the yearly rate of change of thereof.
\end{enumerate}

Regarding the option 1, it follows the steps of Epstein et al., i.e., it
captures the caseload here is that of an individual judge of decisions
that they have to decide and write, or the rate of change of thereof.
That in our eyes reflects the idea of workload of each judge on the
court better than the original caseload of a whole court. The more cases
the judges have to author each year, the busier they are.

However, it is doubtful whether that is how a judge would perceive
workload. We believe our second measure captures the workload of a judge
better. We firstly mined the compositions of panels as well as the
plenary from the text of the decision. We then calculated the number of
unfinished cases each judge had at the time of any given decision as a
judge rapporteur using the date of submission and of decision of a case.
We believe such a measure captures the perceived workload of a judge
much better: a judge knowing that they have, for example, 20 in
comparison to 100 decisions to draft as a judge rapporteur is what
influences the decision to dissent or not.

Similarly, instead of measuring dissent rate on the Constitutional court
in general, we can regress the number of dissents written by each judge
either (similarly to Epstein et al.) per year on the caseload or we can
regress the decision to dissent of any given judge on their workload.
Ideally, we would measure both variables as rate of change. However,
there are many observations of the number of dissents with the value of
0. In these cases, the rate of change would either be infinite or not a
number (as a 0 would appear either in the denominator or numerator of
the rate of change formula). Getting rid of the zeroes would imply a
rather complex transformation (for a more detailed overview of the
possible transformations see Hyndman 2010). We include time as a control
variable because we presume that the workload of judges increases over
time and so may the number of dissents.

To address potential sources of bias in our regression analysis, we
consider the caseload of a judge to be assigned as good as random. The
cases once submitted to the CCC get assigned to individual judges based
on the alphabetic order of their surnames. There is no intentional case
selection in play. Therefore the assignment of the treatment, the
workload of a judge, is independent of other covariates and so is the
outcome of interest. The same applies to our last research question.

\includegraphics{dissents_article_files/figure-latex/caseload_over_time-1.pdf}
We opt for the Bayesian logistic regression model because the dependent
variable is a binomial variable with 1 trial, i.e.~our \emph{Y}, the
decision to dissent or not to dissent of a judge in any given decision:

\[
Y | \pi \sim Bern(\pi)
\]

\hypertarget{result-1}{%
\subsection{Result}\label{result-1}}

\hypertarget{model-diagnosis}{%
\subsubsection{Model diagnosis}\label{model-diagnosis}}

We tried two models, a completely pooled and hierarchical model
clustered around the judges. The main difference between the two models
is that the former model completely ignores individual intercept. The
latter allows for differentiating intercepts between the groups (in our
case the individual judges) and the global intercept. The global
parameter of interest is then informed both by the global trends as well
as the individual intercepts. That can usually lead to higher accuracy
in case of structured or time series data at the cost of higher
computational expenses.

We ran both the models via Stan with 4 Monte Carlo Markov Chains (MCMC)
of 20000 iterations each, the first 10000 warm up iterations being
discarded. We did a diagnosis of all the models. In all cases, the trace
plots show that the chains were stable and probed plausible parameter
values, the density plots of the MCMC show that all 4 chains exhibited
similar behavior, and the autocorrelation between the iterations always
dropped quickly and that the chains were moving around the potential
parameter values quickly.

We now compare the pooled against the hierarchical models. The former
model got the posterior prediction 1.2 of the number of dissents per
judge wrong (0.84 standard deviations off), whereas the former model got
the posterior prediction wrong only by 0.94 of the number of dissents
per judge (0.7 standard deviations off), with 99 \% of the predictions
falling within the 95 \% posterior credible interval. 6-fold
cross-validated check reveals that neither of the models overfitted.
Thus, while the hierarchical model is slightly more computationally
expensive, it yields better results.

\hypertarget{interpreting-the-posterior}{%
\subsubsection{Interpreting the
posterior}\label{interpreting-the-posterior}}

We can see that the estimate of the workload parameter significantly
differs from 0, as the 95 \% and 80 \% uncertainty intervals of the
posterior draws of the parameter lay on the left of 0. We're thus able
to proceed with substantive interpretation of the result.

\includegraphics{dissents_article_files/figure-latex/interpreting_posterior2-1.pdf}
\includegraphics{dissents_article_files/figure-latex/interpreting_posterior2-2.pdf}

The regression table is already transformed into odds from the log odds
output of the model. At first glance, the results are in line with our
theoretical predictions. The intercept is quite low as in the complete
pool of cases, dissents are far few and between.

\begin{longtable}[]{@{}lrrrr@{}}
\toprule\noalign{}
term & estimate\_odds & std.error & conf.low & conf.high \\
\midrule\noalign{}
\endhead
\bottomrule\noalign{}
\endlastfoot
(Intercept) & 0.007 & 1.068 & 0.007 & 0.008 \\
unfinished\_cases & 0.995 & 1.001 & 0.994 & 0.996 \\
\end{longtable}

We can see that for each increase of unfinished cases of a judge in any
decision, the outcome likelihood of dissent decreases by
\textasciitilde0.5 \% (\emph{e}\textsuperscript{-0.00475}). Because the
number of unfinished cases is usually in 10\^{}1 dimensions, the effect
isn't unsubstantial. The result from our analysis is in line with our
intuition. The CCC judges take into account the effort costs of dissent
and square it against their perceived workload. It would be almost
unreasonable to observe opposite effect.

\hypertarget{collegiality-costs-of-dissenting-at-the-ccc}{%
\section{Collegiality costs of dissenting at the
CCC}\label{collegiality-costs-of-dissenting-at-the-ccc}}

\hypertarget{theory-1}{%
\subsection{Theory}\label{theory-1}}

Epstein et al.~address the issue of collegiality costs arising for a
dissenting judge: ``The effort involved in these revisions, and the
resentment at criticism by the dissenting judge, may impose a
collegiality cost on the dissenting judge by making it more difficult
for him to persuade judges to join his majority opinions in future
cases.'' Based on this theory, they predict and indeed empirically
confirm that ``dissents will be less frequent in circuits that have
fewer judges because any two of its judges will sit together more
frequently and thus have a greater incentive to invest in
collegiality.''

While it is hard for us to see how a variation between the number of
members in the plenary session and 3-member panels could be isolated
from a plethora of potential confounding variables, we are able to make
use of the limited term of CCC judges. We test, based on the Epstein et
al.~theory, whether judges that are at the start of their term, and thus
are aware that they will ``sit together more frequently'' invest in
collegiality by averting dissents and whether when their term draws to
an end, they give way to their disagreement. This presumes that the
outlook of sharing the 10 year term with your colleagues at the
beginning of judges' terms increases the collegiality costs of
dissenting, whereas at the end of their terms, the collegiality costs
decrease with the end of the shared term looming on the horizon.

In the following section, we test the hypothesis:

RQ3: The judges are less likely to dissent at the beginning of their
dissents as the collegiality costs of dissent are steep and more likely
to dissent at the end of their terms as the collegiality costs decrease.

\hypertarget{model-1}{%
\subsection{Model}\label{model-1}}

We build on our previous model. We now know that the number of dissents
of a judge rapporteur follows a Poisson distribution. We use a
hierarchical model pooled on the judges. We have no knowledge whatsoever
about the effect of start or end of term on the number of dissents,
thus, we use only weakly informative priors. We have addressed the
potential sources of bias with the workload as an explanatory variable
above.

\hypertarget{result-2}{%
\subsection{Result}\label{result-2}}

\hypertarget{model-diagnosis-1}{%
\subsubsection{Model diagnosis}\label{model-diagnosis-1}}

We ran the model via Stan with 4 Monte Carlo Markov Chains (MCMC) of
20000 iterations each, the first 10000 warm up iterations being
discarded. We did diagnosis of all the models. The trace plots reveal
that the chains were stable and probed plausible parameter values, the
density plots of the MCMC show that all 4 chains exhibited similar
behavior, and the autocorrelation between the iterations always dropped
quickly and that the chains were moving around the potential parameter
values quickly.The posterior predictive check again reveals that our
posterior model reasonably captures the underlying data.

\includegraphics{dissents_article_files/figure-latex/posterior_check_term-1.pdf}

\hypertarget{interpreting-the-posterior-1}{%
\subsubsection{Interpreting the
posterior}\label{interpreting-the-posterior-1}}

The parameters of both variables of interest turned out to be
significantly differing from zero. Therefore, we're able to draw
conclusions from our model.

\includegraphics{dissents_article_files/figure-latex/interpreting_posterior_term2-1.pdf}
\includegraphics{dissents_article_files/figure-latex/interpreting_posterior_term2-2.pdf}

Consistently with the Epstein theory of collegiality costs of dissents,
we find that the likelihood of dissenting is lower by \textasciitilde20
\% (\emph{e}\textsuperscript{-0.23}) during the first two years of
judges' terms and that at the end of the judges' terms, their appetite
for disagreement and dissent increases by roughly 14 \%
(\emph{e}\textsuperscript{-0.13}).

\begin{longtable}[]{@{}lrrrr@{}}
\toprule\noalign{}
term & estimate & std.error & conf.low & conf.high \\
\midrule\noalign{}
\endhead
\bottomrule\noalign{}
\endlastfoot
(Intercept) & 2.87 & 1.10 & 2.53 & 3.24 \\
end\_term & 1.14 & 1.08 & 1.03 & 1.25 \\
start\_term & 0.80 & 1.08 & 0.72 & 0.88 \\
\end{longtable}

Our results are, again, consistent with those of Epstein et al.~The
model reveals that the judges seem to change their dissenting behavior
depending on in which phase of their term do they find themselves.

\hypertarget{the-effect-of-plenary-judicial-coalitions-on-dissenting-behavior-in-panels}{%
\section{The effect of plenary judicial coalitions on dissenting
behavior in
panels}\label{the-effect-of-plenary-judicial-coalitions-on-dissenting-behavior-in-panels}}

\hypertarget{theory-of-judicial-coalitions-at-the-ccc}{%
\subsection{Theory of judicial coalitions at the
CCC}\label{theory-of-judicial-coalitions-at-the-ccc}}

Lastly, we measure the impact of coalitions that formed in the plenary
proceedings on the behavior of judges on the dissenting behavior of
judges. According to Czech legal scholarship (Chmel 2021; Smekal et al.
2021; Vartazaryan 2022), two grand coalitions have formed on the third
term of the CCC between 2013-2023. The articles rely primarily on
network analysis of the dissenting opinions in the plenary proceedings.
We do not intend to delve into validating their conclusions.

Rather, we test whether the presumable existence of the coalitions carry
over to and have any effect on the dissenting behavior of judges in the
panels. Consistent with our theoretical part, we believe that a
dissenting opinion reflects disagreement on the judicial bench. Our
intuition suggests that if indeed there are two coalitions in the
plenary proceedings, which strongly disagree between each other, such a
disagreement should, theoretically, carry over to the panel level. Our
hypothesis is as follows:

RQ4: Judicial coalitions formed in the plenary proceedings affect the
likelihood to dissent in 3-member panels. Having a panel composed of
members of both judicial coalitions increases the disagreement on the
bench and, thus, the likelihood of a dissent.

\hypertarget{model-2}{%
\subsection{Model}\label{model-2}}

Practically speaking, we mined the compositions of panels from the text
of the decision in each and every decision. Following Chmel and
Vartazaryan, we split the 3rd term CCC into two coalitions: the first
coalition consisted of judges Kateřina Šimáčková, Vojtěch Šimíček,
Ludvík David, Jaromír Jirsa, David Uhlíř, Jiří Zemánek, Tomáš Lichovník,
Jan Filip, Milada Tomková and Pavel Šámal, whereas the second coalition
of judges consisted of Radovan Suchánek, Vladimír Sládeček, Josef Fiala,
Jan Musil, Jaroslav Fenyk, and Pavel Rychetský. The Vyhnánek article
goes so far to coin the first coalition as a more left-leaning and the
second as a more right-leaning.

We filtered the 3-member panel on merits decisions\footnote{At the
  3-member panel level, procedural decisions have to be unanimous.
  Therefore, they do not leave any variation of dissenting behavior and
  we leave them out of the model.} of the 3rd term CCC and coded the
following dummy variables. One dummy for each coalition, if all 3
members of the panel in any given decision were members of the same
coalition. On top of that, our model included one dummy variable, if one
member of the panel in the minority came from the other coalition of the
2 majority judges. The assignment to panels is as good as random, thus,
there is no need to control for other covariates. Because our outcome of
interest, the presence of a dissent in any given decision, is a binary
variable, we opted for the binomial logistic regression model. For the
model, we used weakly informative priors as we have no idea about the
potential effect of having the two coalitions

\hypertarget{results}{%
\subsection{Results}\label{results}}

While at first glance, the results are along the lines of what we
expected. The regression table is already transformed into odds from the
log odds output of the model.

\begin{longtable}[]{@{}lrrrr@{}}
\toprule\noalign{}
term & estimate\_odds & std.error & conf.low & conf.high \\
\midrule\noalign{}
\endhead
\bottomrule\noalign{}
\endlastfoot
(Intercept) & 0.03 & 21.44 & 0.00 & 1.75 \\
full\_coal\_1 & 0.68 & 21.60 & 0.01 & 32.89 \\
full\_coal\_2 & 0.00 & 2577638.25 & 0.00 & 0.05 \\
mixed\_coal\_1\_min & 2.03 & 21.58 & 0.04 & 99.94 \\
mixed\_coal\_2\_min & 2.12 & 21.35 & 0.04 & 104.20 \\
\end{longtable}

The estimated odds of dissent are already pretty low, the predicted
likelihood of dissent appearing in any 3-member panel decision on merits
is \textasciitilde3 \% (which isn't at odds with our workload model,
where the pool of considered decisions is slightly wider). In our data,
there is not a single dissent among a panel composed of completely of
members from the second coalition, which explains the 0. On the other
hand, the full presence of the first coalition decreases the odds of
dissent considerably by \textasciitilde20 \%. Lastly, the effects of
having both panels mixed are in either case positive, the odds of both
parameters are \textasciitilde2.

\includegraphics{dissents_article_files/figure-latex/interval_coalition-1.pdf}
\includegraphics{dissents_article_files/figure-latex/interval_coalition-2.pdf}

Interpreting the repercussions for theory is quite difficult, especially
given the lack of data on the second, more conservative coalition and
given the rather large standard errors. While we steer clear of using
the language of causal relationship, the trend is clear and has remained
robust to multiple model specifications: a panel composed of members
from both coalitions from the plenary proceedings increases the
disagreement on the bench. We could not confirm any heterogenous effects
between the two coalitions. We thus conclude that our data and evidence
seems reasonably compatible with the explanation of Czech legal
scholars.

\hypertarget{discussion}{%
\section{Discussion}\label{discussion}}

We successfully transplanted a research design from the US context to
the European context. We had to adjust it to the extent that data
availability precluded us for posing certain research questions or
applying certain methods. Our results are inconclusive as to the
potential to transfer conclusions from empirical legal research
conducted in the US context elsewhere.

On the one hand, we reached a similar conclusion regarding the costs a
dissenting opinion imputes on the majority. In the CCC context the
majority takes dissents seriously and appears to address them in their
opinions. This conclusion is further supported by the fact that the
deeper the disagreement seems to run, the more seriously it is taken by
the majority.

On the other hand, the CCC judges do not seem to act as strategically as
their US counterparts, at least in the context of dissenting opinions.
Increase in their workload (operationalized both as the absolute number
of cases assigned to a judge as well as the rate of change of thereof)
does not appear to decrease the likelihood of that judge dissenting, as
Epstein et al.~concluded.

Our last remark addresses our research design in general: the regression
analysis. We are aware that given the potential outcome frameworks, it
is difficult to sustain all the assumptions of regression research
design. Experimental or quasi-experimental research design such as
difference-in-differences or discontinuity designs should be the golden
standard of social science (Bueno de Mesquita and Fowler 2021). The
relative unmalleability of law in general, but rather conservative
institutions such as courts in particular, leaves little space for
experimental design and the general applicability of law within a legal
system leaves little space for quasi-experimental design. That is not to
say that it is impossible. Although we tried our best to think of and to
address all potential sources of bias and whether the assumptions of the
models of choice were met, we are aware of limitations of
regression-based research design and therefore our conclusions should be
taken with a grain of salt.

\vspace{30pt}

\hypertarget{literature}{%
\section*{Literature}\label{literature}}
\addcontentsline{toc}{section}{Literature}

\hypertarget{refs}{}
\begin{CSLReferences}{1}{0}
\leavevmode\vadjust pre{\hypertarget{ref-angristMostlyHarmlessEconometrics2009}{}}%
Angrist, Joshua D., and Jörn-Steffen Pischke. 2009. \emph{Mostly
{Harmless Econometrics}: {An Empiricist}'s {Companion}}. {Princeton
University Press}. \url{https://books.google.com?id=YSAzEAAAQBAJ}.

\leavevmode\vadjust pre{\hypertarget{ref-angristMasteringMetricsPath2014}{}}%
---------. 2014. \emph{Mastering '{Metrics}: {The Path} from {Cause} to
{Effect}}. {Princeton University Press}.

\leavevmode\vadjust pre{\hypertarget{ref-berdejoElectoralCyclesUS2017}{}}%
Berdejó, Carlos, and Daniel L. Chen. 2017. {``Electoral {Cycles} Among
{US Courts} of {Appeals Judges}.''} \emph{The Journal of Law and
Economics} 60 (3): 479--96. \url{https://doi.org/10.1086/696237}.

\leavevmode\vadjust pre{\hypertarget{ref-boydUntanglingCausalEffects2010}{}}%
Boyd, Christina L., Lee Epstein, and Andrew D. Martin. 2010.
{``Untangling the {Causal Effects} of {Sex} on {Judging}.''}
\emph{American Journal of Political Science} 54 (2): 389--411.
\url{https://www.jstor.org/stable/25652213}.

\leavevmode\vadjust pre{\hypertarget{ref-brekkeCJEUDatabasePlatform2023}{}}%
Brekke, Stein Arne, Joshua C. Fjelstul, Silje Synnøve Lyder Hermansen,
and Daniel Naurin. 2023. {``The {CJEU Database Platform}: {Decisions}
and {Decision-Makers}.''} \emph{Journal of Law and Courts}, January,
1--22. \url{https://doi.org/10.1017/jlc.2022.3}.

\leavevmode\vadjust pre{\hypertarget{ref-buenodemesquitaThinkingClearlyData2021}{}}%
Bueno de Mesquita, Ethan, and Anthony Fowler, eds. 2021. \emph{Thinking
Clearly with Data: A Guide to Quantitative Reasoning and Analysis}. 1st.
edition. {Princeton}: {Princeton University Press}.

\leavevmode\vadjust pre{\hypertarget{ref-carrubbaWhoControlsContent2012}{}}%
Carrubba, Cliff, Barry Friedman, Andrew D. Martin, and Georg Vanberg.
2012. {``Who {Controls} the {Content} of {Supreme Court Opinions}?''}
\emph{American Journal of Political Science} 56 (2): 400--412.
\url{https://doi.org/10.1111/j.1540-5907.2011.00557.x}.

\leavevmode\vadjust pre{\hypertarget{ref-chmelCoOvlivnujeUstavni2021}{}}%
Chmel, Jan. 2021. \emph{Co Ovlivňuje {Ústavní} Soud a Jeho Soudce? /}.
Vydání první. Teoretik ({Leges}). {Leges,}.

\leavevmode\vadjust pre{\hypertarget{ref-clarkLocatingSupremeCourt2010}{}}%
Clark, Tom S., and Benjamin Lauderdale. 2010. {``Locating {Supreme Court
Opinions} in {Doctrine Space}.''} \emph{American Journal of Political
Science} 54 (4): 871--90.
\url{https://doi.org/10.1111/j.1540-5907.2010.00470.x}.

\leavevmode\vadjust pre{\hypertarget{ref-dworkinPoliticalJudgesRule1980}{}}%
Dworkin, Ronald M. 1980. \emph{Political Judges and the Rule of Law}.
{London}: {British Academy}.

\leavevmode\vadjust pre{\hypertarget{ref-eliasekAutomatickaKlasifikaceVyznamovych2020}{}}%
Eliášek, Martin, Jakub Kól, and Miloš Švaňa. 2020. {``Automatická
Klasifikace Významových Celků v Judikatuře.''} \emph{Revue Pro Právo a
Technologie} 11 (21): 3--20. \url{https://doi.org/10.5817/RPT2020-1-1}.

\leavevmode\vadjust pre{\hypertarget{ref-epsteinStrategicRevolutionJudicial2000}{}}%
Epstein, Lee, and Jack Knight. 2000. {``Toward a {Strategic Revolution}
in {Judicial Politics}: {A Look Back}, {A Look Ahead}.''}
\emph{Political Research Quarterly} 53 (3): 625--61.
\url{https://doi.org/10.1177/106591290005300309}.

\leavevmode\vadjust pre{\hypertarget{ref-epsteinWhyWhenJudges2011}{}}%
Epstein, Lee, William M. Landes, and Richard A. Posner. 2011. {``Why
({And When}) {Judges Dissent}: {A Theoretical And Empirical
Analysis}.''} \emph{Journal of Legal Analysis} 3 (1): 101--37.
\url{https://doi.org/10.1093/jla/3.1.101}.

\leavevmode\vadjust pre{\hypertarget{ref-fjelstulHowChamberSystem2021}{}}%
Fjelstul, Joshua. 2021. {``How the {Chamber System} at the {CJEU
Undermines} the {Consistency} of the {Court}'s {Application} of {EU
Law}.''} \emph{Journal of Law and Courts}, November, 717422.
\url{https://doi.org/10.1086/717422}.

\leavevmode\vadjust pre{\hypertarget{ref-foxallWhatJudgesMaximize2004}{}}%
Foxall, Gordon R. 2004. {``What Judges Maximize:toward an Economic
Psychology of the Judicial Utility Function.''} \emph{Liverpool Law
Review} 25 (3): 177--94.
\url{https://doi.org/10.1007/s10991-004-2877-9}.

\leavevmode\vadjust pre{\hypertarget{ref-gandhiSupportVectorMachine2018}{}}%
Gandhi, Rohith. 2018. {``Support {Vector Machine} --- {Introduction} to
{Machine Learning Algorithms}.''} {Towards Data Science}. July 5, 2018.
\url{https://towardsdatascience.com/support-vector-machine-introduction-to-machine-learning-algorithms-934a444fca47}.

\leavevmode\vadjust pre{\hypertarget{ref-harastaAutomaticSegmentationCzech2019}{}}%
Harašta, Jakub, Jaromír Šavelka, František Kasl, and Jakub Míšek. 2019.
{``Automatic {Segmentation} of {Czech Court Decisions} into
{Multi-Paragraph Parts}.''} \emph{Jusletter IT} 4 (M).
\url{https://is.muni.cz/publication/1534440/cs/Automatic-Segmentation-of-Czech-Court-Decisions-into-Multi-Paragraph-Parts/Harasta-Savelka-Kasl-Misek}.

\leavevmode\vadjust pre{\hypertarget{ref-hyndmanTransformingDataZeros2010}{}}%
Hyndman, Rob. 2010. {``Transforming Data with Zeros.''} {Rob Hyndman}.
August 13, 2010.
\url{https://robjhyndman.com/hyndsight/transformations/}.

\leavevmode\vadjust pre{\hypertarget{ref-johnsonBayesRulesIntroduction2022}{}}%
Johnson, Alicia A., Miles Q. Ott, and Mine Dogucu. 2022. \emph{Bayes
{Rules}!: {An Introduction} to {Applied Bayesian Modeling}}. {CRC
Press}. \url{https://books.google.com?id=pISQzgEACAAJ}.

\leavevmode\vadjust pre{\hypertarget{ref-kastellecEmpiricallyEvaluatingCountermajoritarian2016}{}}%
Kastellec, Jonathan P. 2016. {``Empirically {Evaluating} the
{Countermajoritarian Difficulty}: {Public Opinion}, {State Policy}, and
{Judicial Review} Before {\emph{Roe}}{ \emph{v.} }{\emph{Wade}}.''}
\emph{Journal of Law and Courts} 4 (1): 1--42.
\url{https://doi.org/10.1086/683466}.

\leavevmode\vadjust pre{\hypertarget{ref-ludersProportionalityArgumentIdentification2023}{}}%
Lüders, Kilian, and Bent Stohlman. 2023. {``Proportionality as an
Argument: {Identification} of a Judicial Decision Technique.''}
\emph{DRAFT for 5th ANNUAL COMPTEXT Conference}.

\leavevmode\vadjust pre{\hypertarget{ref-maklinGradientBoostingDecision2019}{}}%
Maklin, Cory. 2019. {``Gradient {Boosting Decision Tree Algorithm
Explained}.''} {Towards Data Science}. July 21, 2019.
\url{https://towardsdatascience.com/machine-learning-part-18-boosting-algorithms-gradient-boosting-in-python-ef5ae6965be4}.

\leavevmode\vadjust pre{\hypertarget{ref-mikolovEfficientEstimationWord2013}{}}%
Mikolov, Tomas, Kai Chen, Greg Corrado, and Jeffrey Dean. 2013.
{``Efficient {Estimation} of {Word Representations} in {Vector
Space}.''} September 6, 2013.
\url{https://doi.org/10.48550/arXiv.1301.3781}.

\leavevmode\vadjust pre{\hypertarget{ref-montgomeryHowConditioningPosttreatment2018}{}}%
Montgomery, Jacob M., Brendan Nyhan, and Michelle Torres. 2018. {``How
{Conditioning} on {Posttreatment Variables Can Ruin Your Experiment} and
{What} to {Do} about {It}.''} \emph{American Journal of Political
Science} 62 (3): 760--75. \url{https://www.jstor.org/stable/26598780}.

\leavevmode\vadjust pre{\hypertarget{ref-moyerJudicialInnovationSexual2012}{}}%
Moyer, Laura P., and Holley Tankersley. 2012. {``Judicial {Innovation}
and {Sexual Harassment Doctrine} in the {U}.{S}. {Courts} of
{Appeals}.''} \emph{Political Research Quarterly} 65 (4): 784--98.
\url{https://doi.org/10.1177/1065912911411097}.

\leavevmode\vadjust pre{\hypertarget{ref-posnerWhatJudgesJustices1993}{}}%
Posner, Richard A. 1993. \emph{What {Do Judges} and {Justices
Maximize}?: (The {Same Thing Everyone Else Does})}. {Law School,
University of Chicago}. \url{https://books.google.com?id=ciFUHQAACAAJ}.

\leavevmode\vadjust pre{\hypertarget{ref-posnerHowJudgesThink2010}{}}%
---------. 2010. \emph{How {Judges Think}}. {Harvard University Press}.
\url{https://books.google.com?id=ZVUC8riEVPQC}.

\leavevmode\vadjust pre{\hypertarget{ref-smekalMimopravniVlivyNa2021}{}}%
Smekal, Hubert, Jaroslav Benák, Monika Hanych, Ladislav Vyhnánek, and
Štěpán Janků. 2021. \emph{Mimoprávní Vlivy Na Rozhodování Českého
{Ústavního} Soudu:} {Brno}: {Masaryk University Press}.
\url{https://doi.org/10.5817/CZ.MUNI.M210-9884-2021}.

\leavevmode\vadjust pre{\hypertarget{ref-vartazaryanSitOvaAnalyza2022}{}}%
Vartazaryan, Gor. 2022. {``Sít'ová Analỳza Disentujících Ústavních
Soudců.''} \emph{Pravnik}, no. 12.

\end{CSLReferences}

\end{document}
