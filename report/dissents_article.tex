% Options for packages loaded elsewhere
\PassOptionsToPackage{unicode}{hyperref}
\PassOptionsToPackage{hyphens}{url}
%
\documentclass[
]{article}
\usepackage{amsmath,amssymb}
\usepackage{lmodern}
\usepackage{iftex}
\ifPDFTeX
  \usepackage[T1]{fontenc}
  \usepackage[utf8]{inputenc}
  \usepackage{textcomp} % provide euro and other symbols
\else % if luatex or xetex
  \usepackage{unicode-math}
  \defaultfontfeatures{Scale=MatchLowercase}
  \defaultfontfeatures[\rmfamily]{Ligatures=TeX,Scale=1}
\fi
% Use upquote if available, for straight quotes in verbatim environments
\IfFileExists{upquote.sty}{\usepackage{upquote}}{}
\IfFileExists{microtype.sty}{% use microtype if available
  \usepackage[]{microtype}
  \UseMicrotypeSet[protrusion]{basicmath} % disable protrusion for tt fonts
}{}
\makeatletter
\@ifundefined{KOMAClassName}{% if non-KOMA class
  \IfFileExists{parskip.sty}{%
    \usepackage{parskip}
  }{% else
    \setlength{\parindent}{0pt}
    \setlength{\parskip}{6pt plus 2pt minus 1pt}}
}{% if KOMA class
  \KOMAoptions{parskip=half}}
\makeatother
\usepackage{xcolor}
\usepackage[margin=1in]{geometry}
\usepackage{graphicx}
\makeatletter
\def\maxwidth{\ifdim\Gin@nat@width>\linewidth\linewidth\else\Gin@nat@width\fi}
\def\maxheight{\ifdim\Gin@nat@height>\textheight\textheight\else\Gin@nat@height\fi}
\makeatother
% Scale images if necessary, so that they will not overflow the page
% margins by default, and it is still possible to overwrite the defaults
% using explicit options in \includegraphics[width, height, ...]{}
\setkeys{Gin}{width=\maxwidth,height=\maxheight,keepaspectratio}
% Set default figure placement to htbp
\makeatletter
\def\fps@figure{htbp}
\makeatother
\setlength{\emergencystretch}{3em} % prevent overfull lines
\providecommand{\tightlist}{%
  \setlength{\itemsep}{0pt}\setlength{\parskip}{0pt}}
\setcounter{secnumdepth}{-\maxdimen} % remove section numbering
\ifLuaTeX
  \usepackage{selnolig}  % disable illegal ligatures
\fi
\IfFileExists{bookmark.sty}{\usepackage{bookmark}}{\usepackage{hyperref}}
\IfFileExists{xurl.sty}{\usepackage{xurl}}{} % add URL line breaks if available
\urlstyle{same} % disable monospaced font for URLs
\hypersetup{
  pdftitle={dissents\_article},
  pdfauthor={Štěpán Paulík, Gor},
  hidelinks,
  pdfcreator={LaTeX via pandoc}}

\title{dissents\_article}
\author{Štěpán Paulík, Gor}
\date{2023-05-06}

\begin{document}
\maketitle

In our article, we replicate the study of Epstein, Landes and Posner.
The authors build a model of judicial dissents for the Supreme Court of
the USA and test various hypotheses. They test what motivates judges to
dissent, i.e., whether judges behave strategically, under what
circumstances, i.e., how does the decision to dissent depend on the
composition of panel, and lastly what is the impact of the dissent on
the majority opinion, i.e., whether the majority decision are longer as
a result of having to address arguments raised in a dissent.

Epstein et al.~build a model of judicial behavior for a Supreme Court of
the USA. Their model is based on the strategic behavior of judges. Given
that judges enjoy a life tenure, as is the case both at the SCOTUS and
at the CCC, which we analyse in our article, judges have ``leisure
preferences or, equivalently, effort aversion, which they trade off
against their desire to have a good reputation and to express their
legal and policy beliefs and preferences by their vote and by the
judicial opinion explaining their vote (\ldots).'' In their model, they
include not only the decision to dissent but also a dissent aversion, a
phenomenon that causes judges to not dissent even if they disagree with
the majority opinion.

The authors conduct the study on federal courts of appeal and the
SCOTUS. Due to availability of data and the fact that dissenting is a
common practice only at the CCC, we narrow our object of analysis to the
CCC. Moreover, Epstein, Posner and Landes include in their analysis the
ideological distance between judges. The ideological distance serves as
one of the explanatory variables for dissent aversion. The measures of
ideological position of judges mainly rely on information about their
voting behavior. Regrettably such an information is typically in
continental legal systems not made public: the votes in cases are kept
hidden from public. Therefore, it is near impossible to construct a
measure of political position of judges without knowing how they voted
in each case.

We believe that we can, nonetheless, test the remaining hypotheses from
the research paper. Thus, the hypotheses we will test are as follow:

\$ (1) \$ \$ (2) \$

\hypertarget{length-of-decision}{%
\subsection{Length of decision}\label{length-of-decision}}

Judicial decisions hardly ever follow a uniform structure. Nor do they
contain uniform title signalling which segment of the decision ensues
after the title. Therefore, we utilised supervised machine learning
model We trained a classification model that segmented a CCC decision
into structural parts, including dissent.

\end{document}
